%%%%%%%%%%
% PACKAGES (DEFAULT)

\usepackage{algorithm}					% algorithms
\usepackage[noend]{algpseudocode}		% "
\usepackage{amsmath}					% math environments and symbols
\usepackage{amssymb}					% "
\usepackage{amsthm}					% "
\usepackage{caption, subcaption}			% captions
\usepackage[noautomatic]{imakeidx}			% indexing
\usepackage{natbib}						% bibliography formatting
\usepackage[dvipsnames]{xcolor}			% colours

%%%%%%%%%%
% PACKAGES (CUSTOM)

\usepackage{bm}						% bold math symbols
\usepackage{booktabs}					% better tabular design
\usepackage{cancel}						% command to cancel/"slash out" symbols
\usepackage[scale=1.3]{ccicons}			% Creative Commons licence icons
\usepackage{complexity}					% complexity theory classes
\usepackage{enumitem}					% list spacing customization
\usepackage{listings}					% source code listings
\usepackage{mathtools}					% \xRightarrow command
\usepackage{multicol}					% multiple columns
\usepackage{qtree}						% drawing trees
\usepackage{rotating}					% rotated figures/pages
\usepackage{soul}						% strikeout and underline commands
\usepackage{tipa}						% IPA symbols
\usepackage[hyphens]{url}				% URL formatting

\usepackage{array}						% tabular customization
\usepackage{arydshln}					% dashed lines in tabular
\newcolumntype{L}[1]{>{\raggedright\let\newline\\\arraybackslash\hspace{0pt}}m{#1}}
\newcolumntype{C}[1]{>{\centering\let\newline\\\arraybackslash\hspace{0pt}}m{#1}}
\newcolumntype{R}[1]{>{\raggedleft\let\newline\\\arraybackslash\hspace{0pt}}m{#1}}
\newcommand{\linebreakcell}[3][t]{\begin{tabular}[#1]{@{}#2@{}}#3\end{tabular}}

%%%%%%%%%%
% GRAPHICS PACKAGES (DEFAULT)

\usepackage{tikz}						% vector graphics
\usetikzlibrary{calc}						% "

%%%%%%%%%%
% GRAPHICS PACKAGES (CUSTOM)

\usetikzlibrary{arrows.meta}				% 
\usetikzlibrary{automata}					% 
\usetikzlibrary{decorations.pathmorphing}		% 
\usetikzlibrary{graphs}					% 
\usetikzlibrary{graphs.standard}				% 
\usetikzlibrary{patterns}					% 
\usetikzlibrary{positioning}					% 
\usetikzlibrary{shapes}					% 
\usetikzlibrary{shapes.multipart}			% 
\usetikzlibrary{shapes.symbols}				% 
\usetikzlibrary{trees}						% 

%%%%%%%%%%
% HYPERLINK AND BOOKMARK PACKAGES

\ifdraft
	\usepackage[backref=page]{hyperref}	%
\else
	\usepackage{hyperref}				% 
\fi
\usepackage{bookmark}					% 

%%%%%%%%%%
% DRAFT MODE

\ifdraft
	\usepackage[%						% add "DRAFT" watermark
		angle=45,%
		color=lightgray!15!white,%
		pos={0.5\paperwidth,0.5\paperheight},%
		scale=6,%
		text=\normalfont\textsf{DRAFT}%
	]{draftwatermark}
	\renewcommand*{\backrefalt}[4]{%		% add page backreferences to bibliography
		\ifcase #1 %
		{\footnotesize No citations.}%
		\or
		{\footnotesize (cit.\ on p.\ #2).}%
		\else
		{\footnotesize (cit.\ on pp.\ #2).}%
		\fi
	}
\fi

%%%%%%%%%%
% PRINT MODE

\hypersetup{hypertexnames=true, linktocpage=true}
\ifprint
	\hypersetup{hidelinks, linktocpage=false}
\else
	\hypersetup{colorlinks}
\fi

%%%%%%%%%%
% AMSTHM ENVIRONMENTS

\theoremstyle{plain}						% AMS theorem styles
\newtheorem{theorem}{Theorem}
\newtheorem{lemma}[theorem]{Lemma}
\newtheorem{proposition}[theorem]{Proposition}
\newtheorem{corollary}[theorem]{Corollary}
\newtheorem{conjecture}[theorem]{Conjecture}

\theoremstyle{definition}					% AMS definition styles
\newtheorem{definition}[theorem]{Definition}
\newtheorem{example}[theorem]{Example}

\ifstyle
\else
	\theoremstyle{remark}				% AMS remark styles
	\newtheorem*{remark}{Remark}
	\newtheorem*{note}{Note}
\fi

%%%%%%%%%%
% INDEXING

\makeindex

%%%%%%%%%%
% CUSTOM COMMANDS/ENVIRONMENTS

% commands for advanced ("starred") sections/subsections
\makeatletter
\newcommand\advancedsection[2][section]{
	\def\@seccntformat##1{\protect\llap{\large*}\csname the##1\endcsname\quad}%
	\addtocontents{toc}{%
		\let\protect\mtnumberline\protect\numberline%
		\def\protect\numberline{%
			\hskip0pt%
			\global\let\protect\numberline\protect\mtnumberline%
			\protect\llap{\normalfont\normalsize*}%
			\protect\mtnumberline%
		}%
	}%
	\csname #1\endcsname{#2}%
	\def\@seccntformat##1{\csname the##1\endcsname\quad}%
}
\newcommand\advancedsubsection[2][subsection]{\advancedsection[#1]{#2}}
\makeatother

% under construction environment
\newenvironment{construction}{%
	\par\medskip\noindent
	\sbox0{\raisebox{-1.4em}{\includegraphics{./images/bc-aux-301.pdf}}\space}%
	\hangindent\wd0
	\parindent\hangindent
	\hangafter=-2
	\smash{\llap{\box0}}%
	\slshape
	\ignorespaces
}{\par\medskip}

% dangerous bend environment
\newenvironment{dangerous}{%
	\par\medskip\noindent
	\sbox0{\raisebox{-1.4em}{\includegraphics{./images/dbend.pdf}}\space}%
	\hangindent\wd0
	\parindent\hangindent
	\hangafter=-2
	\smash{\llap{\box0}}%
	\ignorespaces
}{\par\medskip}

% command for epigraphs
\newcommand{\epigraph}[4]{
	\null
	\hfill
	\begin{minipage}{0.85\textwidth}
	\begin{flushright} 
	{\setlength{\parskip}{-2pt} \itshape #1} \\
	\if\relax\detokenize{#4}\relax
		--- {\scshape #2}
	\else
		--- {\scshape #2 (#4)}
	\fi
	\end{flushright}
	\end{minipage}
	\par
}

% command for future sections/subsections/subsubsections
\newcommand{\futuresection}[1]{
	\section{\texorpdfstring{#1 $\diamond$}{#1}}
}
\newcommand{\futuresubsection}[1]{
	\subsection{\texorpdfstring{#1 $\diamond$}{#1}}
}
\newcommand{\futuresubsubsection}[1]{
	\subsubsection{\texorpdfstring{#1 $\diamond$}{#1}}
}

% command for unnumbered sections (listed in ToC)
\newcommand{\unnumberedsection}[1]{
	\phantomsection
	\section*{#1}
	\addcontentsline{toc}{section}{#1}
	\markright{#1}
}

%%%%%%%%%%
% CUSTOM COMMANDS/ENVIRONMENTS
% (only when book style file is not included)

% custom colours
\newcommand{\maincolour}{black}
\newcommand{\secondcolour}{\maincolour!75}
\newcommand{\thirdcolour}{\maincolour!45}
\newcommand{\fourthcolour}{\maincolour!15}
\newcommand{\fifthcolour}{\maincolour!5}
\newcommand{\neutralcolour}{gray}

\ifstyle
\else
	% algorithm styling
	\captionsetup[algorithm]{labelsep=colon}
	\algrenewcommand\algorithmicrequire{\textbf{Input:}}
	\algrenewcommand\algorithmicensure{\textbf{Output:}}

	% custom environments
	\newcommand{\firstwords}[1]{#1}
	\newenvironment{algorithmbox}[1][.]{\begin{algorithm}\begin{algorithmic}\caption{#1}}{\end{algorithmic}\end{algorithm}}
	\newenvironment{colouredbox}{}{}
	\newenvironment{customtitlebox}[1][.]{\medskip\noindent\textbf{#1.}}{\medskip}
	\newenvironment{exercises}{\begin{enumerate}}{\end{enumerate}}
	\newenvironment{proofbox}[1][.]{\medskip\noindent\textit{Proof of #1.}}{\medskip}
	\newcommand{\tmalgorithm}[3][1.0]{#3}
\fi

%%%%%%%%%%
% OTHER CUSTOM COMMANDS/ENVIRONMENTS

% general mathematical notation
\newcommand{\from}{\colon}								% function notation, corresponds with \to
\newcommand{\bigdot}[1]{\overset{\mbox{\large\bfseries .}}{#1}}	% big version of dot to go over symbols
\newcommand{\symdiff}{\mathbin{\triangle}}					% symmetric difference of two sets

% notation specific to regular languages/models
\newcommand{\regex}[1]{\ensuremath{\bm #1}}
\renewcommand{\RE}{\textsf{RE}}
\newcommand{\DFA}{\textsf{DFA}}
\newcommand{\NFA}{\textsf{NFA}}
\mathchardef\mhyphen="2D
\newcommand{\ENFA}{\ensuremath{\epsilon\mhyphen\textsf{NFA}}}

% notation specific to context-free languages/models
\newcommand{\CFG}{\textsf{CFG}}
\newcommand{\PDA}{\textsf{PDA}}

% notation specific to Turing machines and decidable/semidecidable languages
\renewcommand{\D}{\textsf{D}}
\newcommand{\SD}{\textsf{SD}}
\newcommand{\TM}{\textsf{TM}}
\def\blankspace{\leavevmode\hbox{\tt\char`\ }}					% visible space

% notation specific to reductions
\newcommand{\mreducesto}{\leq_{\mathrm{m}}}

% notation specific to complexity classes
\newclass{\NEXPSPACE}{NEXPSPACE}

% command to highlight mathematical text
\newcommand{\highlightmath}[1]{\colorbox{\fourthcolour}{\color{\maincolour}$\mathstrut\displaystyle#1$}}

% commands to draw illustration of film reel
\newcommand\rulebox[1]{%
	\begingroup
	\fontsize{100}{5}\selectfont
	\fboxsep0.1pt%
	\colorbox{white}{\raisebox{0pt}[0.85cm][0.25cm]{\makebox[2.2cm][c]{\textcolor{black}{#1}}}}%
	\endgroup
}
\newcommand{\whitebox}{\hfill\textcolor{white}{\rule[1mm]{1.4mm}{2.8mm}}\hfill}
\newcommand{\whiteboxlarge}[1]{\hfill\rulebox{#1}\hfill}
\newcommand{\filmbox}[1]{%
	\setlength{\fboxsep}{0pt}%
	\colorbox{black}{%
	\begin{minipage}{2.4cm}
		\rule{0mm}{4.8mm}\whitebox\whitebox\whitebox\whitebox\whitebox%
		\whitebox\whitebox\whitebox\whitebox\null\\%
		\null\hfill\whiteboxlarge{\texttt{#1}}\hfill\null\\[1mm]%
		\null\whitebox\whitebox\whitebox\whitebox\whitebox%
		\whitebox\whitebox\whitebox\whitebox\null
	\end{minipage}}}

% commands to set style of code listings
\lstdefinestyle{codeblock}{
	% How/what to match
	sensitive=true,
	% Border (above and below)
	frame=lines,
	% Extra margin on line (align with paragraph)
	xleftmargin=\parindent,
	% Put extra space under caption
	belowcaptionskip=1\baselineskip,
	% Colors
	backgroundcolor=\color{white},
	basicstyle=\color{black}\ttfamily,
	keywordstyle=\color{gray},
	commentstyle=\color{gray},
	stringstyle=\color{gray},
	numberstyle=\color{gray},
	identifierstyle=\color{gray},
	% Break long lines into multiple lines?
	breaklines=true,
	% Show a character for spaces?
	showstringspaces=false,
	tabsize=2,
	escapechar={!},
}

% command to add vertical strikethrough to character
% (used specifically in CPL code example)
\newcommand*{\vertchar}[2][0pt]{%
	\tikz[
		inner sep=0pt,
		shorten >=-.15ex,
		shorten <=-.15ex,
		line cap=round,
		baseline=(c.base),
	]\draw
	(0,0) node (c) {#2}
	($(c.south)+(#1,0)$) -- ($(c.north)+(#1,0)$);%
}