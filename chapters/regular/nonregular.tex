\section{Proving a Language is Nonregular}\label{sec:nonregular}

\firstwords{At this point}, it should be evident that finite automata and regular expressions are nice models to use when discussing computation in the abstract. They're easy to define, easy to reason about, and they have a lot of nice properties that we can use in proofs. However, they are not the be-all and end-all of theoretical computer science. (Otherwise, this would be a rather short course!)

Both finite automata and regular expressions suffer the drawback of not having any way to store or recall data. Finite automata don't have any storage mechanism, and regular expressions don't allow for lookback. As we said in the section introducing finite automata, once the finite automaton reads a symbol and transitions to a state, it can never return to that symbol. For all intents and purposes, the symbol is lost forever, and the finite automaton doesn't even remember having read it. Likewise, once a regular expression matches a symbol in a word and moves on to the next symbol, it has no way of remembering any previous symbols that were matched.

Naturally, this means that there exist some languages that cannot be recognized by a finite automaton (or, equivalently, represented by a regular expression), and therefore such languages cannot be regular. For instance, this is the canonical example of a language that no finite automaton can recognize:
\begin{equation*}
L_{\text{a}=\text{b}} = \{\texttt{a}^{n}\texttt{b}^{n} \mid n \geq 0\}.
\end{equation*}
In this language, every word has an equal number of \texttt{a}s and \texttt{b}s, and all occurrences of \texttt{a} appear before the first occurrence of \texttt{b}. Some examples of words in this language are \texttt{ab}, \texttt{aaabbb}, \texttt{aaaaaabbbbbb}, and $\epsilon$.

Why can't any finite automaton recognize this language? Because of that word \emph{finite}. A finite automaton consists of a finite number of states, but in order to recognize this language, we would need to add a ``chain" consisting of $2n + 1$ states to accept the word $\texttt{a}^{n}\texttt{b}^{n}$ for every $n \geq 0$; such a construction is illustrated in Figure~\ref{fig:abequalautomaton}. Since $n$ has no upper bound, we would need an infinite number of states! No finite automaton can recognize this language, because no finite automaton has a way of keeping track of the value $n$ or counting the symbols using only a finite number of states.

\begin{figure}
\centering
\begin{tikzpicture}[node distance=1.75cm, >=latex, every state/.style={fill=white}]
\node[state, initial, accepting] (q0) {$q_{0}$};
\node[state] (q1) [right of=q0] {$q_{1}$};
\node[state, accepting] (q2) [below of=q1] {$q_{2}$};
\node[state, draw=black!80] (q3) [right of=q1] {\color{black!80}$q_{3}$};
\node[state, draw=black!80] (q4) [below of=q3] {\color{black!80}$q_{4}$};
\node[state, draw=black!60] (q5) [right of=q3] {\color{black!60}$q_{5}$};
\node[state, draw=black!60] (q6) [below of=q5] {\color{black!60}$q_{6}$};
\node[state, draw=black!40] (q7) [right of=q5] {\color{black!40}$q_{7}$};
\node[state, draw=black!40] (q8) [below of=q7] {\color{black!40}$q_{8}$};
\node[state, draw=none, fill=none, right of=q7] (qi) {\color{black!20}$\dots$};
\node[state, draw=none, fill=none, right of=q8] (qj) {\color{black!20}$\dots$};

\path[-latex] (q0) edge [above] node {\texttt{a}} (q1);
\path[-latex] (q1) edge [left] node {\texttt{b}} (q2);

\path[-latex] (q1) edge [above] node {\texttt{a}} (q3);
\path[-latex, color=black!80] (q3) edge [left] node {\texttt{b}} (q4);
\path[-latex] (q4) edge [below] node {\texttt{b}} (q2);

\path[-latex, color=black!80] (q3) edge [above] node {\texttt{a}} (q5);
\path[-latex, color=black!60] (q5) edge [left] node {\texttt{b}} (q6);
\path[-latex, color=black!80] (q6) edge [below] node {\texttt{b}} (q4);

\path[-latex, color=black!60] (q5) edge [above] node {\texttt{a}} (q7);
\path[-latex, color=black!40] (q7) edge [left] node {\texttt{b}} (q8);
\path[-latex, color=black!60] (q8) edge [below] node {\texttt{b}} (q6);

\path[-latex, color=black!40] (q7) edge [above] node {\texttt{a}} (qi);
\path[-latex, color=black!40] (qj) edge [below] node {\texttt{b}} (q8);
\end{tikzpicture}
\caption{A ``finite" automaton supposedly recognizing the language $L_{\text{a}=\text{b}}$}
\label{fig:abequalautomaton}
\end{figure}

However, we can't totally rely on the claim that a finite automaton is incapable of recognizing a language if it has to count symbols. For instance, consider the language
\begin{equation*}
L_{\text{a}} = \{\texttt{a}^{n} \mid n \geq 0\}.
\end{equation*}
This language contains an infinite number of words: one word for each $n \geq 0$, exactly like in $L_{\text{a}=\text{b}}$. But it's easy for a finite automaton to recognize $L_{\text{a}}$, and using only one state!
\begin{center}
\begin{tikzpicture}[node distance=2cm, >=latex, every state/.style={fill=white}]
\node[state, initial, accepting] (q0) {$q_{0}$};
\path[-latex] (q0) edge [loop right] node {\texttt{a}} (q0);
\end{tikzpicture}
\end{center}

Thus, it should hopefully be clear that we need to take a slightly more intricate approach in order to prove a language is not regular. There are many more nonregular languages than there are regular languages, so instead of focusing on some sort of property that a nonregular language might have, let's instead find a property every regular language must have. We can then prove a language is nonregular by showing that the language \emph{doesn't} have that property.

\subsection{The Pumping Lemma for Regular Languages}

The property of regular languages that we will make use of is the following: for every regular language, if we take a word in the language of sufficient length, then we can repeat (or \emph{pump}) a middle portion of that word an arbitrary number of times and produce a new word that belongs to the same regular language. This fact is known, appropriately enough, as the \emph{pumping lemma} for regular languages.

\begin{lemma}[Pumping lemma for regular languages]\label{lem:pumpingregular}
For all regular languages $L$, there exists $p \geq 1$ where, for all $w \in L$ with $|w| \geq p$, there exists a decomposition of $w$ into three parts $w = xyz$ such that
\begin{enumerate}
\item $|y| > 0$;
\item $|xy| \leq p$; and
\item for all $i \geq 0$, $xy^{i}z \in L$.
\end{enumerate}
\end{lemma}

Clearly, the pumping lemma contains a lot of notation and terminology to take in at once---not to mention four alternating quantifiers in a row! Let's take a closer look at the lemma from three different perspectives.

\subsubsection*{An Informal Description}

We'll begin by breaking the pumping lemma down piece-by-piece to see what it tells us.
\begin{colouredbox}
\begin{itemize}
\item \textbf{\textit{For all regular languages $\bm{L}$,}} \\ We can take any regular language $L$, and it will satisfy the pumping lemma.
\item \textbf{\textit{there exists $\bm{p \geq 1}$}} \\ Depending on the language $L$, the pumping lemma claims that there exists a constant $p$ for that language. We call $p$ the \emph{pumping constant}.

(If you're curious, $p$ is the number of states in the finite automaton recognizing $L$.)
\item \textbf{\textit{where, for all $\bm{w \in L}$ with $\bm{|w| \geq p}$,}} \\ We can take any word from $L$ with length at least $p$, and it will satisfy the pumping lemma.
\item \textbf{\textit{there exists a decomposition of $\bm{w}$ into three parts $\bm{w = xyz}$}} \\ Depending on the word $w$, the pumping lemma claims that $w$ can be split into three parts, labelled $x$, $y$, and $z$. The $y$ part is what we will use to do the pumping; the $x$ and $z$ parts are just the start and end parts of $w$ that don't get pumped.
\item \textbf{\textit{such that 1.\ $\bm{|y| > 0}$;}} \\ This condition ensures that the $y$ part of $w$ is nonempty, so that we have something to pump.
\item \textbf{\textit{2.\ $\bm{|xy| \leq p}$;}} \\ This condition ensures that there exists some state in the finite automaton recognizing $L$ that is visited more than once, and furthermore, we will visit that state during the computation before we finish reading the part $y$.

(This condition is essentially an application of the pigeonhole principle.)
\item \textbf{\textit{and 3.\ for all $\bm{i \geq 0}$, $\bm{xy^{i}z \in L}$.}} \\ This is the actual pumping part of the pumping lemma. This condition ensures that, no matter how many copies of the $y$ part we include in our word (even zero copies), the resulting word will still belong to the language.
\end{itemize}
\end{colouredbox}

\subsubsection*{A Formal Proof}

Now that we have a greater understanding of what the pumping lemma says, let's take a look at the proof of the lemma. Remember, the property of all regular languages that we're relying on is that if we take a word of sufficient length from the language, then we can pump a middle portion of that word arbitrarily many times and always obtain a word that still belongs to the language. This means that if we consider a finite automaton recognizing that language, there must exist a loop somewhere within that automaton.

\begin{proofbox}[Lemma~\ref{lem:pumpingregular}]
Let $\mathcal{M} = (Q, \Sigma, \delta, q_{0}, F)$ be a deterministic finite automaton recognizing the language $L$, and let $p$ denote the number of states of $\mathcal{M}$.

Take a word $w = w_{1}w_{2} \dots w_{n}$ of length $n$ from $L$, where $n \geq p$, and let $r_{1}, \dots, r_{n+1}$ be the accepting computation of $\mathcal{M}$ on $w$. Specifically, let $r_{i+1} = \delta(r_{i}, w_{i})$ for all $1 \leq i \leq n$. Clearly, this accepting computation has length $n+1 \geq p+1$.

By the pigeonhole principle, there must exist at least two states in the first $p+1$ states of the accepting computation that are the same. Say that the first occurrence of the same state is $r_{j}$ and the second occurrence is $r_{\ell}$. Since $r_{\ell}$ occurs within the first $p+1$ states of the accepting computation, we know that $\ell \leq p+1$.

Decompose the word $w$ into parts $x = w_{1} \dots w_{j-1}$, $y = w_{j} \dots w_{\ell-1}$, and $z = w_{\ell} \dots w_{n}$. As the part $x$ is read, $\mathcal{M}$ transitions from state $r_{1}$ to state $r_{j}$. Likewise, as $y$ is read, $\mathcal{M}$ transitions from $r_{j}$ to $r_{j}$, and as $z$ is read, $\mathcal{M}$ transitions from $r_{j}$ to $r_{n+1}$. Since we are considering an accepting computation, $r_{n+1}$ is a final state, and so $\mathcal{M}$ must accept the word $xy^{i}z$ for all $i \geq 0$. Moreover, we know that $j \neq \ell$, so $|y| > 0$. Lastly, since $\ell \leq p+1$, we have that $|xy| \leq p$. Therefore, all three conditions of the pumping lemma are satisfied.
\end{proofbox}

\begin{figure}[b]
\centering
\begin{tikzpicture}[node distance=2cm, >=latex, every state/.style={fill=white}, decoration={%
      snake,
      segment length=2mm,
      amplitude=0.4mm,
      pre length=4pt,
      post length=4pt,
    }]
\node[state, initial, minimum size=1cm] (q0) {$r_{0}$};
\node[state, minimum size=1cm] (q1) [right of=q0] {$r_{j}$};
\node[state] (qn) [above=0.75cm of q1, draw=none] {$y$};
\node[state, accepting, minimum size=1cm] (q2) [right of=q1] {$r_{n+1}$};

\path[-latex, draw=black, decorate] (q0) -- node[below, yshift=-0.2cm] {$x$} (q1);
\path[-latex, draw=black, decorate] (q1) to[in=50,out=130,looseness=6] (q1);
\path[-latex, draw=black, decorate] (q1) -- node[below, yshift=-0.2cm] {$z$} (q2);
\end{tikzpicture}
\caption{The pumping lemma for regular languages, presented diagrammatically}
\label{fig:pumpinglemma}
\end{figure}

Diagrammatically, this proof can be approximated by Figure~\ref{fig:pumpinglemma}. The wavy transition lines in the figure denote some chain of transitions starting at one state and ending at another state, where we don't care about the states in between. From the figure, we can see that all of the states of the finite automaton between $r_{0}$ and $r_{j}$ are used to read the part $x$, all of the states between $r_{j}$ and $r_{n+1}$ are used to read the part $z$, and there exists a loop of states that both starts and ends with $r_{j}$ that is used to read the part $y$. We can take this loop as many times as we want while reading the input word, and taking one journey around the loop corresponds to ``pumping" the word once.

\subsubsection*{A Fun Game}

Alternatively, we can think of the pumping lemma as an adversarial game, where we're trying to show that some language $L$ is nonregular while our opponent is trying to show that $L$ is, in fact, regular. If we win the game, then $L$ is nonregular, while if our opponent wins, then $L$ is regular. The rules of this game are given in Figure~\ref{fig:pumpinglemmagame}, so that you can play it at the next party you attend.

\begin{figure}
\centering
{\footnotesize
\setlength{\fboxsep}{10pt}
\noindent\fbox{%
\parbox{0.9\textwidth}{%
\vspace{-0.75em}
\begin{center}
\textbf{Rules of the Pumping Lemma Game}
\end{center}
\begin{enumerate}[leftmargin=*]
\item \ul{Your opponent} chooses $p \geq 1$ and claims it is the pumping constant for $L$.
\item \ul{You} choose a word $w \in L$ with $|w| \geq p$ and claim this word can't be decomposed into parts $w = xyz$ that satisfy the three conditions of the pumping lemma.
\item \ul{Your opponent} chooses a decomposition $w = xyz$ such that $|y| \geq 0$ and $|xy| \leq p$, satisfying the first two conditions automatically, and claims that this decomposition will also satisfy the third condition.
\item \ul{You} choose $i \geq 0$ such that $xy^{i}z \not\in L$.
\end{enumerate}
If you complete Step 4, then you win the game!

If you can't find any $i \geq 0$ in Step 4, then you lose the game.

If any of the claims in Steps 1--3 are false, then the person who made the claim loses the game.
}
}
}
\caption{The pumping lemma game}
\label{fig:pumpinglemmagame}
\end{figure}

\subsubsection*{Using the Pumping Lemma}

As we noted, every regular language must satisfy the pumping lemma, and so any language that does not satisfy the pumping lemma must not be regular. This means that we can use the lemma to prove a language is nonregular by contradiction: assuming the language were regular, it should satisfy the pumping lemma, but if we can somehow pump a sufficently long word to produce a word that does \emph{not} belong to the language, our assumption of regularity must not hold.

Even though the pumping lemma looks complex, reducing it to a series of steps as we did here reveals that any proof of the nonregularity of a language simply has to follow each of the steps. As a result, all nonregularity proofs tend to share a similar structure.

Let's take a look at an example of a pumping lemma proof using our canonical nonregular language, $L_{\text{a}=\text{b}}$.

\begin{example}
Let $\Sigma = \{\texttt{a}, \texttt{b}\}$, and consider the language $L_{\text{a}=\text{b}} = \{\texttt{a}^{n}\texttt{b}^{n} \mid n \geq 0\}$. We will use the pumping lemma to show that this language is nonregular.

Assume by way of contradiction that the language is regular, and let $p$ denote the pumping constant given by the pumping lemma. We choose the word $w = \texttt{a}^{p}\texttt{b}^{p}$. Clearly, $w \in L_{\text{a}=\text{b}}$ and $|w| \geq p$. Thus, there exists a decomposition $w = xyz$ satisfying the three conditions of the pumping lemma.

We consider three cases, depending on the contents of the part $y$ of the word $w$:
\begin{enumerate}
\item The part $y$ contains only \texttt{a}s. In this case, pumping $y$ once to obtain the word $xy^{2}z$ results in the word containing more \texttt{a}s than \texttt{b}s, and so $xy^{2}z \not\in L_{\text{a}=\text{b}}$. This violates the third condition of the pumping lemma.

\item The part $y$ contains only \texttt{b}s. In this case, since the first $p$ symbols of $w$ are \texttt{a}s, we must have that $|xy| > p$. This violates the second condition of the pumping lemma.

\item The part $y$ contains both \texttt{a}s and \texttt{b}s. Again, in this case, since the first $p$ symbols of $w$ are \texttt{a}s, we must have that $|xy| > p$. This violates the second condition of the pumping lemma.
\end{enumerate}
In all cases, one of the conditions of the pumping lemma is violated. As a consequence, the language cannot be regular.
\end{example}

Let's step through each component of this proof. We began by assuming our language was regular. From this assumption, the pumping lemma tells us that there exists some value $p \geq 1$ that we can use in our next step: choosing an appropriate word from the language. We choose a word $w$ that is sufficiently long; that is, of length at least $p$. (We chose $\texttt{a}^{p}\texttt{b}^{p}$ here, which makes for a good strategy: try to incorporate the value $p$ into the chosen word in some way.) The pumping lemma then tells us that, since our word $w$ is long enough, there exists some decomposition $w = xyz$ that satisfies the three conditions of the lemma. From here, the remainder of the proof consists of us checking every possible decomposition of $w$ and finding some violated condition for each decomposition.

\begin{dangerous}
Two of the most common mistakes when using the pumping lemma are fixing a specific value for $p$ and choosing a specific decomposition $w = xyz$. We must not do either of these! Fixing a specific value for $p$ is not allowed because the statement of the pumping lemma tells us only that \emph{there exists} a value $p$, not what this value is specifically. Likewise, choosing a specific decomposition is not allowed because the pumping lemma again tells us only that \emph{there exists} a decomposition that satisfies the three conditions. Observe that the example we just saw keeps things as general as possible: it doesn't fix a specific value for $p$, and it considers all possible decompositions before arriving at a conclusion.
\end{dangerous}

A language doesn't necessarily have to count symbols in order to be nonregular. Since finite automata don't have any form of storage, they can't remember symbols they read earlier in an input word. This means that finite automata can't recall parts of a word, and so they can't recognize languages like $L_{\text{double}} = \{ww \mid w \in \Sigma^{*}\}$. Here, we prove that a similar language is nonregular: the language of palindromes, $ww^{\text{R}}$. (The notation $w^{\text{R}}$ denotes the \emph{reversal} of the word $w$.) Palindromes are words that read the same backward as they do forward.

\begin{example}
Let $\Sigma = \{\texttt{a}, \texttt{b}\}$, and consider the language $L_{\text{pal}} = \{ ww^{\text{R}} \mid w \in \Sigma^{*}\}$. We will use the pumping lemma to show that this language is nonregular.

Assume by way of contradiction that the language is regular, and let $p$ denote the pumping constant given by the pumping lemma. We choose the word $w = \texttt{a}^{p}\texttt{bb}\texttt{a}^{p}$. Clearly, $w \in L_{\text{pal}}$ and $|w| \geq p$. Thus, there exists a decomposition $w = xyz$ satisfying the three conditions of the pumping lemma.

Since the second condition of the pumping lemma tells us that $|xy| \leq p$, it must be the case that, in any decomposition, we have $xy = \texttt{a}^{k}$ for some $k \leq p$. Consequently, we have $y = \texttt{a}^{\ell}$ for some $1 \leq \ell \leq k$.

If we pump $y$ once to obtain the word $xy^{2}z$, then we obtain the word $\texttt{a}^{p+\ell}\texttt{bb}\texttt{a}^{p}$, which is no longer a palindrome. This violates the third condition of the pumping lemma. As a consequence, the language cannot be regular.
\end{example}

Lastly, recall the third condition of the pumping lemma: for all $i \geq 0$, $xy^{i}z \in L$. The third condition allows us not only to pump \emph{up} by adding copies of $y$ to the word, but also to pump \emph{down} by removing $y$ from the word. In some cases, pumping down can help us to prove a language is nonregular.

\begin{example}
Let $\Sigma = \{\texttt{a}, \texttt{b}\}$, and consider the language $L_{\text{a}>\text{b}} = \{ \texttt{a}^{i}\texttt{b}^{j} \mid i > j\}$. We will use the pumping lemma to show that this language is nonregular.

Assume by way of contradiction that the language is regular, and let $p$ denote the pumping constant given by the pumping lemma. We choose the word $w = \texttt{a}^{p+1}\texttt{b}^{p}$. Clearly, $w \in L_{\text{a}>\text{b}}$ and $|w| \geq p$. Thus, there exists a decomposition $w = xyz$ satisfying the three conditions of the pumping lemma.

Since the second condition of the pumping lemma tells us that $|xy| \leq p$, it must be the case that, in any decomposition, we have $xy = \texttt{a}^{k}$ for some $k \leq p$. Consequently, we have $y = \texttt{a}^{\ell}$ for some $1 \leq \ell \leq k$.

If we pump $y$ one or more times, then we will always end up with a word that contains more \texttt{a}s than \texttt{b}s, and this word will always belong to the language $L_{\text{a}>\text{b}}$.

However, if we pump $y$ down to obtain the word $xy^{0}z = xz$, then our word will be of the form $\texttt{a}^{p+1-\ell}\texttt{b}^{p}$. Since $\ell \geq 1$, our resultant word has at most as many \texttt{a}s as \texttt{b}s, and so it no longer belongs to the language $L_{\text{a}>\text{b}}$. This violates the third condition of the pumping lemma. As a consequence, the language cannot be regular.
\end{example}

\futuresubsection{The Myhill--Nerode Theorem}

\begin{construction}
Following our discussion of the pumping lemma for regular languages, I will reveal that the pumping lemma doesn't actually work for all languages: there exist nonregular languages that satisfy the lemma's conditions. (See, e.g., \citet*{Ehrenfeucht1981PumpingLemmasRegularSets} and \citet{Johnsonbaugh1990ConversesPumpingLemmas}.) This is because the pumping lemma merely gives a necessary, but not sufficient, condition for regularity. To rectify this, I will talk about the Myhill--Nerode theorem, which gives both necessary and sufficient conditions for regularity.
\end{construction}