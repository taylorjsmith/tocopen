\chapter{Proving Undecidability}

\firstwords{Thus far, we only have} a couple of examples of decision problems that are undecidable, but these examples are big ones: the membership and non-membership problems for Turing machines. These particular problems being undecidable presents a huge issue for us, since the simple matter of testing membership of an input word in a Turing machine's language is fundamental to answering nearly any other question about the capabilities of a Turing machine.

We know that other undecidable languages exist---in fact, as we observed earlier, there are uncountably many such languages! And, as we might expect, many decision problems for Turing machines fall into this category. So how do we prove such problems truly are undecidable?

In the previous chapter, we combined the techniques of \emph{arithmetization} and \emph{diagonalization} to obtain our first undecidability result. Arithmetization is a fancy word that describes the process of encoding the behaviour of a Turing machine as a string of symbols; say, taking a \texttt{1} as the $i$th symbol if some Turing machine accepts the description of another Turing machine $\langle \mathcal{M}_{i} \rangle$ and \texttt{0} if it rejects the description. We then know from our discussion of diagonalization how these strings of symbols can be manipulated to produce a new string that leads to some absurd and undefined behaviour. But admittedly, undecidability proofs via diagonalization are lengthy and tedious. Is there some other (hopefully easier) way to prove that a decision problem is undecidable?

Enter the notion of \emph{reducibility}. A reduction is a way for us to relate the difficulty of one decision problem to the difficulty of another. If we know how to turn one problem into another, then we can use an existing Turing machine or algorithm for the second problem also to decide the first problem. On the other hand, if we know how to turn one problem into another and we know that the first problem cannot be decided by any Turing machine or algorithm, then we can conclude that it's hopeless for us to decide the second problem. Therein lies the other (easier) way for us to prove undecidability: if we can turn either of our known-undecidable problems $\mathit{A}_{\TM}$ or $\overline{\mathit{A}_{\TM}}$ into some new problem, then we have our proof!

If a reduction seems a bit abstract based on what you've read so far, consider that you encounter examples of reductions in your daily life without even realizing it. For example, if you've recently gone to the library to sign out a book you want to read, the problem of locating that book in the stacks is no more difficult than the problem of searching for that book in the library's catalog. If you know how to use the library's catalog, then you know how to acquire the classification number of the book and where to find it on the shelf.

In this chapter, we will use reductions to prove that many decision problems for Turing machines are undecidable, including one of the most famous problems in all of computer science. Each of our proofs of undecidability will have a very similar structure, highlighting the versatility of this approach over our more tedious and ad hoc diagonalization approach. We will then put a spin on the idea of reducibility in order to go one step further and prove that certain decision problems pertaining to context-free languages are likewise undecidable.

\section{Many-One Reductions}\label{sec:manyonereductions}

\firstwords{In our proof} showing that $A_{\TM}$ was undecidable, we constructed a Turing machine $\mathcal{X}$ that took as input $\langle \mathcal{M} \rangle$, the encoding of a Turing machine $\mathcal{M}$, converted that input to the form $\langle \mathcal{M}, \langle \mathcal{M} \rangle \rangle$, and gave that converted input to another Turing machine $\mathcal{M}_{\mathrm{A}}$. The machine $\mathcal{X}$ then used the output of $\mathcal{M}_{\mathrm{A}}$ to determine what its own output should be. This was the ad hoc diagonalization approach: what we wish to avoid going forward.

Generalizing this notion---that is, the notion of a Turing machine taking an input corresponding to some decision problem $X$, converting it into an input corresponding to some other decision problem $Y$, and then giving that converted input to another Turing machine for $Y$---gives us the foundation for proving undecidability using \emph{many-one reductions} or, more generally, just \emph{reductions}.

\begin{remark}
There exist other kinds of reductions that we will study later in this book. In this chapter, though, we will only consider many-one reductions, so we will often use the word ``reduction" here as a shorthand to refer to many-one reductions.
\end{remark}

Computationally speaking, a many-one reduction is the middle step in our aforementioned generalization that converts the instance of the problem $X$ into an instance of the other problem $Y$. The reason why this step is called a ``many-one" reduction is because the conversion process is performed by a computable function, to which we were introduced in Section~\ref{subsec:computingfunctions}. If a computable function exists that gets us from $X$ to $Y$, then we say that $X$ is \emph{many-one reducible} (or just \emph{reducible}) to $Y$.

\begin{definition}[Many-one reduction]\label{def:manyonereduction}
Given two decision problems $X$ and $Y$ over alphabets $\Sigma$ and $\Gamma$, respectively, problem $X$ is many-one reducible to problem $Y$ if there exists a computable function $f \from \Sigma^{*} \to \Gamma^{*}$ where, for all $w \in \Sigma^{*}$, $w \in X$ if and only if $f(w) \in Y$.
\end{definition}

Having a reduction tells us not only that we can transform every instance $w$ of $X$ into an instance $f(w)$ of $Y$, but also that we \emph{cannot} transform non-instances of $X$ into instances of $Y$ or vice versa. Moreover, since $w \in X$ if and only if $f(w) \in Y$, we know that the transformed instance will produce the same output as the original instance---since we're dealing with decision problems, this means that both $w$ and $f(w)$ will give us either a ``yes" answer or a ``no" answer. This observation reveals something very useful indeed: since we'll get the same answer for the instance of $Y$ as we would for the instance of $X$, we can use a reduction along with a decision algorithm for problem $Y$ to decide the original problem $X$.

\begin{figure}
\centering
\begin{tikzpicture}
\draw[draw=\fourthcolour, thick, fill=\fourthcolour, rounded corners] (0,1) rectangle (8,4);
\draw[draw=\fifthcolour, thick, fill=\fifthcolour] (0.75,1.5) rectangle (3.25,3);
\draw[draw=black, thick, fill=black] (4.75,1.5) rectangle (7.25,3);

\node[draw=none, color=\maincolour] at (4,3.5) {Decider for $X$};
\node[draw=none, color=\maincolour, text width=2cm, align=center] at (2,2.25) {Reduction from $X$ to $Y$};
\node[draw=none, color=white, text width=2cm, align=center] at (6,2.25) {Decider for $Y$};

\coordinate (A) at (0.75,2.25);
\coordinate (B) at (-1.5,2.25);
\draw[-latex, color=\maincolour, thick] (B) -- (A);
\node[draw=none, color=\maincolour, align=center, font=\footnotesize] at (-0.8,2.75) {Instance \\ $w$ of $X$};

\coordinate (C) at (4.75,2.25);
\coordinate (D) at (3.25,2.25);
\draw[-latex, color=\maincolour, thick] (D) -- (C);
\node[draw=none, color=\maincolour, font=\footnotesize] at (4,2.65) {$f(w)$};

\coordinate (E) at (9.7,2.25);
\coordinate (F) at (7.25,2.25);
\draw[-latex, color=\maincolour, thick] (F) -- (E);
\node[draw=none, color=\maincolour, align=center, font=\footnotesize] at (8.9,2.75) {$f(w) \in Y$ \\ iff $w \in X$};
\end{tikzpicture}
\caption{A Turing machine for $X$ using the reduction $X \mreducesto Y$}
\label{fig:manyonereduction}
\end{figure}

We denote a reduction from $X$ to $Y$ by the notation $X \mreducesto Y$. Figure~\ref{fig:manyonereduction} illustrates how such a reduction works, while Figure~\ref{fig:manyonereductionsets} shows how the computable function $f$ maps elements from one set to another.

\begin{dangerous}
Reductions are in general not reversible, so the direction of a reduction is important: if $X \mreducesto Y$, then we say that we reduce \emph{from} $X$ \emph{to} $Y$. Mixing up the direction of a reduction is a very common mistake made by the novice and the expert alike, so don't feel discouraged if it happens to you at least once.
\end{dangerous}

If we have a reduction from $X$ to $Y$, then we can make some claims about the relative difficulty of $X$ based on what we know about $Y$, or vice versa. The existence of a reduction from $X$ to $Y$ implies that finding an answer to $X$ is no more difficult than finding an answer to $Y$, or equivalently, finding an answer to $Y$ is at least as difficult as finding an answer to $X$. This is because, as Figure~\ref{fig:manyonereduction} illustrates, we must use the decider for $Y$ as an intermediate step in the overall decider for $X$. Thus, we have the following rules of thumb:
\begin{colouredbox}
\begin{itemize}
\item if $X$ reduces to $Y$ and $Y$ is ``easy", then we know that $X$ must similarly be ``easy"; and
\item if $X$ reduces to $Y$ and $X$ is ``hard", then we know that $Y$ must similarly be ``hard".
\end{itemize}
\end{colouredbox}
We can intuit about both of these rules of thumb as follows. If $Y$ is ``easy", then we already have a decider for $Y$, and we need only build the decider for $X$ around the decider for $Y$. On the other hand, if $X$ is ``hard", then everything inside a supposed decider for $X$---including the decider for $Y$---must be ``hard" for us to solve.

It's worth emphasizing once more that directionality is crucial with reductions. It wouldn't make sense for us to reduce from an ``easy" problem to a ``hard" problem, since our Turing machine could just decide the ``easy" problem directly while skipping over the supposed decider for the ``hard" problem: we don't need the reduction at all in this case.\par
\epigraph{If you are taller than someone who is tall, then you must be tall.\par
But if someone tall is taller than you, you might be\par
short or tall---we wouldn't yet know.}{Joel David Hamkins}{in a post on X (formerly Twitter) dated February 27, 2024}{}
\vspace{1em}

For now, we write ``easy" and ``hard" in quotation marks, since these notions are still informal. Soon, we will introduce complexity classes and define more precise notions of easiness and hardness for decision problems.

\begin{figure}
\centering
\begin{tikzpicture}
	\node (p1) at (0, 0) {};
	\node (p2) at (1.33, 1.5) {};
	\node (p3) at (0.75, 3.6) {};
	\node (p4) at (-1, 3.6) {};
	\node (p5) at (-1.5, 1) {};
	\draw[draw=\fourthcolour, fill=\fourthcolour] plot [smooth cycle,tension=1] coordinates {(p1) (p2) (p3) (p4) (p5)};
	\draw[draw=\thirdcolour, fill=\thirdcolour] (-0.1, 1.6) ellipse (0.7cm and 1cm);
	
	\node (q1) at (4, 0) {};
	\node (q2) at (5.5, 1) {};
	\node (q3) at (5.33, 3.3) {};
	\node (q4) at (3.5, 3.8) {};
	\node (q5) at (2.66, 1.33) {};
	\draw[draw=\fourthcolour, fill=\fourthcolour] plot [smooth cycle,tension=1] coordinates {(q1) (q2) (q3) (q4) (q5)};
	\draw[draw=\thirdcolour, fill=\thirdcolour] (4.2, 1.8) ellipse (0.7cm and 1cm);
	
	\node[color=white] at (-0.09, 2) {\large $X$};
	\node[color=\maincolour] at (-1.33, 4) {\large $\Sigma^{*}$};
	\node[color=white] at (4.25, 2.2) {\large $Y$};
	\node[color=\maincolour] at (5.5, 4) {\large $\Gamma^{*}$};
	
	\node[color=\maincolour] at (2, 0.25) {$f$};
	\node[color=\maincolour] at (2.1, 4.25) {$f$};

	\fill[color=\maincolour] (-0.4, 3) circle (2pt);
	\fill[color=\maincolour] (4.6, 3.2) circle (2pt);
	\fill[color=\maincolour] (-0.1, 1.2) circle (2pt);
	\fill[color=\maincolour] (4.2, 1.4) circle (2pt);

	\path[-Latex, thick, color=\maincolour] (-0.4, 3) edge[bend left] (4.6, 3.2);
	\path[-Latex, thick, color=\maincolour] (-0.1, 1.2) edge[bend right] (4.2, 1.4);
\end{tikzpicture}
\caption{How a reduction $X \mreducesto Y$ maps elements. All elements in $X$ are mapped by $f$ to some element in $Y$, while all other elements in $\Sigma^{*} \setminus X$ are mapped by $f$ to some other element in $\Gamma^{*} \setminus Y$}
\label{fig:manyonereductionsets}
\end{figure}

\subsection{Properties of Reductions}

Let's now establish some basic facts about the many-one reduction relation itself. We've already observed that reductions are in general not reversible, so we can say that the many-one reduction relation is not symmetric: if $X \mreducesto Y$, then it is not always the case that $Y \mreducesto X$ as well. However, we can prove two other nice properties.

\begin{lemma}\label{lem:manyonereductionreflexivetransitive}
The many-one reduction relation $\mreducesto$ is reflexive and transitive.

\begin{proof}
Let $X$, $Y$, and $Z$ be arbitrary decision problems.

To show that $\mreducesto$ is reflexive, take $f(x) = x$ as our computable function. Then $X \mreducesto X$ for all decision problems $X$.

To show that $\mreducesto$ is transitive, suppose that $X \mreducesto Y$ by way of some computable function $f$ and $Y \mreducesto Z$ by way of some other computable function $g$. Then $X \mreducesto Z$ by taking $h(x) = g(f(x))$ as our computable function.
\end{proof}
\end{lemma}

Another fact about reductions that will come in handy for us later is that the many-one reduction relation is closed under complement.

\begin{lemma}\label{lem:manyonereductioncomplement}
Given two decision problems $X$ and $Y$, $X \mreducesto Y$ if and only if $\overline{X} \mreducesto \overline{Y}$.

\begin{proof}
Since $X \mreducesto Y$, we know by Definition~\ref{def:manyonereduction} that there exists some computable function $f$ where, for all $w$, $w \in X$ if and only if $f(w) \in Y$.

If $w \in \overline{X}$, then $w \not\in X$, so $f(w) \not\in Y$ and thus $f(w) \in \overline{Y}$. Similarly, if $w \not\in \overline{X}$, then $w \in X$, so $f(w) \in Y$ and thus $f(w) \not\in \overline{Y}$. Therefore, $w \in \overline{X}$ if and only if $f(w) \in \overline{Y}$, and the same computable function $f$ gives us the reduction $\overline{X} \mreducesto \overline{Y}$.
\end{proof}
\end{lemma}

\subsection{Reductions, Decidability, and Semidecidability}

We can combine the notion of a decidable problem with that of a reduction to allow us to characterize an unknown problem in terms of another known problem.

\begin{theorem}\label{thm:YdecidableXdecidable}
If $Y$ is decidable and $X \mreducesto Y$, then $X$ is decidable.

\begin{proof}
Since $Y$ is decidable, there exists a Turing machine $\mathcal{M}_{Y}$ that decides instances of $Y$. Moreover, since $X \mreducesto Y$, there exists a computable function $f$ that reduces instances of $X$ to instances of $Y$.

We construct a Turing machine $\mathcal{M}_{X}$ that takes as input a word $w$ and performs the following steps:
\tmalgorithm{$\mathcal{M}_{X}$}{
\begin{enumerate}
\item Compute $f(w)$ using the reduction $X \mreducesto Y$.
\item Run $\mathcal{M}_{Y}$ on the result $f(w)$.
\item If $\mathcal{M}_{Y}$ accepts, then accept. If $\mathcal{M}_{Y}$ rejects, then reject.
\end{enumerate}
}
The computable function $f$---that is, our reduction from $X$ to $Y$---tells us that if $f(w) \in Y$, then it must be the case that $w \in X$. On the other hand, if $f(w) \not\in Y$, then $w \not\in X$. In either case, we can use the answer produced by the Turing machine $\mathcal{M}_{Y}$ to obtain an answer for our original decision problem $X$ and its original input $w$.
\end{proof}
\end{theorem}
\noindent
The main benefit of Theorem~\ref{thm:YdecidableXdecidable} is that, as long as we know two things---namely, that $Y$ is decidable and that there exists a reduction $X \mreducesto Y$---we don't need to construct an entirely new Turing machine just to decide $X$. We can just sit back and let the existing Turing machine $\mathcal{M}_{Y}$ do all the work on the reduced instance of $X$!

Of course, the main focus of this chapter is proving undecidability, and so it makes sense for us to connect reductions to this notion as well. By taking the contrapositive of Theorem~\ref{thm:YdecidableXdecidable}, we get the following important result that we will use frequently in future proofs.

\begin{corollary}\label{cor:XundecidableYundecidable}
If $X$ is undecidable and $X \mreducesto Y$, then $Y$ is undecidable.
\end{corollary}

Indeed, if we have an undecidable problem $X$ and a reduction $X \mreducesto Y$ to some other decision problem $Y$ we wish to prove undecidable, Corollary~\ref{cor:XundecidableYundecidable} gives us a general template for such a proof:
\begin{colouredbox}
\begin{enumerate}
\item Assume by way of contradiction that $Y$ is decidable by some Turing machine $\mathcal{M}_{Y}$.
\item Construct the following Turing machine $\mathcal{M}_{X}$ that takes as input a word $w$ and supposedly decides $X$ using the machine $\mathcal{M}_{Y}$:
\tmalgorithm[0.92]{$\mathcal{M}_{X}$}{
\begin{enumerate}
\item[\textbf{1.}] \textbf{Compute $\bm{f(w)}$ using the reduction $\bm{X \mreducesto Y}$.}
\item[2.] Run $\mathcal{M}_{Y}$ on the result $f(w)$.
\item[3.] If $\mathcal{M}_{Y}$ accepts, then accept. If $\mathcal{M}_{Y}$ rejects, then reject.
\end{enumerate}
}
\item Since $X$ is undecidable, conclude that such machines $\mathcal{M}_{X}$ and $\mathcal{M}_{Y}$ cannot exist, and so $Y$ cannot be decidable either.
\end{enumerate}
\end{colouredbox}
\noindent
The key part of any proof of undecidability that uses reductions comes in the first step of the description of $\mathcal{M}_{X}$, which we have emphasized in \textbf{boldface}. We must develop the reduction $X \mreducesto Y$ specifically for the decision problems $X$ and $Y$ under consideration in the proof, and so this step is the one that most often requires creative thinking and customization. Thankfully, the rest of the proof is mostly boilerplate and can be reused.

\begin{dangerous}
Another warning about the directionality of reductions: take careful note of the wordings of Theorem~\ref{thm:YdecidableXdecidable} and Corollary~\ref{cor:XundecidableYundecidable}. If $X$ is \emph{decidable} and $X \mreducesto Y$, then we can't make any conclusions about the decidability of $Y$. It's possible that $Y$ may be undecidable even if $X$ is decidable.
\end{dangerous}

We can make similar claims about semidecidability instead of decidability by using essentially the same proof as in Theorem~\ref{thm:YdecidableXdecidable}. The difference here, of course, is that we no longer have the guarantee that our Turing machine $\mathcal{M}_{Y}$ will always halt.

\begin{theorem}\label{thm:YsemidecidableXsemidecidable}
If $Y$ is semidecidable and $X \mreducesto Y$, then $X$ is semidecidable.
\end{theorem}

Again, taking the contrapositive gives us another important result that will come in handy later.

\begin{corollary}\label{cor:XnotsemidecidableYnotsemidecidable}
If $X$ is not semidecidable and $X \mreducesto Y$, then $Y$ is not semidecidable.
\end{corollary}
\section{The Halting Problem}\label{sec:haltingproblem}

\firstwords{Most, if not all,} of the code we write in our daily lives has the desirable behaviour that it eventually ends its computation and produces an output. For example, the following code can easily be seen to stop after some time:
\begin{colouredbox}
\begin{algorithmic}
\State $i \gets 1$
\For{$2 \leq j \leq 10$}
	\State $i \gets i \times j$
\EndFor
\State \Return $i$
\end{algorithmic}
\end{colouredbox}
\noindent
On the other hand, the following code will never stop, ceaselessly churning on and on until either the user intervenes or the computer shuts off:
\begin{colouredbox}
\begin{algorithmic}
\State $i \gets 0$
\While{true}
	\State $i \gets i + 1$
\EndWhile
\end{algorithmic}
\end{colouredbox}
\noindent
Does it happen to be the case that any useful code we write exhibits the behaviour of eventually stopping and producing something for us? Let's consider one more block of code, which has both a ``while true" infinite loop and a ``return" statement within that loop giving us a possible avenue to stop computing:
\begin{colouredbox}
\begin{algorithmic}
\State $i \gets 4$
\While{true}
	\If{$i$ is not the sum of two prime numbers}
		\State \Return false
	\Else
		\State \textbf{print} the two prime numbers that sum to $i$
		\State $i \gets i + 2$
	\EndIf
\EndWhile
\end{algorithmic}
\end{colouredbox}
\noindent
The utility of this code might be up for debate if you're not in any pressing need to know whether a number can be represented as a sum of two primes, but mathematicians would argue that this code is very useful: it will reach that internal ``return" statement if and only if the Goldbach conjecture is false! So, will this code ever stop and return false? The Goldbach conjecture has been verified to hold for all even numbers up to $4 \times 10^{18}$---an incredibly large value---but no general proof or disproof is yet known. Thus, if we could determine whether this code will ever come to a halt (and assuming it doesn't do so due to a computer bug), it would be very big news in the mathematical world.

The question of whether code halts has deep connections and implications across computer science, from the most abstract theory to the most applied software engineering. Some infinite-looping behaviour is desirable in code---say, in software that polls the status of a computer---but in other code, it is very important for us to know whether or not we'll eventually get the answer we desire. This has led to the \emph{halting problem} becoming one of the most famous problems in all of computer science.

Although we have framed our discussion so far in terms of code, the halting problem formally asks whether the computation of a Turing machine halts on some given input word. Of course, thanks to the Church--Turing thesis in Section~\ref{sec:churchturingthesis}, we know that code and Turing machines are ``the same" in the sense that code implements algorithms and algorithms can be computed on Turing machines. Thus, we can formulate the halting problem precisely as follows:
\begin{equation*}
\mathit{H}_{\TM} = \{\langle \mathcal{M}, w \rangle \mid \mathcal{M} \text{ is a Turing machine that halts on input } w\}.
\end{equation*}
Observe that the formulation of $\mathit{H}_{\TM}$ looks very similar to that of $A_{\TM}$. However, there exists a subtle difference between the two problems: $A_{\TM}$ asks not only whether a given Turing machine \emph{halts}, but also whether it \emph{accepts} a given input word. By contrast, $\mathit{H}_{\TM}$ only cares about whether the machine halts.

We can at least say that $\mathit{H}_{\TM}$ is semidecidable, since we can construct a Turing machine that takes as input a description of a Turing machine $\mathcal{M}$ along with an input word $w$ and accepts if and only if the simulated computation of $\mathcal{M}$ on $w$ halts.

\begin{theorem}\label{thm:HTMsemidecidable}
$\mathit{H}_{\TM}$ is semidecidable.

\begin{proof}
Construct a Turing machine $\mathcal{M}_{\mathrm{HTM}}$ that takes as input $\langle \mathcal{M}, w \rangle$, where $\mathcal{M}$ is a Turing machine and $w$ is a word, and performs the following steps:
\tmalgorithm{$\mathcal{M}_{\mathrm{HTM}}$}{
\begin{enumerate}
\item Run $\mathcal{M}$ on the input word $w$.
\item If $\mathcal{M}$ halts, then accept.
\end{enumerate}
}
This Turing machine accepts its input $\langle \mathcal{M}, w \rangle$ if and only if the Turing machine $\mathcal{M}$ halts when given its input $w$. If $\mathcal{M}$ enters an infinite loop on the input $w$, then $\mathcal{M}_{\mathrm{HTM}}$ will likewise loop forever. Thus, $\mathcal{M}_{\mathrm{HTM}}$ semidecides the halting problem.
\end{proof}
\end{theorem}

Now, we will prove the undecidability of $\mathit{H}_{\TM}$ by way of reduction from our known-undecidable problem $A_{\TM}$. Note that reducing \emph{from} $A_{\TM}$ \emph{to} $\mathit{H}_{\TM}$ means that we can turn instances of $A_{\TM}$ into instances of $\mathit{H}_{\TM}$, and since we know that $A_{\TM}$ is undecidable, this must mean that $\mathit{H}_{\TM}$ is similarly undecidable by Corollary~\ref{cor:XundecidableYundecidable}.

\begin{theorem}\label{thm:HTMundecidable}
$\mathit{H}_{\TM}$ is undecidable.

\begin{proof}
Assume by way of contradiction that $\mathit{H}_{\TM}$ is decidable, and suppose that $\mathcal{M}_{\mathrm{H}}$ is a Turing machine that decides $\mathit{H}_{\TM}$.

We construct a new Turing machine $\mathcal{M}_{\mathrm{A}}$ for the membership problem $\mathit{A}_{\TM}$ that uses the reduction $\mathit{A}_{\TM} \mreducesto \mathit{H}_{\TM}$. The machine $\mathcal{M}_{\mathrm{A}}$ takes as input $\langle \mathcal{M}, w \rangle$, where $\mathcal{M}$ is a Turing machine and $w$ is an input word, and performs the following steps:
\tmalgorithm{$\mathcal{M}_{\mathrm{A}}$}{
\begin{enumerate}
\item Define a computable function $f$ that takes as input $\langle \mathcal{M}, w \rangle$ and produces as output $\langle \mathcal{M}', w \rangle$, where $\mathcal{M}'$ behaves as follows:
	\tmalgorithm[0.925]{$\mathcal{M}'$}{
	\begin{enumerate}
	\item Run $\mathcal{M}$ on the input word given to $\mathcal{M}'$.
	\item If $\mathcal{M}$ accepts, then accept. If $\mathcal{M}$ rejects, then loop forever.
	\end{enumerate}
	}
\item Run $\mathcal{M}_{\mathrm{H}}$ on the result $\langle \mathcal{M}', w \rangle$ produced by $f$.
\item If $\mathcal{M}_{\mathrm{H}}$ accepts, then accept. If $\mathcal{M}_{\mathrm{H}}$ rejects, then reject.
\end{enumerate}
}
Observe that if $\langle \mathcal{M}, w \rangle \in \mathit{A}_{\TM}$, then $\mathcal{M}$ accepts $w$ and $\mathcal{M}'$ likewise accepts and therefore halts, meaning that $\langle \mathcal{M}', w \rangle \in \mathit{H}_{\TM}$. Otherwise, if $\langle \mathcal{M}, w \rangle \not\in \mathit{A}_{\TM}$, then $\mathcal{M}$ either rejects $w$ or loops forever, and in either case $\mathcal{M}'$ loops forever, meaning that $\langle \mathcal{M}', w \rangle \not\in \mathit{H}_{\TM}$.

If the machine $\mathcal{M}_{\mathrm{H}}$ existed to decide $\mathit{H}_{\TM}$, then we could decide $A_{\TM}$ using the machine $\mathcal{M}_{\mathrm{A}}$. However, we know that $A_{\TM}$ is undecidable. Thus, $\mathcal{M}_{\mathrm{H}}$ must not exist, and so $\mathit{H}_{\TM}$ must be undecidable.
\end{proof}
\end{theorem}

In our proof that $\mathit{H}_{\TM}$ is undecidable, we constructed a Turing machine $\mathcal{M}_{\mathrm{A}}$ that ostensibly decides $A_{\TM}$ and is composed of two parts: a reduction that computes a function $f$ turning instances of $A_{\TM}$ into instances of $\mathit{H}_{\TM}$, and a black-box Turing machine $\mathcal{M}_{\mathrm{H}}$ that decides $\mathit{H}_{\TM}$. As before, the Turing machine using this reduction from $\mathit{A}_{\TM}$ to $\mathit{H}_{\TM}$ is illustrated in Figure~\ref{fig:manyonereductionATMHTM}.

\begin{figure}
\centering
\begin{tikzpicture}
\draw[draw=\fourthcolour, thick, fill=\fourthcolour, rounded corners] (0,1) rectangle (7.1,4);
\draw[draw=\fifthcolour, thick, fill=\fifthcolour] (0.75,1.5) rectangle (3.25,3);
\draw[draw=black, thick, fill=black] (4.75,1.5) rectangle (6.35,3);

\node[draw=none, color=\maincolour, font=\large] at (3.55,3.5) {$\mathcal{M}_{\mathrm{A}}$};
\node[draw=none, color=\maincolour, text width=2cm, align=center] at (2,2.25) {Reduction from $\mathit{A}_{\TM}$ to $\mathit{H}_{\TM}$};
\node[draw=none, color=white, text width=2cm, align=center, font=\large] at (5.55,2.25) {$\mathcal{M}_{\mathrm{H}}$};

\coordinate (A) at (0.75,2.25);
\coordinate (B) at (-1.4,2.25);
\draw[-latex, color=\maincolour, thick] (B) -- (A);
\node[draw=none, color=\maincolour, align=center, font=\footnotesize] at (-0.7,2.65) {$\langle \mathcal{M}, w \rangle$};

\coordinate (C) at (4.75,2.25);
\coordinate (D) at (3.25,2.25);
\draw[-latex, color=\maincolour, thick] (D) -- (C);
\node[draw=none, color=\maincolour, font=\footnotesize] at (4,2.65) {$\langle \mathcal{M}', w \rangle$};

\coordinate (E) at (9.8,2.25);
\coordinate (F) at (6.35,2.25);
\draw[-latex, color=\maincolour, thick] (F) -- (E);
\node[draw=none, color=\maincolour, align=center, font=\footnotesize] at (8.5,2.75) {$\langle \mathcal{M}', w \rangle \in \mathit{H}_{\TM}$ \\ iff $\langle \mathcal{M}, w \rangle \in \mathit{A}_{\TM}$};
\end{tikzpicture}
\caption{A Turing machine for $\mathit{A}_{\TM}$ using the reduction $\mathit{A}_{\TM} \mreducesto \mathit{H}_{\TM}$}
\label{fig:manyonereductionATMHTM}
\end{figure}

We don't know how the black-box Turing machine $\mathcal{M}_{\mathrm{H}}$ decides $\mathit{H}_{\TM}$; we only assume that it exists. However, we \emph{do} know how the reduction works---this is step 1 in our procedure! The function $f$ takes the description of the input Turing machine $\mathcal{M}$ and converts it into a new Turing machine $\mathcal{M}'$ that halts on its input word if and only if $\mathcal{M}$ accepts the same input word.

At this point, you may ask yourself: doesn't this reduction implicitly decide $A_{\TM}$, since it has to figure out whether $\mathcal{M}$ accepts or rejects its input word? This is a reasonable question, since if the reduction did work in this way, we would find ourselves trapped in a snare of circular logic. Fortunately for us, we avoid such a trap, since the reduction does no deciding on its own: it only modifies the description of $\mathcal{M}$. Namely,
\begin{colouredbox}
\begin{itemize}
\item if $\langle \mathcal{M} \rangle$ has a transition leading to $q_{\text{accept}}$, then the reduction leaves this transition as is; and
\item if $\langle \mathcal{M} \rangle$ has a transition leading to $q_{\text{reject}}$, then the reduction modifies this transition to instead enter a new ``infinite loop" state and render $q_{\text{reject}}$ unreachable.
\end{itemize}
\end{colouredbox}
Thus, the reduction only changes how certain transitions of $\mathcal{M}$ behave, and it doesn't consider the output of $\mathcal{M}$ on any particular input word.

Ultimately, our construction establishes that $\mathcal{M}$ accepts $w$ if and only if $\mathcal{M}'$ halts on $w$ (i.e., $\langle \mathcal{M}, w \rangle \in A_{\TM}$ if and only if $\langle \mathcal{M}', w \rangle \in \mathit{H}_{\TM}$). At the same time, $\mathcal{M}$ does not accept $w$ or $\mathcal{M}$ loops forever on $w$ if and only if $\mathcal{M}'$ loops forever on $w$ (i.e., $\langle \mathcal{M}, w \rangle \not\in A_{\TM}$ if and only if $\langle \mathcal{M}', w \rangle \not\in \mathit{H}_{\TM}$).

You may be wondering whether we needed such a complicated proof to show that the halting problem is undecidable, and strictly speaking, we didn't. In fact, we can return to the code-oriented approach with which we started this section! \citet{Strachey1965AnImpossibleProgram} showed that there can exist no function that behaves like our Turing machine $\mathcal{M}_{H}$ using just a few lines of code. In his approach, Strachey takes $T[R]$ to be a Boolean function receiving an arbitrary program $R$ as input, where $R$ has no free variables as arguments. (In other words, such a program $R$ is ``fixed" in the sense that it will always exhibit the same behaviour when run.) The function $T$ behaves in the following way: for all programs $R$, $T[R]$ returns true if $R$ terminates when run, and $T[R]$ returns false otherwise. Then, Strachey writes the following program $P$:
\begin{colouredbox}
\begin{algorithmic}
\State \textbf{rec} \textbf{routine} $P$
\State \hspace{\algorithmicindent} \defaultS \ $L$ : \textbf{if} $T[P]$ \textbf{go to} $L$
\State \hspace{\algorithmicindent}\hspace{\algorithmicindent} \Return \vertchar{\defaultS}
\end{algorithmic}
\end{colouredbox}
\noindent
Unless you're familiar with the programming language CPL, this code may look a little obscure to you, but we can look beyond the syntax to tease out the general idea of what the code is doing. As we noted, $T[P]$ checks whether or not $P$ terminates when run, and it returns either true or false. If $T[P]$ returns true, then we will enter the body of the if statement and jump to the line labelled $L$, which puts us back at the beginning of the if statement; that is, $P$ loops forever. On the other hand, if $T[P]$ returns false, then we will skip over the if statement entirely and return an empty output; that is, $P$ halts. In either case, $P$ behaves in a manner opposite to what $T[P]$ indicated, and so the function $T$ cannot exist!

\epigraph{I have never actually seen a proof of this in print, and\par
though Alan Turing once gave me a verbal proof (in a railway\par
carriage on the way to a Conference at the NPL in 1953),\par
I unfortunately and promptly forgot the details.}{Christopher Strachey}{An impossible program}{}
\section{More Undecidable Problems for Turing Machines}\label{sec:moreundecidableTM}

\firstwords{At this point,} our collection of undecidable problems is slowly growing: we know now that each of $\mathit{A}_{\TM}$, $\overline{\mathit{A}_{\TM}}$, and $\mathit{H}_{\TM}$ is undecidable. Using these facts together with the power of reducibility, we can prove many other problems for Turing machines are undecidable.

\subsection*{Emptiness Problem}

Let's start by considering the familiar emptiness problem for Turing machines, $\mathit{E}_{\TM}$. In the previous section, we showed that $\mathit{H}_{\TM}$ is undecidable by reducing from $\mathit{A}_{\TM}$, since both of these decision problems rely in some way on halting: $\mathit{H}_{\TM}$ just focuses on a Turing machine halting on some input word, while for $\mathit{A}_{\TM}$, we know that a Turing machine must necessarily halt in the process of accepting some input word.

The decision problem $\mathit{E}_{\TM}$ is different, though: if a Turing machine's language is empty, then it must \emph{not} accept any input words we give to it. Therefore, it doesn't make much sense for us to involve $\mathit{A}_{\TM}$ in our proof of undecidability, since this decision problem relies on accepting! Instead, since we're focused on not accepting, let's go with the complementary decision problem $\overline{\mathit{A}_{\TM}}$.

Recall that $\mathit{E}_{\TM}$ expects to receive a description of a Turing machine as input, while $\overline{\mathit{A}_{\TM}}$ expects to receive as input both a description of a Turing machine and an input word to give to that machine. Our reduction therefore must have the property that, for any Turing machine $\mathcal{M}$ and input word $w$, $\langle \mathcal{M}, w \rangle \in \overline{\mathit{A}_{\TM}}$ if and only if $\langle \mathcal{M}' \rangle \in \mathit{E}_{\TM}$ for some Turing machine $\mathcal{M}'$.

The big question is: what is $\mathcal{M}'$, and how can we ensure $\mathcal{M}'$ doesn't accept any input words if and only if $\mathcal{M}$ doesn't accept $w$? Our approach for this proof will be to construct $\mathcal{M}'$ in such a way that, no matter what input word it receives, it completely ignores that word and only checks the behaviour of $\mathcal{M}$ on the specific input word $w$. Then, whatever answer $\mathcal{M}$ gives for $w$ will become the answer given by $\mathcal{M}'$. (See Figure~\ref{fig:ETMreductionmachine} for an illustration of what $\mathcal{M}'$ is doing.) In essence, $\mathcal{M}'$ generalizes the answer given by $\mathcal{M}$: if $\mathcal{M}$ does not accept $w$, then $\mathcal{M}'$ will accept \emph{nothing}, while if $\mathcal{M}$ accepts $w$, then $\mathcal{M}'$ will accept \emph{everything}.

\begin{figure}
\centering
\begin{tikzpicture}
\draw[draw=\fourthcolour, thick, fill=\fourthcolour, rounded corners] (1.05,1) rectangle (7.7,4);
\draw[draw=\fifthcolour, thick, fill=\fifthcolour] (3.95,1.5) rectangle (5.55,3);

\node[draw=none, color=\maincolour, font=\large] at (4.5,3.5) {$\mathcal{M}'$};
\node[draw=none, color=\maincolour, text width=2cm, align=center, font=\large] at (4.75,2.25) {$\mathcal{M}$};

\node[draw=none, fill=white, cloud, minimum size=1.25cm] at (2.1,2.35) {};
\node[draw=none, color=\maincolour, font=\footnotesize, rotate=-36] at (2.2, 2.5) {\textit{poof}};

\coordinate (A) at (2,2.25);
\coordinate (B) at (-1,2.25);
\draw[-latex, color=\maincolour, thick] (B) -- (A);
\node[draw=none, color=\maincolour, align=center, font=\footnotesize] at (0,2.65) {Arbitrary \\ input word $x$};

\coordinate (C) at (3.95,2.25);
\coordinate (D) at (3.25,2.25);
\draw[-latex, color=\maincolour, thick] (D) -- (C);
\node[draw=none, color=\maincolour, font=\footnotesize] at (3.55,2.5) {$w$};

\coordinate (E) at (9.9,1.75);
\coordinate (F) at (5.55,1.75);
\coordinate (G) at (9.9,2.45);
\coordinate (H) at (5.55,2.45);
\draw[-latex, color=\maincolour, thick] (F) -- (E);
\draw[-latex, color=\maincolour, thick] (H) -- (G);
\node[draw=none, color=\maincolour, align=center, font=\footnotesize] at (6.6,2.7) {$\mathcal{M}$ accepts $w$};
\node[draw=none, color=\maincolour, align=center, font=\footnotesize] at (6.6,2) {$\mathcal{M}$ rejects $w$};
\node[draw=none, color=\maincolour, align=center, font=\footnotesize] at (8.75,2.7) {$\mathcal{M}'$ accepts $x$};
\node[draw=none, color=\maincolour, align=center, font=\footnotesize] at (8.75,2) {$\mathcal{M}'$ rejects $x$};
\end{tikzpicture}
\caption{The Turing machine $\mathcal{M}'$ from the proof of Theorem~\ref{thm:ETMundecidable}}
\label{fig:ETMreductionmachine}
\end{figure}

At this point, we can start to see how the proof will come together. Our reduction will turn the encoding $\langle \mathcal{M}, w \rangle$ into an encoding $\langle \mathcal{M}' \rangle$, where the behaviour of $\mathcal{M}'$ depends solely on the behaviour of $\mathcal{M}$ given $w$. Then, supposing we can test the emptiness of the language $L(\mathcal{M}')$, we can also determine whether $w \not\in L(\mathcal{M})$.

\begin{theorem}\label{thm:ETMundecidable}
$\mathit{E}_{\TM}$ is undecidable.

\begin{proof}
Assume by way of contradiction that $\mathit{E}_{\TM}$ is decidable, and suppose that $\mathcal{M}_{\mathrm{E}}$ is a Turing machine that decides $\mathit{E}_{\TM}$.

We construct a new Turing machine $\mathcal{M}_{\overline{\mathrm{A}}}$ for the non-membership problem $\overline{\mathit{A}_{\TM}}$ that uses the reduction $\overline{\mathit{A}_{\TM}} \mreducesto \mathit{E}_{\TM}$. The machine $\mathcal{M}_{\overline{\mathrm{A}}}$ takes as input $\langle \mathcal{M}, w \rangle$, where $\mathcal{M}$ is a Turing machine and $w$ is an input word, and performs the following steps:
\tmalgorithm{$\mathcal{M}_{\overline{\mathrm{A}}}$}{
\begin{enumerate}
\item Define a computable function $f$ that takes as input $\langle \mathcal{M}, w \rangle$ and produces as output $\langle \mathcal{M}' \rangle$, where $\mathcal{M}'$ behaves as follows:
	\tmalgorithm[0.925]{$\mathcal{M}'$}{
	\begin{enumerate}
	\item Run $\mathcal{M}$ on the input word $w$.
	\item If $\mathcal{M}$ accepts $w$, then accept. If $\mathcal{M}$ rejects $w$, then reject.
	\end{enumerate}
	}
\item Run $\mathcal{M}_{\mathrm{E}}$ on the result $\langle \mathcal{M}' \rangle$ produced by $f$.
\item If $\mathcal{M}_{\mathrm{E}}$ accepts, then accept. If $\mathcal{M}_{\mathrm{E}}$ rejects, then reject.
\end{enumerate}
}
Observe that if $\langle \mathcal{M}, w \rangle \in \overline{\mathit{A}_{\TM}}$, then $\mathcal{M}$ does not accept $w$ and $\mathcal{M}'$ will not accept any input word given to it, meaning that $\langle \mathcal{M}' \rangle \in \mathit{E}_{\TM}$. Otherwise, if $\langle \mathcal{M}, w \rangle \not\in \overline{\mathit{A}_{\TM}}$, then $\mathcal{M}$ accepts $w$ and $\mathcal{M}'$ will accept every input word given to it, meaning that $\langle \mathcal{M}' \rangle \not\in \mathit{E}_{\TM}$.

If the machine $\mathcal{M}_{\mathrm{E}}$ existed to decide $\mathit{E}_{\TM}$, then we could decide $\overline{\mathit{A}_{\TM}}$ using the machine $\mathcal{M}_{\overline{\mathrm{A}}}$. However, we know that $\overline{\mathit{A}_{\TM}}$ is undecidable. Thus, $\mathcal{M}_{\mathrm{E}}$ must not exist, and so $\mathit{E}_{\TM}$ must be undecidable.
\end{proof}
\end{theorem}

The careful reader might have noticed that the construction we used to prove Theorem~\ref{thm:ETMundecidable} in fact goes one step further in telling us about the properties of $\mathit{E}_{\TM}$. Since we know that $\overline{\mathit{A}_{\TM}}$ is non-semidecidable by Theorem~\ref{thm:coATMnotsemidecidable}, the reduction $\overline{\mathit{A}_{\TM}} \mreducesto \mathit{E}_{\TM}$ combined with Corollary~\ref{cor:XnotsemidecidableYnotsemidecidable} allows us to establish the following even stronger fact about $\mathit{E}_{\TM}$.

\begin{corollary}\label{cor:ETMnotsemidecidable}
$\mathit{E}_{\TM}$ is not semidecidable.
\end{corollary}

On the other hand, we at least have a glimmer of hope when it comes to handling instances of the Turing machine ``non-emptiness" problem $\overline{\mathit{E}_{\TM}}$.

\begin{theorem}\label{thm:coETMsemidecidable}
$\overline{\mathit{E}_{\TM}}$ is semidecidable.

\begin{proof}
Construct a Turing machine $\mathcal{M}_{\overline{\mathrm{ETM}}}$ that takes as input $\langle \mathcal{M} \rangle$, where $\mathcal{M}$ is a Turing machine, and performs the following steps:
\tmalgorithm{$\mathcal{M}_{\overline{\mathrm{ETM}}}$}{
\begin{enumerate}
\item Enumerate all words over $\Sigma^{*}$, producing a list $\{w_{1}, w_{2}, w_{3}, \dots\}$.
\item For all $i \in \{1, 2, 3, \dots\}$, simulate $i$ steps of the computation of $\mathcal{M}$ on the first $i$ words $w_{1}$ to $w_{i}$.
\item If $\mathcal{M}$ ever enters its accepting state $q_{\text{accept}}$ on any of the words, then accept.
\end{enumerate}
}
Observe first that since $\Sigma^{*}$ is a countably infinite union of finite sets, it is itself countably infinite, and so we can perform the enumeration specified in Step 1 of $\mathcal{M}_{\overline{\mathrm{ETM}}}$.

This Turing machine accepts its input $\langle \mathcal{M} \rangle$ if and only if the Turing machine $\mathcal{M}$ accepts at least one input word $w_{j}$ where $j \in \{1, 2, 3, \dots\}$, and if no word is accepted by $\mathcal{M}$, then the Turing machine loops forever. Thus, $\mathcal{M}_{\overline{\mathrm{ETM}}}$ semidecides the non-emptiness problem for Turing machines.
\end{proof}
\end{theorem}

As a consequence of Theorem~\ref{thm:coETMsemidecidable}, we can also say that $\mathit{E}_{\TM}$ is co-semidecidable.

Unfortunately, semidecidability is all that we can hope for with $\overline{\mathit{E}_{\TM}}$: since reductions are closed under complement by Lemma~\ref{lem:manyonereductioncomplement}, knowing that $\overline{\mathit{A}_{\TM}} \mreducesto \mathit{E}_{\TM}$ also tells us that $\mathit{A}_{\TM} \mreducesto \overline{\mathit{E}_{\TM}}$, and since $\mathit{A}_{\TM}$ is undecidable, we conclude that $\overline{\mathit{E}_{\TM}}$ must also be undecidable.

Before we continue, let's revisit our choice to use the decision problem $\overline{\mathit{A}_{\TM}}$ in our reduction showing that $\mathit{E}_{\TM}$ is undecidable. Even though we saw an earlier justification for why it didn't make much sense for us to use $\mathit{A}_{\TM}$ in this situation, would it nevertheless have been possible for us to reduce $\mathit{A}_{\TM}$ to $\mathit{E}_{\TM}$ as an alternative proof? Interestingly, no! Using many-one reductions, there is in fact no way for us to reduce $\mathit{A}_{\TM}$ to $\mathit{E}_{\TM}$, and the proof follows directly from what we already know about both problems.

\begin{theorem}\label{thm:ATMnotmanyonereducibleETM}
$\mathit{A}_{\TM} \not\mreducesto \mathit{E}_{\TM}$.

\begin{proof}
Assume by way of contradiction that $\mathit{A}_{\TM} \mreducesto \mathit{E}_{\TM}$, and let $f$ be the computable function performing such a reduction. Since reductions are closed under complement by Lemma~\ref{lem:manyonereductioncomplement}, we also have that $\overline{\mathit{A}_{\TM}} \mreducesto \overline{\mathit{E}_{\TM}}$ by the same computable function $f$.

We know by Theorem~\ref{thm:coETMsemidecidable} that $\overline{\mathit{E}_{\TM}}$ is semidecidable, so by Theorem~\ref{thm:YsemidecidableXsemidecidable}, this implies that $\overline{\mathit{A}_{\TM}}$ is also semidecidable. However, this contradicts Theorem~\ref{thm:coATMnotsemidecidable}, so no such reduction can exist.
\end{proof}
\end{theorem}

Theorem~\ref{thm:ATMnotmanyonereducibleETM} raises a fascinating point about many-one reductions: sometimes, a many-one reduction between two decision problems simply cannot exist! This, in turn, highlights the importance of creatively constructing our reductions: if we wanted to prove directly that $\mathit{E}_{\TM}$ was undecidable, and we focused on using the classic undecidable problem $\mathit{A}_{\TM}$ in our proof, then we would've quickly hit a brick wall. Introducing the complementary problems $\overline{\mathit{E}_{\TM}}$ and $\overline{\mathit{A}_{\TM}}$ to our suite of undecidable problems gives us more flexibility.

\begin{remark}
Confusingly, some textbooks and learning materials purport to give a ``reduction" from $\mathit{A}_{\TM}$ to $\mathit{E}_{\TM}$ in their proof of the undecidability of $\mathit{E}_{\TM}$. These proofs typically use an informal notion of reducibility that disguises the fact that the reduction is actually from $\mathit{A}_{\TM}$ to $\overline{\mathit{E}_{\TM}}$.
\end{remark}

\subsection*{Universality Problem}

Let's now move on to the universality problem for Turing machines, $\mathit{U}_{\TM}$. Since this decision problem asks whether the language of a Turing machine contains all words, one might think (as we did with finite automata) that $\mathit{U}_{\TM}$ is just the opposite of $\mathit{E}_{\TM}$, and so we can immediately draw a connection to our complementary decision problem $\overline{\mathit{E}_{\TM}}$. Unfortunately, things aren't so easy when it comes to Turing machines, as the class of semidecidable languages isn't closed under complement! A positive answer to an instance of $\overline{\mathit{E}_{\TM}}$ only tells us that the Turing machine in question accepts \emph{at least one} word, not that it accepts \emph{all} words.

This doesn't mean we need to start from scratch, though: we can in fact reuse some of the ideas we had in showing that $\mathit{E}_{\TM}$ is undecidable to prove that $\mathit{U}_{\TM}$ is undecidable. If a Turing machine's language is universal, then it must accept \emph{every} word we give to it. This statement sounds very similar to $\mathit{A}_{\TM}$, which asks whether a Turing machine accepts whatever specific word we gave to it. Thus, just like we reduced $\overline{\mathit{A}_{\TM}}$ to $\mathit{E}_{\TM}$ in the previous section, it seems promising here for us to reduce $\mathit{A}_{\TM}$ to $\mathit{U}_{\TM}$.

Recall that $\mathit{U}_{\TM}$ expects to receive a description of a Turing machine as input, while $\mathit{A}_{\TM}$ expects to receive as input both a description of a Turing machine and an input word to give to that machine. Thus, our reduction must have the property that, for any Turing machine $\mathcal{M}$ and input word $w$, $\langle \mathcal{M}, w \rangle \in \mathit{A}_{\TM}$ if and only if $\langle \mathcal{M}' \rangle \in \mathit{U}_{\TM}$ for some Turing machine $\mathcal{M}'$. As before, we must construct a Turing machine $\mathcal{M}'$ that accepts all input words if and only if $\mathcal{M}$ accepts $w$.

It's no coincidence that we denoted this Turing machine by $\mathcal{M}'$ again: it turns out that we can use the exact same approach that we used with the Turing machine emptiness problem! Again, we will ensure that no matter what input word $\mathcal{M}'$ receives, it will ignore that word and instead simulate the computation of $\mathcal{M}$ on its input word $w$. Then, whatever answer $\mathcal{M}$ gives for $w$ will become the answer given by $\mathcal{M}'$. We're following the same idea by having $\mathcal{M}'$ generalize the answer given by $\mathcal{M}$: if $\mathcal{M}$ accepts $w$, then $\mathcal{M}'$ will accept \emph{everything}, while if $\mathcal{M}$ rejects $w$, then $\mathcal{M}'$ will accept \emph{nothing}.

Our reduction will therefore behave identically to our previous reduction for $\mathit{E}_{\TM}$: we will turn the encoding $\langle \mathcal{M}, w \rangle$ into an encoding $\langle \mathcal{M}' \rangle$, where the behaviour of $\mathcal{M}'$ depends solely on the behaviour of $\mathcal{M}$ given $w$. Then, supposing we can test the universality of the language $L(\mathcal{M}')$, we can also determine whether $w \in L(\mathcal{M})$.

\begin{theorem}\label{thm:UTMundecidable}
$\mathit{U}_{\TM}$ is undecidable.

\begin{proof}
Assume by way of contradiction that $\mathit{U}_{\TM}$ is decidable, and suppose that $\mathcal{M}_{\mathrm{U}}$ is a Turing machine that decides $\mathit{U}_{\TM}$.

We construct a new Turing machine $\mathcal{M}_{\mathrm{A}}$ for the membership problem $\mathit{A}_{\TM}$ that uses the reduction $\mathit{A}_{\TM} \mreducesto \mathit{U}_{\TM}$. The machine $\mathcal{M}_{\mathrm{A}}$ takes as input $\langle \mathcal{M}, w \rangle$, where $\mathcal{M}$ is a Turing machine and $w$ is an input word, and performs the following steps:
\tmalgorithm{$\mathcal{M}_{\mathrm{A}}$}{
\begin{enumerate}
\item Define a computable function $f$ that takes as input $\langle \mathcal{M}, w \rangle$ and produces as output $\langle \mathcal{M}' \rangle$, where $\mathcal{M}'$ behaves as follows:
	\tmalgorithm[0.925]{$\mathcal{M}'$}{
	\begin{enumerate}
	\item Run $\mathcal{M}$ on the input word $w$.
	\item If $\mathcal{M}$ accepts $w$, then accept. If $\mathcal{M}$ rejects $w$, then reject.
	\end{enumerate}
	}
\item Run $\mathcal{M}_{\mathrm{U}}$ on the result $\langle \mathcal{M}' \rangle$ produced by $f$.
\item If $\mathcal{M}_{\mathrm{U}}$ accepts, then accept. If $\mathcal{M}_{\mathrm{U}}$ rejects, then reject.
\end{enumerate}
}
Observe that if $\langle \mathcal{M}, w \rangle \in \mathit{A}_{\TM}$, then $\mathcal{M}$ accepts $w$ and $\mathcal{M}'$ will accept every input word given to it, meaning that $\langle \mathcal{M}' \rangle \in \mathit{U}_{\TM}$. Otherwise, if $\langle \mathcal{M}, w \rangle \not\in \mathit{A}_{\TM}$, then $\mathcal{M}$ does not accept $w$ and $\mathcal{M}'$ will not accept any input word given to it, meaning that $\langle \mathcal{M}' \rangle \not\in \mathit{U}_{\TM}$.

If the machine $\mathcal{M}_{\mathrm{U}}$ existed to decide $\mathit{U}_{\TM}$, then we could decide $\mathit{A}_{\TM}$ using the machine $\mathcal{M}_{\mathrm{A}}$. However, we know that $\mathit{A}_{\TM}$ is undecidable. Thus, $\mathcal{M}_{\mathrm{U}}$ must not exist, and so $\mathit{U}_{\TM}$ must be undecidable.
\end{proof}
\end{theorem}

Now, even if we have a negative decidability result, we know that $\mathit{A}_{\TM}$ is semidecidable, so does the reduction $\mathit{A}_{\TM} \mreducesto \mathit{U}_{\TM}$ give us any clue as to the status of semidecidability for $\mathit{U}_{\TM}$? No---and be careful to remember the wording of Theorem~\ref{thm:YsemidecidableXsemidecidable}! If we have a reduction $X \mreducesto Y$ and we know that $Y$ is semidecidable, then we can conclude that $X$ is semidecidable, but we can't conclude anything about $Y$ in the case where $X$ is semidecidable.

To tackle the question of semidecidability for $\mathit{U}_{\TM}$, let us introduce another decision problem known as the \emph{totality problem} for Turing machines:
\begin{equation*}
\mathit{T}_{\TM} = \{\langle \mathcal{M} \rangle \mid \mathcal{M} \text{ is a Turing machine that halts on all input words}\}.
\end{equation*}
By analogy, if $\mathit{H}_{\TM}$ is the ``halting only" version of $\mathit{A}_{\TM}$, then $\mathit{T}_{\TM}$ is the ``halting only" version of $\mathit{U}_{\TM}$. Naturally, since both $\mathit{H}_{\TM}$ and $\mathit{T}_{\TM}$ talk about halting, we can reduce one to the other in a rather straightforward way, and our reduction again uses our tried-and-true Turing machine $\mathcal{M}'$.

\begin{lemma}\label{lem:HTMmanyonereducibleTTM}
$\mathit{H}_{\TM} \mreducesto \mathit{T}_{\TM}$.

\begin{proof}
Define a computable function $f$ that takes as input $\langle \mathcal{M}, w \rangle$ and produces as output $\langle \mathcal{M}' \rangle$, where $\mathcal{M}'$ behaves as follows:
\tmalgorithm{$\mathcal{M}'$}{
\begin{enumerate}
\item Run $\mathcal{M}$ on the input word $w$.
\item If $\mathcal{M}$ accepts $w$, then accept. If $\mathcal{M}$ rejects $w$, then reject.
\end{enumerate}
}
Observe that if $\langle \mathcal{M}, w \rangle \in \mathit{H}_{\TM}$, then $\mathcal{M}$ halts on $w$ and $\mathcal{M}'$ will halt on any input word given to it, meaning that $\langle \mathcal{M}' \rangle \in \mathit{T}_{\TM}$. Otherwise, if $\langle \mathcal{M}, w \rangle \not\in \mathit{H}_{\TM}$, then $\mathcal{M}$ does not halt on $w$ and $\mathcal{M}'$ will likewise loop forever, meaning that $\langle \mathcal{M}' \rangle \not\in \mathit{T}_{\TM}$.
\end{proof}
\end{lemma}

Since we know that reductions are closed under complement, Lemma~\ref{lem:HTMmanyonereducibleTTM} tells us that $\overline{\mathit{H}_{\TM}} \mreducesto \overline{\mathit{T}_{\TM}}$ as well. At the same time, we can also reduce the complement of the halting problem, $\overline{\mathit{H}_{\TM}}$, directly to $\mathit{T}_{\TM}$. Of course, since we're attempting to test \emph{non-}halting with this reduction, we must be careful not to fall into the trap of looping forever by bounding the length of our Turing machine's computation.

\begin{lemma}\label{lem:HTMmanyonereduciblecoTTM}
$\overline{\mathit{H}_{\TM}} \mreducesto \mathit{T}_{\TM}$.

\begin{proof}
Define a computable function $f$ that takes as input $\langle \mathcal{M}, w \rangle$ and produces as output $\langle \mathcal{M}'' \rangle$, where $\mathcal{M}''$ behaves as follows:
\tmalgorithm{$\mathcal{M}''$}{
\begin{enumerate}
\item Run $\mathcal{M}$ on the input word $w$ for $|x|$ computation steps, where $|x|$ is the length of the input word $x$ given to $\mathcal{M}''$.
\item If $\mathcal{M}$ halts on $w$ within those $|x|$ computation steps, then loop forever. Otherwise, accept.
\end{enumerate}
}
Observe that if $\langle \mathcal{M}, w \rangle \in \overline{\mathit{H}_{\TM}}$, then $\mathcal{M}$ will not halt on $w$ and $\mathcal{M}''$ will halt on any input word given to it, meaning that $\langle \mathcal{M}'' \rangle \in \mathit{T}_{\TM}$. Otherwise, if $\langle \mathcal{M}, w \rangle \not\in \overline{\mathit{H}_{\TM}}$, then $\mathcal{M}$ will halt on $w$ in some number of computation steps $s$ and $\mathcal{M}''$ will not halt on any input word of length at most $s$, meaning that $\langle \mathcal{M}'' \rangle \not\in \mathit{T}_{\TM}$.
\end{proof}
\end{lemma}

You might be wondering at this point why we've introduced the complement of the halting problem, $\overline{\mathit{H}_{\TM}}$, and what this has to do with the totality problem for Turing machines. We haven't forgotten about our question of semidecidability for $\mathit{U}_{\TM}$; all of this is just building up to the answer.

Recall from Theorem~\ref{thm:HTMundecidable} that there exists a reduction $\mathit{A}_{\TM} \mreducesto \mathit{H}_{\TM}$. Since reductions are (as we well know by now) closed under complement, this means that there also exists a reduction $\overline{\mathit{A}_{\TM}} \mreducesto \overline{\mathit{H}_{\TM}}$. But Theorem~\ref{thm:coATMnotsemidecidable} tells us that $\overline{\mathit{A}_{\TM}}$ is not semidecidable, and therefore $\overline{\mathit{H}_{\TM}}$ must not be semidecidable either. Now, everything begins to fall into place: since $\overline{\mathit{H}_{\TM}} \mreducesto \overline{\mathit{T}_{\TM}}$ and $\overline{\mathit{H}_{\TM}} \mreducesto \mathit{T}_{\TM}$, it must be the case that neither $\mathit{T}_{\TM}$ nor $\overline{\mathit{T}_{\TM}}$ is semidecidable.

Completing this line of reasoning by connecting $\mathit{T}_{\TM}$ to $\mathit{U}_{\TM}$ gives us the final remarkable result: $\mathit{U}_{\TM}$ is not just undecidable, but it is also neither semidecidable nor co-semidecidable, and so it falls completely outside of our language hierarchy!

\begin{theorem}\label{thm:UTMnotsemidecidable}
$\mathit{U}_{\TM}$ is neither semidecidable nor co-semidecidable.

\begin{proof}
We begin by demonstrating the existence of a reduction $\mathit{T}_{\TM} \mreducesto \mathit{U}_{\TM}$. Define a computable function $f$ that takes as input $\langle \mathcal{M} \rangle$ and produces as output $\langle \mathcal{N} \rangle$, where $\mathcal{N}$ behaves as follows:
\tmalgorithm{$\mathcal{N}$}{
\begin{enumerate}
\item Behave exactly as $\mathcal{M}$ behaves, except redirect all transitions to $q_{\text{reject}}$ to instead go to $q_{\text{accept}}$.
\end{enumerate}
}
Observe that if $\langle \mathcal{M} \rangle \in \mathit{T}_{\TM}$, then $\mathcal{M}$ halts on every input word and either accepts or rejects. Thus, $\mathcal{N}$ will also halt on every input word, but it will always accept, meaning that $\langle \mathcal{N} \rangle \in \mathit{U}_{\TM}$. Otherwise, if $\langle \mathcal{M} \rangle \not\in \mathit{T}_{\TM}$, then $\mathcal{M}$ must not halt on at least one input word, and so $\mathcal{N}$ will likewise never accept that input word, meaning that $\langle \mathcal{N} \rangle \not\in \mathit{U}_{\TM}$.

From the reduction $\mathit{T}_{\TM} \mreducesto \mathit{U}_{\TM}$, we know also that $\overline{\mathit{T}_{\TM}} \mreducesto \overline{\mathit{U}_{\TM}}$. However, since neither $\mathit{T}_{\TM}$ nor $\overline{\mathit{T}_{\TM}}$ is semidecidable, it must be the case that neither $\mathit{U}_{\TM}$ nor $\overline{\mathit{U}_{\TM}}$ is semidecidable. Saying that $\overline{\mathit{U}_{\TM}}$ is not semidecidable is equivalent to saying that $\mathit{U}_{\TM}$ is not co-semidecidable.
\end{proof}
\end{theorem}

\subsection*{Equivalence Problem}

Recall that the equivalence problem for Turing machines, $\mathit{EQ}_{\TM}$, asks whether the languages of two Turing machines are equivalent; that is, whether no word belongs to one language but not the other.

As we might reasonably expect by this point, $\mathit{EQ}_{\TM}$ is an undecidable problem, just like all of the other problems we've studied thus far for Turing machines. But in each of our previous undecidability proofs, the underlying decision problems focused only on a single Turing machine: $\mathit{A}_{\TM}$ and $\mathit{H}_{\TM}$ both took as input an encoding of the form $\langle \mathcal{M}, w \rangle$, while each of $\mathit{E}_{\TM}$, $\mathit{U}_{\TM}$, and $\mathit{T}_{\TM}$ took as input an encoding of the form $\langle \mathcal{M} \rangle$. The problem $\mathit{EQ}_{\TM}$, by contrast, takes as input $\langle \mathcal{M}, \mathcal{N} \rangle$, where both $\mathcal{M}$ and $\mathcal{N}$ are Turing machines.

How, then, can we prove that $\mathit{EQ}_{\TM}$ is undecidable via some many-one reduction? We just need to use a little cleverness in our construction. Take, for example, the problem $\mathit{U}_{\TM}$. Going back to the definition of this problem, we have that $\langle \mathcal{M} \rangle \in \mathit{U}_{\TM}$ if and only if $L(\mathcal{M}) = \Sigma^{*}$. Look at what we just used in that definition: an equals sign! If we construct a new Turing machine specially designed to accept all input words, then we can compare the language of our original Turing machine $\mathcal{M}$ to the language of this new Turing machine, and therein lies the foundation for our reduction.

\begin{theorem}\label{thm:EQTMundecidable}
$\mathit{EQ}_{\TM}$ is undecidable.

\begin{proof}
Assume by way of contradiction that $\mathit{EQ}_{\TM}$ is decidable, and suppose that $\mathcal{M}_{\mathrm{EQ}}$ is a Turing machine that decides $\mathit{EQ}_{\TM}$.

We construct a new Turing machine $\mathcal{M}_{\mathrm{U}}$ for the universality problem $\mathit{U}_{\TM}$ that uses the reduction $\mathit{U}_{\TM} \mreducesto \mathit{EQ}_{\TM}$. The machine $\mathcal{M}_{\mathrm{U}}$ takes as input $\langle \mathcal{M} \rangle$, where $\mathcal{M}$ is a Turing machine, and performs the following steps:
\tmalgorithm{$\mathcal{M}_{\mathrm{U}}$}{
\begin{enumerate}
\item Define a computable function $f$ that takes as input $\langle \mathcal{M} \rangle$ and produces as output $\langle \mathcal{M}, \mathcal{M}_{\Sigma^{*}} \rangle$, where $\mathcal{M}_{\Sigma^{*}}$ behaves as follows:
	\tmalgorithm[0.925]{$\mathcal{M}_{\Sigma^{*}}$}{
	\begin{enumerate}
	\item Accept.
	\end{enumerate}
	}
\item Run $\mathcal{M}_{\mathrm{EQ}}$ on the result $\langle \mathcal{M}, \mathcal{M}_{\Sigma^{*}} \rangle$ produced by $f$.
\item If $\mathcal{M}_{\mathrm{EQ}}$ accepts, then accept. If $\mathcal{M}_{\mathrm{EQ}}$ rejects, then reject.
\end{enumerate}
}
Observe that if $\langle \mathcal{M} \rangle \in \mathit{U}_{\TM}$, then $\mathcal{M}$ accepts every input word given to it. Therefore, $L(\mathcal{M})$ is equal to $L(\mathcal{M}_{\Sigma^{*}})$, meaning that $\langle \mathcal{M}, \mathcal{M}_{\Sigma^{*}} \rangle \in \mathit{EQ}_{\TM}$. Otherwise, if $\langle \mathcal{M} \rangle \not\in \mathit{U}_{\TM}$, then $\mathcal{M}$ does not accept at least one input word, and therefore its language cannot be equal to $L(\mathcal{M}_{\Sigma^{*}})$, meaning that $\langle \mathcal{M}, \mathcal{M}_{\Sigma^{*}} \rangle \not\in \mathit{EQ}_{\TM}$.

If the machine $\mathcal{M}_{\mathrm{EQ}}$ existed to decide $\mathit{EQ}_{\TM}$, then we could decide $\mathit{U}_{\TM}$ using the machine $\mathcal{M}_{\mathrm{U}}$. However, we know that $\mathit{U}_{\TM}$ is undecidable. Thus, $\mathcal{M}_{\mathrm{EQ}}$ must not exist, and so $\mathit{EQ}_{\TM}$ must be undecidable.
\end{proof}
\end{theorem}

Keen readers might have noticed that we could alternatively prove Theorem~\ref{thm:EQTMundecidable} via the reduction $\mathit{E}_{\TM} \mreducesto \mathit{EQ}_{\TM}$. In this case, the computable function $f$ would construct a Turing machine $\mathcal{M}_{\emptyset}$ that rejects all input words, but the rest of the proof would be otherwise identical.

There's a good reason why we chose to reduce from $\mathit{U}_{\TM}$, though: this reduction provides an immediate proof that the equivalence problem for Turing machines is neither semidecidable nor co-semidecidable, giving us a second decision problem that falls completely outside of our language hierarchy.

\begin{theorem}\label{thm:EQTMnotsemidecidable}
$\mathit{EQ}_{\TM}$ is neither semidecidable nor co-semidecidable.

\begin{proof}
We demonstrated the existence of a reduction $\mathit{U}_{\TM} \mreducesto \mathit{EQ}_{\TM}$ in the proof of Theorem~\ref{thm:EQTMundecidable}, and since $\mathit{U}_{\TM}$ is not semidecidable by Theorem~\ref{thm:UTMnotsemidecidable}, $\mathit{EQ}_{\TM}$ is likewise not semidecidable.

From the reduction $\mathit{U}_{\TM} \mreducesto \mathit{EQ}_{\TM}$, we know also that $\overline{\mathit{U}_{\TM}} \mreducesto \overline{\mathit{EQ}_{\TM}}$, and since $\overline{\mathit{U}_{\TM}}$ is not semidecidable by Theorem~\ref{thm:UTMnotsemidecidable}, $\overline{\mathit{EQ}_{\TM}}$ is likewise not semidecidable. Saying that $\overline{\mathit{EQ}_{\TM}}$ is not semidecidable is equivalent to saying that $\mathit{EQ}_{\TM}$ is not co-semidecidable.
\end{proof}
\end{theorem}

\subsection*{Inclusion Problem}

\begin{construction}
As we might expect, the inclusion problem is also undecidable for Turing machines, and this is a fact that will soon be proven in this section.
\end{construction}
\section{Reducing from Turing Machine Computations}\label{sec:reductionsTMcomputations}

\firstwords{At this point}, we're nearly done with our exploration of decision problems applied to our common models of computation. We've seen that all of the common decision problems about finite automata and other regular models are decidable, while all of the common decision problems about Turing machines are undecidable---and, in some cases, not even semidecidable. However, there remain three omissions from our landscape of decidability results: we haven't yet taken a closer look at the universality, equivalence, or inclusion problems for context-free languages.

As we will see, the class of context-free languages is interesting in that it presents a sort-of middle ground where not every problem is decidable, but where we can still decide certain problems. We observed this in the previous chapter by proving that both $\mathit{A}_{\CFG}$ and $\mathit{E}_{\CFG}$ (as well as their pushdown automaton analogues) are decidable. Unlike the class of regular languages, where the decidability of $\mathit{U}_{\DFA}$ followed via a combination of the decidability of $\mathit{E}_{\DFA}$ and closure of the regular languages under complement, the question of decidability for $\mathit{U}_{\CFG}$ isn't as straightforward. This is primarily because, as we know, the class of context-free languages isn't closed under complement. This may suggest to us that $\mathit{U}_{\CFG}$ is undecidable, but at the same time, how can we prove the undecidability of $\mathit{U}_{\CFG}$ if the only tool we have at our disposal is a many-one reduction from an undecidable problem for a completely different model?

While pondering this same question, Juris Hartmanis made a fascinating discovery that relates context-free languages to Turing machines \citeyearpar{Hartmanis1967CFLsAndTMComputations}. If we take all of the configurations of a Turing machine as it processes some input word, we can combine these configurations into something called a \emph{computation history}, which is effectively a complete record of everything the Turing machine did over the course of its computation. Hartmanis' key discovery was that the computation history of any Turing machine is itself a context-free language!

What this means for us is that, if we wish to prove that a decision problem for context-free languages is undecidable, we no longer need to rely on coming up with a much-too-powerful many-one reduction from an undecidable problem for Turing machines. Instead, we can reframe the undecidable Turing machine problem in terms of the computation histories of some Turing machine, and if we could construct a context-free grammar that generates these computation histories, then that grammar would allow us to answer the undecidable problem. Essentially, we're using a more informal notion of a ``reduction" from Turing machine computation histories to context-free grammars.

\begin{figure}[t]
\centering
\begin{subfigure}[b]{0.5\textwidth}
\centering
\begin{tikzpicture}[node distance=2.5cm, >=latex, every state/.style={fill=white}]
\node[state, initial] (q0) {$q_{0}$};
\node[state] (q1) [right of=q0] {$q_{1}$};
\node[state] (q2) [below=1cm of q0] {$q_{2}$};
\node[state, accepting] (q3) [below=1cm of q1] {$q_{3}$};

\path[-latex] (q0) edge [loop above] node {$\texttt{b} \mapsto \texttt{b}, R$} (q0);
\path[-latex] (q0) edge [bend left, above] node {$\texttt{a} \mapsto \texttt{a}, R$} (q1);
\path[-latex] (q1) edge [bend left, below] node {$\texttt{b} \mapsto \texttt{b}, R$} (q0);
\path[-latex] (q1) edge [loop above] node {$\texttt{a} \mapsto \texttt{a}, R$} (q1);
\path[-latex] (q0) edge [left] node {$\blankspace \mapsto \blankspace, R$} (q2);
\path[-latex] (q1) edge [right] node {$\blankspace \mapsto \blankspace, R$} (q3);
\end{tikzpicture}
\caption{A Turing machine $\mathcal{M}$}
\label{subfig:TMcomputationhistorymachine}
\end{subfigure}
\hspace{0.5em}
\begin{subfigure}[b]{0.45\textwidth}
\centering
\captionsetup{width=.7\linewidth}
$ \begin{aligned}
C_{1}: \ & q_{0} \, \texttt{a} \, \texttt{b} \, \texttt{a} \, \texttt{a} \, \blankspace \\
C_{2}: \ & \texttt{a} \, q_{1} \, \texttt{b} \, \texttt{a} \, \texttt{a} \, \blankspace \\
C_{3}: \ & \texttt{a} \, \texttt{b} \, q_{0} \, \texttt{a} \, \texttt{a} \, \blankspace \\
C_{4}: \ & \texttt{a} \, \texttt{b} \, \texttt{a} \, q_{1} \, \texttt{a} \, \blankspace \\
C_{5}: \ & \texttt{a} \, \texttt{b} \, \texttt{a} \, \texttt{a} \, q_{1} \, \blankspace \\
C_{6}: \ & \texttt{a} \, \texttt{b} \, \texttt{a} \, \texttt{a} \, \blankspace \, q_{3}
\end{aligned} $
\caption{The computation history of $\mathcal{M}$ on $w = \texttt{abaa}$}
\label{subfig:TMcomputationhistorylist}
\end{subfigure}

\begin{subfigure}[b]{0.9\textwidth}
\centering
\captionsetup{width=.9\linewidth}
\vspace{1.5em}
$q_{0}\texttt{abaa} \blankspace \, 
\texttt{\#} \, 
\blankspace \texttt{aab} q_{1} \texttt{a} \,
\texttt{\#} \, 
\texttt{ab} q_{0} \texttt{aa} \blankspace \, 
\texttt{\#} \, 
\blankspace \texttt{a} q_{1} \texttt{aba} \,
\texttt{\#} \, 
\texttt{abaa} q_{1} \blankspace \, 
\texttt{\#} \, 
q_{3} \blankspace \texttt{aaba}$
\vspace{0.5em}
\caption{The computation history of $\mathcal{M}$ on $w = \texttt{abaa}$, represented as a string. Observe that every second configuration is reversed}
\label{subfig:TMcomputationhistorystring}
\end{subfigure}
\caption{An example of a Turing machine and its computation history on an input word}
\label{fig:TMcomputationhistory}
\end{figure}

Before we continue along this line of thought, let's reacquaint ourselves with some Turing machine definitions. Recall from Section~\ref{subsec:TMconfigurations} that we took the configuration of a Turing machine to be a representation of the current state, tape contents, and input head position of that Turing machine at some point in its computation. In other terms, a configuration is a ``snapshot" of the Turing machine mid-computation. Depending on the current state of the machine, a configuration may be a start configuration if the current state is $q_{0}$, an accepting configuration if the current state is $q_{\text{accept}}$, or a rejecting configuration if the current state is $q_{\text{reject}}$.

Taking a sequence of configurations together, we get the aforementioned computation history, which we may represent either as a set of individual configurations or as a single string of concatenated configurations; examples of each are depicted in Figure~\ref{fig:TMcomputationhistory}.

\begin{definition}[Computation history]\label{def:computationhistory}
Given a Turing machine $\mathcal{M}$ and an input word $w$, a computation history for $\mathcal{M}$ on $w$ is a string of the form
\begin{equation*}
C_{1} \, \texttt{\#} \, (C_{2})^{\text{R}} \, \texttt{\#} \, C_{3} \, \texttt{\#} \, (C_{4})^{\text{R}} \, \texttt{\#} \, \dots,
\end{equation*}
where $\texttt{\#} \not\in \Gamma$ is a special boundary marker, $C_{i}$ is the $i$th configuration of $\mathcal{M}$, and $(C_{i})^{\text{R}}$ denotes the reversal of configuration $C_{i}$.
\end{definition}

Note that a computation history for a Turing machine $\mathcal{M}$ only exists when $\mathcal{M}$ halts on its input word $w$. As a result, the sequence of configurations $C_{1}, C_{2}, \dots, C_{n}$ always has a finite number of elements. Deterministic computations have exactly one computation history per input word, while nondeterministic computations may have multiple computation histories for the same input word.

Just like how code may or may not contain syntax errors, we can have either \emph{valid} or \emph{invalid} computation histories depending on the configurations themselves and the order in which the configurations are sequenced. A valid computation history is one whose configurations satisfy four criteria.

\begin{definition}[Valid computation history]\label{def:validcomputationhistory}
Given a Turing machine $\mathcal{M}$ and an input word $w$, a valid computation history for $\mathcal{M}$ on $w$ is a computation history where all of the following are true:
\begin{enumerate}
\item For all $1 \leq i \leq n$, each configuration $C_{i}$ is of the form $\Gamma^{*} q \Gamma^{*}$, where $q \in Q$ is a state of $\mathcal{M}$;
\item The configuration $C_{1}$ is a start configuration of the form $q_{0} w$, where $q_{0}$ is the initial state of $\mathcal{M}$ and $w \in \Sigma^{*}$;
\item The configuration $C_{n}$ is either an accepting configuration or a rejecting configuration of the form $\Gamma^{*} q_{f} \Gamma^{*}$, where $q_{f} \in \{q_{\text{accept}}, q_{\text{reject}}\}$; and
\item For all $1 \leq i \leq (n-1)$, $C_{i} \vdash C_{i+1}$.
\end{enumerate}
\end{definition}

Naturally, then, an invalid computation history belongs to the complement of the language of valid computation histories; to be precise, an invalid computation history is one whose configurations \emph{do not} satisfy at least one of the four criteria provided in Definition~\ref{def:validcomputationhistory}.

\begin{definition}[Invalid computation history]\label{def:invalidcomputationhistory}
Given a Turing machine $\mathcal{M}$ and an input word $w$, an invalid computation history for $\mathcal{M}$ on $w$ is a computation history where at least one of the following is true:
\begin{enumerate}
\item For some $1 \leq i \leq n$, configuration $C_{i}$ is not of the form $\Gamma^{*} q \Gamma^{*}$, where $q \in Q$ is a state of $\mathcal{M}$; or
\item The configuration $C_{1}$ is not a start configuration; or
\item The configuration $C_{n}$ is neither an accepting configuration nor a rejecting configuration; or
\item For some $1 \leq i \leq (n-1)$, $C_{i} \mkern-2mu\not\mkern2mu\vdash C_{i+1}$.
\end{enumerate}
\end{definition}

We put a particular emphasis on invalid computation histories here since they will be the key to obtaining our desired undecidability results for context-free languages. Indeed, to draw the same connection between Turing machines and context-free languages as Hartmanis did, we can prove that each of the four violated Turing machine configuration conditions is recognized by some context-free model of computation.

\begin{lemma}\label{lem:invalidcomputationhistoryCFL}
Given a Turing machine $\mathcal{M}$ and an input word $w$, the set of invalid computation histories of $\mathcal{M}$ on $w$ is a context-free language.

\begin{proof}
We will prove this statement by showing that each of the four individual violated criteria corresponds to some context-free language.
\begin{enumerate}
\item Where some configuration $C_{i}$ is not of the form $\Gamma^{*} q \Gamma^{*}$, we can construct a finite automaton that checks whether a nondeterministically selected configuration does not consist of a (possibly empty) prefix of symbols from $\Gamma$, followed by an encoding of a state in $\mathcal{M}$, and finally a (possibly empty) suffix of symbols from $\Gamma$. Since all regular languages are context-free, the set of configurations violating the first criterion is context-free.

\item Where the configuration $C_{1}$ is not a start configuration, we can again construct a finite automaton that checks whether this configuration does not consist of the encoding of the state $q_{0}$ followed by a (possibly empty) suffix of symbols from $\Sigma$. For the same reason as before, the set of configurations violating the second criterion is regular, and therefore context-free.

\item Where the configuration $C_{n}$ is neither an accepting nor a rejecting configuration, we can once more construct a finite automaton that functions much like in the first case, except it checks specifically for encodings of either of the states $q_{\text{accept}}$ or $q_{\text{reject}}$. Thus, the set of configurations violating the third criterion is regular, and therefore context-free.

\item To check whether $C_{i} \mkern-2mu\not\mkern2mu\vdash C_{i+1}$ for some $i$, due to the way the computation history string is formatted, we must consider two possible subcases based on whether configuration $C_{i}$ is odd-indexed or even-indexed:
	\begin{enumerate}
	\item Where $C_{i} \mkern-2mu\not\mkern2mu\vdash C_{i+1}$ for some odd $i$, we can construct a pushdown automaton that receives as input both the encoding of $\mathcal{M}$ as well as the computation history of $\mathcal{M}$ on $w$. This pushdown automaton then nondeterministically reads an even number of boundary markers \texttt{\#} from the computation history before beginning to read the configuration $C_{i}$. As the pushdown automaton reads $C_{i}$, it consults the encoding of $\mathcal{M}$ to determine some configuration $C'$ such that $C_{i} \vdash C'$, and pushes the symbols of $C'$ to its stack. Once the pushdown automaton reads another boundary marker \texttt{\#} from the computation history, it compares the configuration $(C_{i+1})^{\text{R}}$ to $C'$ symbol-by-symbol, and if there is any mismatch in symbols, it accepts.
	
	\item Where $C_{i} \mkern-2mu\not\mkern2mu\vdash C_{i+1}$ for some even $i$, we use essentially the same pushdown automaton construction as in the first subcase, making the appropriate modifications to handle $(C_{i})^{\text{R}}$.
	\end{enumerate}
In either case, as a consequence of the existence of these pushdown automata, the set of configurations violating the fourth criterion is context-free.
\end{enumerate}
Since the class of context-free languages is closed under union, taking the union of these individual context-free languages gives us our result.
\end{proof}
\end{lemma}
\section{Undecidable Problems for Context-Free Languages}\label{sec:undecidableproblemscontextfree}

\firstwords{Having established what it means} for a computation history to be invalid, and having showed that there exists a connection between the invalid computation histories of a Turing machine and our context-free models of computation, we have all we need to establish the undecidability of our remaining decision problems for context-free languages.

\subsection*{Universality Problem}

When we established the decidability of the universality problem for regular languages, we relied on Theorem~\ref{thm:FAclosurecomplement}, which showed that the class of regular languages was closed under complement. Taking the complement of a given finite automaton's language allowed us to use our existing decision procedure for $\mathit{E}_{\DFA}$ to decide $\mathit{U}_{\DFA}$, as shown in Theorem~\ref{thm:UDFAdecidable}.

For context-free languages, we know by Theorem~\ref{thm:ECFGdecidable} that $\mathit{E}_{\CFG}$ is decidable, so one may be tempted to try a similar approach to establish the supposed decidability of $\mathit{U}_{\CFG}$. However, Theorem~\ref{thm:CFLnonclosurecomplement} tells us that the class of context-free languages is \emph{not} closed under complement, and so this approach is a non-starter.

Instead, using a straightforward argument that relies on the connection between invalid computation histories and context-free languages, we can show that $\mathit{U}_{\CFG}$ is in fact \emph{undecidable}---the first such result we have for a model weaker than Turing machines!

\begin{theorem}\label{thm:UCFGundecidable}
$\mathit{U}_{\CFG}$ is undecidable.

\begin{proof}
Suppose that $\mathcal{M}$ is an arbitrary Turing machine. By Lemma~\ref{lem:invalidcomputationhistoryCFL}, we can construct a context-free grammar $G$ having the property that $L(G) = \Sigma^{*}$ if and only if $L(\mathcal{M}) = \emptyset$. This is because if $\mathcal{M}$ accepts no input words, then every input word given to $\mathcal{M}$ would produce an invalid computation history, and the aforementioned lemma tells us that the set of invalid computation histories of $\mathcal{M}$ is a context-free language.

However, if it were possible for us to check whether $L(G) = \Sigma^{*}$, then it would also become possible for us to decide whether $L(\mathcal{M}) = \emptyset$, and we know by Theorem~\ref{thm:ETMundecidable} that $\mathit{E}_{\TM}$ is undecidable. Therefore, $\mathit{U}_{\CFG}$ must also be undecidable.
\end{proof}
\end{theorem}

At this point, take a moment to reflect on the argument we just used to establish the undecidability of $\mathit{U}_{\CFG}$. Unlike our previous undecidability proofs using many-one reductions, we didn't need to construct any Turing machines here. Rather, we simply relied on a property that is true of any Turing machine: the set of invalid computation histories is a context-free language, and so we can construct a grammar for this language.

\begin{dangerous}
Note that nowhere did we say the language of our grammar $G$ \emph{is} $\Sigma^{*}$; all we said is that $L(G) = \Sigma^{*}$ \emph{if and only if} $L(\mathcal{M}) = \emptyset$. The language of $G$ depends on whatever arbitrary Turing machine $\mathcal{M}$ we are given, but if we had a method of deciding whether the language of $G$ was universal, then that same method could decide whether the language of this Turing machine $\mathcal{M}$ is empty.
\end{dangerous}

Naturally, since context-free grammars and pushdown automata are equivalent, this negative decidability result transfers over to our other context-free model of computation.

\begin{corollary}\label{cor:UPDAundecidable}
$\mathit{U}_{\PDA}$ is undecidable.
\end{corollary}

\subsection*{Equivalence Problem}

Recall that we can test the equivalence of two languages $L_{1}$ and $L_{2}$ by testing the emptiness of their symmetric difference, where the symmetric difference operation is defined in terms of union, intersection, and complement:
\begin{equation*}
L_{1} \symdiff L_{2} = \left( L_{1} \cap \overline{L_{2}} \right) \cup \left( \overline{L_{1}} \cap L_{2} \right).
\end{equation*}
Now, again, we know by Theorem~\ref{thm:ECFGdecidable} that we can test the emptiness of a context-free grammar, so we can test emptiness for two individual, separate context-free grammars just as easily. However, the class of context-free languages is not closed under intersection (by Theorem~\ref{thm:CFLnonclosureintersection}), nor is it closed under complement (by Theorem~\ref{thm:CFLnonclosurecomplement}). Thus, we have no way of testing the emptiness of the symmetric difference of two context-free grammars, and so it is hopeless for us to expect that we can decide the equivalence problem for context-free grammars.

But how do we formally prove the undecidability of $\mathit{EQ}_{\CFG}$? We need only fix one of our two context-free grammars to be ``special", which will then allow us to use what we know about $\mathit{U}_{\CFG}$ to arrive at our desired outcome.

\begin{theorem}\label{thm:EQCFGundecidable}
$\mathit{EQ}_{\CFG}$ is undecidable.

\begin{proof}
Let $G_{1}$ be an arbitrary context-free grammar, and let $G_{2}$ be a context-free grammar that generates all words over its alphabet $\Sigma$. If it were possible for us to check whether $L(G_{1}) = L(G_{2})$, then it would also become possible for us to decide whether $L(G_{1}) = \Sigma^{*}$, and we know by Theorem~\ref{thm:UCFGundecidable} that $\mathit{U}_{\CFG}$ is undecidable. Therefore, $\mathit{EQ}_{\CFG}$ must also be undecidable.
\end{proof}
\end{theorem}

In this proof, we have reduced in the sense that we made an instance of the decision problem $\mathit{EQ}_{\CFG}$ appear like an instance of the decision problem $\mathit{U}_{\CFG}$, and since the latter problem is undecidable, so too must the former problem be undecidable.

As we would expect, the same outcome holds for pushdown automata.

\begin{corollary}\label{cor:EQPDAundecidable}
$\mathit{EQ}_{\PDA}$ is undecidable.
\end{corollary}

\subsection*{Inclusion Problem}

\begin{construction}
Yes, the inclusion problem is also undecidable for context-free languages. This fact will be proved in due time.
\end{construction}
\futuresection{Post's Correspondence Problem}\label{sec:postcorrespondenceproblem}

\begin{construction}
I plan to write a short section on Post's correspondence problem~\citeyearpar{Post1946RecursivelyUnsolvableProblem}, its undecidability, and its applicability to showing other problems are undecidable.
\end{construction}
\futuresection{Rice's Theorem}\label{sec:ricestheorem}

\begin{construction}
To close this chapter, I'll discuss Rice's theorem~\citeyearpar{Rice1953ClassesRecursivelyEnumerable}: a generalization of the halting problem that reveals every nontrivial semantic property of Turing machines is undecidable! Hopefully, by this point, the reader's hope will not have been totally wiped out before getting to later chapters.
\end{construction}

\unnumberedsection{Chapter Notes}

\firstwords{The body of work} pertaining to undecidability is vast, and the papers therein are often dense with mathematical and logical notation. \citet{Davis1965TheUndecidable} collects many of the major papers in his anthology and provides brief commentaries for each paper.

\begin{enumerate}
\item[\ref{sec:manyonereductions}.] Many-one reductions were first studied by~\citet{Post1944RecursivelyEnumerableSets}. As we noted, the name ``many-one" reflects the fact that a reduction is a function, but Post chose this particular name to set this notion of reducibility apart from his more restrictive notion of \emph{one-one reducibility}, where the function is constrained to be injective. \citet{Shapiro1956DegreesOfComputability} later studied ideas similar to those of Post, but referred to many-one reducibility as \emph{strong reducibility}.

Other authors refer to many-one reductions under a different name; for instance, \cite{Sipser2013TheoryOfComputation3rdEd} calls them \emph{mapping reductions}.

The result stated in Lemma~\ref{lem:manyonereductionreflexivetransitive} follows more or less directly from the definition of a many-one reduction, but \citet{Shapiro1956DegreesOfComputability} explicitly lists these same properties in his paper.

\item[\ref{sec:haltingproblem}.] The Goldbach conjecture was first formulated in a letter from Christian Goldbach to Leonhard Euler written in 1742. In our discussion, we noted that the conjecture was verified for all even numbers up to $4 \times 10^{18}$. This verification was performed by~\citet*{OliveiraESilva2014EmpricalVerificationGoldbach}.

The earliest mention of a decision problem that asks about ``a machine which \textup{[\,\dots]} eventually stops" appears in Kleene's book~\citeyearpar[chapter XIII, section 71]{Kleene1952IntroductionToMetamathematics}; there, Kleene further asserts that the problem is undecidable. The term ``halting problem" was introduced by~\citet[chapter 5, section 2]{Davis1958ComputabilityUnsolvability}, who again proved that the problem is undecidable. For more information, \citet{Lucas2021OriginsHaltingProblem} has written a detailed survey on the history of the halting problem.

Although a number of authors claim that Alan Turing originally studied the halting problem in his \citeyear{Turing1936OnComputableNumbers} paper, this is not the case! While Turing did prove the undecidability of a \emph{satisfactoriness problem} in his paper, and while this problem does ask whether a given machine $\mathcal{M}$ is ``circular" or ``circle-free", this problem is not equivalent to the halting problem. Indeed, using the notion of \emph{degrees of unsolvability} \citep{Post1944RecursivelyEnumerableSets}, we note that while the halting problem belongs to the class of recursively enumerable sets of numbers (i.e., $\mathit{H}_{\TM}$ is of degree $\mathbf{0}'$), the satisfactoriness problem belongs to a strictly larger class (i.e., satisfactoriness is of degree $\mathbf{0}''$). See the aforementioned survey by \citet{Lucas2021OriginsHaltingProblem} for a proof of this fact.

An alternative, more Seussical proof of the undecidability of the halting problem was published by \citet{Pullum2000LoopSnooper}. \par
\epigraph{No program can say what another will do.\par
Now, I won't just assert that, I'll prove it to you:\par
I will prove that although you might work till you drop,\par
you can't predict whether a program will stop.}{Geoffrey Pullum}{Scooping the Loop Snooper}{}
\vspace{1em}

\item[\ref{sec:moreundecidableTM}.] There are many (many) more undecidable problems than those we have discussed in this section. The book chapter by \citet{Davis1977UnsolvableProblems} provides a nice overview of a number of undecidable problems in computer science and fields beyond, including group theory, combinatorics, and number theory. \citet{Harkleroad1996WhatComputersCantDo} gives a gentler introduction to some of the same undecidability results.

\item[\ref{sec:reductionsTMcomputations}.] As we noted, \citet{Hartmanis1967CFLsAndTMComputations} was the first to draw a connection between context-free languages and the computations of Turing machines. In tandem with the result that all invalid computation histories of a Turing machine are context-free, Hartmanis further showed that all valid computation histories of a Turing machine can be represented as the intersection of two context-free languages.

\item[\ref{sec:undecidableproblemscontextfree}.] Both the equivalence problem and the inclusion problem were shown to be undecidable by \citet*{BarHillel1961FormalPropertiesPhraseStructureGrammars}, whose proofs relied on a reduction from the Post correspondence problem. The undecidability of the universality problem follows from its relationship to the equivalence problem. \citet{Hartmanis1967CFLsAndTMComputations} gave alternative proofs of these same facts by reducing from the halting problem rather than the Post correspondence problem.

\item[\ref{sec:postcorrespondenceproblem}.] \textsl{Chapter notes will be added when this section is written.}

\item[\ref{sec:ricestheorem}.] \textsl{Chapter notes will be added when this section is written.}
\end{enumerate}