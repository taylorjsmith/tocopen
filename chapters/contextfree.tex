\chapter{Context-Free Languages}\label{chap:contextfree}

\firstwords{Recall that}, in our discussion on regular languages, we introduced the notion of a regular expression. This expression essentially performed a kind of pattern matching to accept text in a certain form and reject all other text not of that form.

We can take this idea of matching patterns in text and modify it to work not just for individual words, but for the structure and composition of the entire text. This ability comes in the form of \emph{grammars}, which provide us with a set of \emph{rules} that we can follow to produce words that belong to a certain language. If a grammar produces all and only those words belonging to a certain language, then we say the grammar \emph{generates} that language.

The idea of creating grammars for languages is nothing new; linguists have been using grammars to study natural languages for centuries, dating all the way back to the work of the ancient grammarian \citet{Panini500BCEAstadhyayi}, who created an early grammar for Sanskrit. Mathematicians developed \emph{rewriting rules} in the early 1900s to transform strings of symbols, and with the mid-century advent of computer science, grammars began to be applied to formal languages and programming languages.

If you look at the specification manual for any programming language, you will likely find tucked away somewhere in the documentation a grammar for that language. This grammar, which could number into the tens of pages, describes precisely what the structure of a program written in that language should look like. In fact, this grammar is exactly what the compiler relies on to check for syntax errors in your program!

\begin{figure}
\centering
\begin{minipage}{0.8\linewidth}
\textit{Statement:} \\
\hspace*{1.5em}	\textit{Block} \\
\hspace*{1.5em}	\texttt{assert} \textit{Expression [} \texttt{:} \textit{Expression]} \texttt{;} \\
\hspace*{1.5em}	\texttt{if} \textit{ParExpression Statement [}\texttt{else} \textit{Statement]} \\
\hspace*{1.5em}	\texttt{for (} \textit{ForControl} \texttt{)} \textit{Statement} \\
\hspace*{1.5em}	\texttt{while} \textit{ParExpression Statement} \\
\hspace*{1.5em}	\texttt{do} \textit{Statement} \texttt{while} \textit{ParExpression} \texttt{;} \\
\hspace*{1.5em}	\texttt{try} \textit{Block ( Catches \textpipe\ [Catches]} \texttt{finally} \textit{Block )} \\
\hspace*{1.5em}	\texttt{switch} \textit{ParExpression} \texttt{\{} \textit{SwitchBlockStatementGroups} \texttt{\}} \\
\hspace*{1.5em}	\texttt{synchronized} \textit{ParExpression Block} \\
\hspace*{1.5em}	\texttt{return} \textit{[Expression]} \texttt{;} \\
\hspace*{1.5em}	\texttt{throw} \textit{Expression} \texttt{;} \\
\hspace*{1.5em}	\texttt{break} \textit{[Identifier]} \\
\hspace*{1.5em}	\texttt{continue} \textit{[Identifier]} \\
\hspace*{1.5em}	\texttt{;} \\
\hspace*{1.5em}	\textit{StatementExpression} \texttt{;} \\
\hspace*{1.5em}	\textit{Identifier } \texttt{:} \textit{ Statement}
\end{minipage}
\caption{An excerpt from the grammar in the \textit{Java Language Specification}}
\label{fig:javalanguage}
\end{figure}

As an example, let's consider the excerpt depicted in Figure~\ref{fig:javalanguage}, taken from the grammar found in the \textit{Java Language Specification}~\citep*[chapter 18]{Gosling2005JavaLanguageSpecification}. This part of the grammar checks code blocks such as if-else statements, for and while loops, and so on. Every \textit{italicized} word corresponds to another rule in the grammar, while \texttt{monospaced} words are language keywords. For instance, the \texttt{if} rule checks that every if-else block in a program follows the syntax that the compiler expects: it begins with the keyword \texttt{if} together with a \textit{parenthesized expression}, followed by a \textit{statement}, and ending with an optional \texttt{else} block.

\section{Context-Free Grammars}\label{sec:contextfreegrammars}

\firstwords{The Java grammar} is an example of a \emph{context-free grammar}. Such a grammar consists of a set of rules that we can use, in this instance, to generate valid programs in Java. These rules take on a very general form: observe, for example, that we can replace a \textit{Statement} by a \textit{Block}, or by the line \texttt{return} \textit{[Expression]} \texttt{;}, or by a number of other combinations of keywords and rules, all of which are specified by the lines of the grammar following the \textit{Statement} label.

Before we look at some more examples, let's formalize the notion of a context-free grammar. To construct a grammar, we need only four elements.

\begin{definition}[Context-free grammar]\label{def:contextfreegrammar}
A context-free grammar is a tuple $(V, \Sigma, R, S)$, where
\begin{itemize}
\item $V$ is a finite set of elements called \emph{nonterminal symbols};
\item $\Sigma$ is a finite set of elements called \emph{terminal symbols}, where $\Sigma \cap V = \emptyset$;
\item $R$ is a finite set of \emph{rules}, where each rule consists of a nonterminal on the left-hand side and a combination of nonterminals and terminals on the right-hand side; and
\item $S \in V$ is the \emph{start nonterminal}.
\end{itemize}
\end{definition}

In a context-free grammar, the set of nonterminal symbols $V$ correspond to parts of a word that we have yet to ``fill in" with terminal symbols from $\Sigma$. The set of rules $R$ tell us how we can perform this ``filling in". If we have a rule of the form $A \rightarrow \alpha$, then we can replace any instance of the symbol $A$ in our word with whatever symbols make up $\alpha$. The start nonterminal $S$ is self-explanatory; it is the first thing in our word that we ``fill in".

Returning to our Java grammar excerpt in Figure~\ref{fig:javalanguage}, we see that (for example) some of the nonterminals in the grammar include \textit{Statement}, \textit{Block}, \textit{Identifier}, and \textit{ParExpression}, while some of the terminals include \texttt{if}, \texttt{while}, \texttt{for}, and \texttt{;} (semicolon).

Importantly, we have in our definition of a context-free grammar that $\Sigma \cap V = \emptyset$; that is, the set of terminals and the set of nonterminals must be disjoint. This is to prevent the grammar from confusing terminals and nonterminals, and this is exactly why the Java language designers used uppercase letters in their nonterminals and lowercase letters in their terminals.

\subsection{Language of a Context-Free Grammar}

The sequence of rule applications we follow beginning with the start nonterminal $S$ and ending with a completed word containing symbols from $\Sigma$ is called a \emph{derivation}. Each word of the form $(V \cup \Sigma)^{*}$ produced during a derivation is sometimes referred to as a \emph{sentential form}.

For any nonterminal $A$ and terminals $u$, $w$, and $v$, if we have a rule $A \rightarrow w$ in our grammar and some step of our derivation takes us from $uAv$ to $uwv$, then we say that $uAv$ \emph{yields} $uwv$ and we write $uAv \Rightarrow uwv$. We can represent a sequence of ``yields" relations using similar notation; given words $x$ and $y$, if $x = y$ or if there exists a sequence $x_{1}, x_{2}, \dots, x_{k}$ where $k \geq 0$ such that
\begin{equation*}
x \Rightarrow x_{1} \Rightarrow x_{2} \Rightarrow \dots \Rightarrow x_{k} \Rightarrow y,
\end{equation*}
then we write $x \Rightarrow^{*} y$. This is very similar to the Kleene star notation, where the star indicates zero or more ``yields" relations taking us from $x$ to $y$.

With this, we can define the \emph{language of a grammar} $G$ over an alphabet $\Sigma$ by $L(G) = \{w \in \Sigma^{*} \mid S \Rightarrow^{*} w\}$. In other terms, the language of a grammar contains all words that can be derived by that grammar beginning with the start nonterminal $S$.

\begin{example}\label{ex:equalzeroone}
Consider the context-free grammar where $V = \{S, A\}$, $\Sigma = \{\texttt{a}, \texttt{b}\}$, and $R$ contains two rules:
\begin{align*}
S	&\rightarrow \texttt{a}A\texttt{b} \\
A	&\rightarrow \texttt{a}A\texttt{b} \mid \epsilon
\end{align*}

Using this context-free grammar, we can generate words like
\begin{align*}
&S \Rightarrow \highlightmath{\texttt{a}A\texttt{b}} \Rightarrow \texttt{a}\highlightmath{\epsilon}\texttt{b} = \texttt{ab}, \\
&S \Rightarrow \highlightmath{\texttt{a}A\texttt{b}} \Rightarrow \texttt{a}\highlightmath{\texttt{a}A\texttt{b}}\texttt{b} \Rightarrow \texttt{aa}\highlightmath{\epsilon}\texttt{bb} = \texttt{aabb}, \text{ and} \\
&S \Rightarrow \highlightmath{\texttt{a}A\texttt{b}} \Rightarrow \texttt{a}\highlightmath{\texttt{a}A\texttt{b}}\texttt{b} \Rightarrow \texttt{aa}\highlightmath{\texttt{a}A\texttt{b}}\texttt{bb} \Rightarrow \texttt{aaa}\highlightmath{\epsilon}\texttt{bbb} = \texttt{aaabbb},
\end{align*}
and so on. For each step, the highlighted symbols indicate which symbols were added at that step. We get things started by replacing the $S$ nonterminal by $\texttt{a}A\texttt{b}$, and from there we may replace the $A$ nonterminal as many times as we like.

This context-free grammar generates all words over the alphabet $\Sigma = \{\texttt{a}, \texttt{b}\}$ where the number of \texttt{a}s is equal to the number of \texttt{b}s, where there is at least one \texttt{a} and one \texttt{b}, and where all \texttt{a}s come before any \texttt{b}s in the word. Thus, the language of this grammar is $L_{\texttt{a}=\texttt{b}} = \{\texttt{a}^{n}\texttt{b}^{n} \mid n \geq 1\}$.
\end{example}

Observe that the rule $A$ in Example~\ref{ex:equalzeroone} included a vertical bar. This is simply a shorthand for writing multiple rules where each rule contains $A$ on the left-hand side. Writing $A \rightarrow \texttt{a}A\texttt{b} \mid \epsilon$ is therefore equivalent to writing
\begin{align*}
A 	&\rightarrow \texttt{a}A\texttt{b} \\
A 	&\rightarrow \epsilon
\end{align*}

There are very few limitations we must abide by when we write rules for a context-free grammar. All we need to ensure is that the left-hand side of each rule consists of exactly one nonterminal by itself. The right-hand side of each rule can contain any combination of terminals and nonterminals, including the empty word $\epsilon$.

\begin{example}\label{ex:balancedparens}
Consider the context-free grammar where $V = \{S\}$, $\Sigma = \{\texttt{(}, \texttt{)}\}$, and $R$ contains one rule:
\begin{align*}
S	&\rightarrow \texttt{(}S\texttt{)} \mid SS \mid \epsilon
\end{align*}
This rule allows us to surround an occurrence of $S$ with parentheses, to ``duplicate" an occurrence of $S$, or to replace some occurrence of $S$ with $\epsilon$, effectively removing that occurrence of $S$ from the derivation.

Using this context-free grammar, we can generate a word like
\begin{align*}
S 	&\Rightarrow \highlightmath{SS} \\
	&\Rightarrow \highlightmath{\texttt{(}S\texttt{)}}S \\
	&\Rightarrow \texttt{(}\highlightmath{\texttt{(}S\texttt{)}}\texttt{)}S \\
	&\Rightarrow \texttt{((}\highlightmath{\epsilon}\texttt{))}S \\
	&\Rightarrow \texttt{(())}S \\
	&\Rightarrow \texttt{(())}\highlightmath{\texttt{(}S\texttt{)}} \\
	&\Rightarrow \texttt{(())(}\highlightmath{\epsilon}\texttt{)} = \texttt{(())()}.
\end{align*}
Again, the highlighted symbols indicate which symbols were added at a given step.

This context-free grammar generates all words over the alphabet $\Sigma = \{\texttt{(}, \texttt{)}\}$ where each word contains \emph{balanced parentheses}: every opening parenthesis is matched by a closing parenthesis, and each pair of parentheses is correctly nested. We can express the language of the grammar as
\begin{align*}
L_{\texttt{()}} = \{w \in \{\texttt{(}, \texttt{)}\}^{*} \mid &\text{ all prefixes of } w \text{ contain no more \texttt{)}s than \texttt{(}s, } \\ 
	&\text{ and } |w|_{\texttt{(}} = |w|_{\texttt{)}}\}.
\end{align*}
\end{example}

\begin{remark}
Languages of balanced symbols are also known as \emph{Dyck languages}, named for the mathematician Walther von Dyck, who studied the word problem for free groups. This problem can be thought of as a language where we must balance both parentheses \texttt{(} \texttt{)} and brackets \texttt{[} \texttt{]}. In the computer science context, such languages were originally called \emph{D-events} \citep{Schutzenberger1962CertainElementaryFamilies} before later being termed \emph{Dyck sets} \citep{Schutzenberger1963OnCFLsPDAs}.
\end{remark}

\subsubsection*{Context-Free Languages}

With our notion of a context-free grammar, it's easy for us to define a \emph{context-free language}. Just like we defined a regular language to be a language represented by a regular expression, we can define a context-free language in terms of a context-free grammar.

\begin{definition}[Context-free language]\label{def:contextfreelanguage}
If some language $L$ is generated by a context-free grammar, then $L$ is context-free.
\end{definition}

Thus, both the language of words $\texttt{a}^{n}\texttt{b}^{n}$ and the language of balanced parentheses are context-free languages. As a shorthand, we denote the class of languages generated by a context-free grammar by \CFG.

The class of context-free languages is remarkably less restrictive than the class of regular languages and, as we've seen, context-freeness allows us to perform certain simple actions like counting or matching symbols. Let's now consider a couple of other examples of context-free languages and their grammars.

\begin{example}
Consider the language $L = \{\texttt{a}^{2i}\texttt{b}^{i}\texttt{c}^{j+2} \mid i, j \geq 0\}$ over the alphabet $\Sigma = \{\texttt{a}, \texttt{b}, \texttt{c}\}$. Here, we can see that the counts of \texttt{a}s and \texttt{b}s are related, while the number of \texttt{c}s is independent of the number of \texttt{a}s and \texttt{b}s.

Let's construct a context-free grammar generating words in $L$. Evidently, we will need two kinds of rules: one rule will generate the \texttt{a}s and \texttt{b}s together, while the other rule will generate the \texttt{c}s. We can use the start nonterminal to apply these rules in the correct order. Our context-free grammar will therefore look like the following:
\begin{align*}
S	&\rightarrow UV \\
U	&\rightarrow \texttt{aa}U\texttt{b} \mid \epsilon \\
V	&\rightarrow \texttt{c}V \mid \texttt{cc}
\end{align*}
Let's now take a look at each rule in turn.
\begin{itemize}
\item The first rule, $S$, ensures that we apply the $U$ rule before the $V$ rule. This in turn ensures that all \texttt{a}s and \texttt{b}s occur before the \texttt{c}s in the generated word.
\item The second rule, $U$, either recursively produces two \texttt{a}s and one \texttt{b} or produces the empty word. This ensures that we maintain the correct count of $2i$ \texttt{a}s and $i$ \texttt{b}s.
\item Finally, the third rule, $V$, either recursively produces one \texttt{c} or produces the symbols \texttt{cc}. This ensures that we have exactly $j+2$ \texttt{c}s in our generated word.
\end{itemize}
\end{example}

\begin{example}
Recall our context-free language $L_{\texttt{a}=\texttt{b}} = \{\texttt{a}^{n}\texttt{b}^{n} \mid n \geq 1\}$. Here, let's consider a more general language:
\begin{equation*}
L_{\text{mixed}\texttt{a}=\texttt{b}} = \{w \in \{\texttt{a}, \texttt{b}\}^{*} \mid |w|_{\texttt{a}} = |w|_{\texttt{b}}\}.
\end{equation*}
Observe that the main difference with this language is that the order of \texttt{a}s and \texttt{b}s no longer matters; we just need the same count of \texttt{a}s and \texttt{b}s. Can we construct a context-free grammar for $L_{\text{mixed}\texttt{a}=\texttt{b}}$?

Since order no longer matters, we just need our context-free grammar to generate a pair of \texttt{a}s and \texttt{b}s each time we add terminal symbols. For this, we can use essentially the same rule as we used in our context-free grammar for $L_{\texttt{a}=\texttt{b}}$: $S \rightarrow \texttt{a}S\texttt{b} \mid \texttt{b}S\texttt{a}$. We also need a rule that allows us to mix the order of \texttt{a}s and \texttt{b}s; for instance, to place two \texttt{a}s or two \texttt{b}s next to each other, or to generate words with matching first and last symbols (like \texttt{abba}). For this, we can use a rule similar to one we included in our context-free grammar for $L_{\texttt{()}}$: $S \rightarrow SS$.

Thus, our context-free grammar will look like the following:
\begin{align*}
S	&\rightarrow SS \mid \texttt{a}S\texttt{b} \mid \texttt{b}S\texttt{a} \mid \epsilon
\end{align*}
\end{example}

As an aside, one very common and popular question asks why these grammars and languages are given the name ``context-free". To understand where this name comes from, we must take a closer look at the form of any rule in a context-free grammar. Definition~\ref{def:contextfreegrammar} states that each rule in a context-free grammar consists of ``a nonterminal on the left-hand side and a combination of nonterminals and terminals on the right-hand side", and if we were to represent this using symbols, we would get rules that are of the form
\begin{equation*}
A \rightarrow \alpha,
\end{equation*}
where $A \in V$ and $\alpha \in (V \cup \Sigma)^{*}$. Now, if during some derivation we wish to replace an occurrence of the nonterminal $A$ with whatever symbols are in $\alpha$, we can just substitute $A$ for the symbols in $\alpha$ directly. In other words, the symbols surrounding $A$ (also known as the \emph{context}) don't have any effect on the substitution, and so the process of replacing $A$ with $\alpha$ is \emph{free of context}! By contrast, if we had rules of the form
\begin{equation*}
\beta A \gamma \rightarrow \beta \alpha \gamma,
\end{equation*}
where $A \in V$, $\alpha \in (V \cup \Sigma)^{+}$, and $\beta, \gamma \in (V \cup \Sigma)^{*}$, then we could replace $A$ with $\alpha$ only when $A$ appears within the context $\beta \, \square \, \gamma$. This gives rise to the notions of \emph{context-sensitive grammars} and \emph{context-sensitive languages}, which are interesting in their own right but omitted from our discussion here.

%\subsection{Constructing Context-Free Grammars}

%\begin{construction}
% I would like to write a short section about how one can design a context-free grammar for a given language/application/etc.
%\end{construction}

\subsection{Ambiguity}

If we are given a derivation of a word for some context-free grammar, we need not always represent it in a linear fashion like we did in previous examples. We could alternatively represent it as a tree structure, where the root of the tree corresponds to the start nonterminal $S$ and each branch of the tree adds a new nonterminal or terminal symbol. We refer to such trees as \emph{parse trees}.

Parse trees are very familiar to linguists: the idea is used all the time to break down sentences or phrases into their constituent components, like nouns, verbs, and so on. In doing so, linguists are able to study the structures of sentences in different languages. For example, consider the parse tree for an English sentence depicted in Figure~\ref{fig:parsetree}.
\begin{figure}
\centering
\scalebox{0.9}{
\Tree[.S
		[.NP
			[.Det \textit{the} ]
			[.Nom
				[.Adj \textit{intelligent} ]
				[.N \textit{professor} ]
			]
		]
		[.VP
			[.V \textit{illustrated} ]
			[.NP
				[.Det \textit{a} ]
				[.Nom
					[.Adj \textit{beautiful} ]
					[.N \textit{tree} ]
				]
			]
		]
	]
}
\caption{A parse tree for an English sentence}
\label{fig:parsetree}
\end{figure}
In this tree, the sentence (S) is broken down into a noun phrase (NP) and a verb phrase (VP); the noun phrase is broken down further into a determiner (Det) and a nominal (Nom); and so on. There are all kinds of rules specifying exactly how we can break down English sentences in this way.

In every parse tree, the root of the tree is the start nonterminal $S$, the leaves of the tree contain terminal symbols from $\Sigma$ (or $\epsilon$), and all other vertices of the tree contain nonterminal symbols from $V$. If a parse tree contains an internal (non-leaf) vertex $A$, and all the children of the vertex $A$ are labelled $a_{1}, a_{2}, \dots, a_{n}$, then the underlying grammar's rule set must contain a rule of the form $A \rightarrow a_{1}a_{2} \dots a_{n}$.

For most grammars we deal with, there exists exactly one way to generate any given word in the language of the grammar, and thus exactly one parse tree for each word. However, this is not always the case. There are some grammars that allow us to generate the same word in more than one way.

Perhaps one of the most well-known examples where this is the case---namely, from viral posts online that ask you to simplify $8 \div 2 (2 + 2)$ or something similar---is the grammar generating the language of arithmetic expressions. If you recall grade school mathematics, you'll remember that there is an order of operations that specify the order in which we should apply arithmetic operations in a given expression. We first evaluate expressions in parentheses, then exponents, then multiplications and divisions, and finally additions and subtractions.

Let's consider a simplified set of operations, where we only use parentheses, addition, and multiplication. The grammar generating the language of arithmetic expressions using these three operators together with the standard set of numbers is as follows:
\begin{align}
\label{eq:grammararithmeticbad}
\begin{split}
S	&\rightarrow \texttt{(} S \texttt{)} \\
S	&\rightarrow S + S \\
S	&\rightarrow S \times S \\
S	&\rightarrow \texttt{num}
\end{split}
\end{align}

If we consider the expression $\texttt{num} \times \texttt{num} + \texttt{num}$, we discover that there exists more than one way to generate this expression, depending on whether we apply the $+$ rule or the $\times$ rule first. This is evidenced by the fact that there exist two parse trees for the same expression:

\begin{center}
\scalebox{0.9}{
\Tree[.$S$ 
		[.$S$ 
			[.$S$ 
				[.\texttt{num} ]
			]
			[.$\times$ ]
			[.$S$ 
				[.\texttt{num} ]
			]
		]
		[.$+$ ]
		[.$S$ 
			[.\texttt{num} ]
		]
	]
\hspace{1.5cm}
\Tree[.$S$ 
		[.$S$ 
			[.\texttt{num} ]
		]
		[.$\times$ ]
		[.$S$ 
			[.$S$ 
				[.\texttt{num} ]
			]
			[.$+$ ]
			[.$S$ 
				[.\texttt{num} ]
			]
		]
	]
}
\end{center}

We don't need to do anything tricky in order to obtain these different parse trees. In fact, both parse trees can be obtained simply by applying rules to each nonterminal from left to right; that is, at some level of the parse tree where there exists two nonterminals, we apply a rule to the first (left) nonterminal before the second (right) nonterminal. This process is known as a \emph{leftmost derivation}.

If there exists some word in the language of a grammar for which there is more than one leftmost derivation of that word, then we say that the word is derived \emph{ambiguously}. Likewise, the grammar producing that word is itself ambiguous.

\begin{definition}[Ambiguous context-free grammar]
A context-free grammar $G$ is ambiguous if there exists some word $w \in L(G)$ that can be derived ambiguously.
\end{definition}

\begin{example}
The grammar from Example~\ref{ex:balancedparens} generating our language of words with balanced parentheses is ambiguous. Consider again the word \texttt{(())()}. There exist two different parse trees corresponding to leftmost derivations of this word:

\begin{center}
\scalebox{0.9}{
\Tree[.$S$ 
		[.$S$ 
			[.\texttt{(} ]
			[.$S$ 
				[.\texttt{(} ]
				[.$S$ 
					[.$\epsilon$ ]
				]
				[.\texttt{)} ]
			]
			[.\texttt{)} ]
		]
		[.$S$ 
			[.\texttt{(} ]
			[.$S$ 
				[.$\epsilon$ ]
			]
			[.\texttt{)} ]
		]
	]
\hspace{1.5cm}
\Tree[.$S$ 
		[.$S$ 
			[.$S$ 
				[.$\epsilon$ ]
			]
			[.$S$ 
				[.\texttt{(} ]
				[.$S$ 
					[.\texttt{(} ]
					[.$S$ 
						[.$\epsilon$ ]
					]
					[.\texttt{)} ]
				]
				[.\texttt{)} ]
			]
		]
		[.$S$ 
			[.\texttt{(} ]
			[.$S$ 
				[.$\epsilon$ ]
			]
			[.\texttt{)} ]
		]
	]
}
\end{center}
\end{example}

\subsubsection*{Reducing and Removing Ambiguity}

In certain cases, if we have an ambiguous context-free grammar, then we can create a context-free grammar for the same language that has reduced, or even no, ambiguity.

As an example, recall our three-operation arithmetic grammar:
\begin{align}
\begin{split}
S	&\rightarrow \texttt{(} S \texttt{)} \\
S	&\rightarrow S + S \\
S	&\rightarrow S \times S \\
S	&\rightarrow \texttt{num}
\end{split}
\tag{\ref{eq:grammararithmeticbad}}
\end{align}
Nothing in this grammar forces us to use one rule before another, so we end up being able to derive the same word via different sequences of rule applications. However, we can construct an unambiguous grammar simply by adding a little more structure to our rules---specifically, by adding a few more nonterminals:
\begin{align}
\label{eq:grammararithmeticgood}
\begin{split}
S	&\rightarrow E \\
E	&\rightarrow E + T \mid T \\
T	&\rightarrow T \times F \mid F \\
F	&\rightarrow \texttt{(} E \texttt{)} \mid \texttt{num}
\end{split}
\end{align}
Now, each nonterminal plays a particular role. The nonterminal $S$, as usual, serves as our starting point and produces an expression, $E$. Each expression consists of subexpressions $E$ or additive terms $T$. Likewise, each term consists of subterms $T$ or multiplicative factors $F$. Finally, each factor can be either a \texttt{num} or a parenthesized subexpression, starting the whole process over again.

Our revised grammar now allows us to draw one unambiguous parse tree for the expression $\texttt{num} \times \texttt{num} + \texttt{num}$:

\begin{center}
\scalebox{0.9}{
\Tree[.$S$ 
		[.$E$ 
			[.$E$ 
				[.$T$ 
					[.$T$ 
						[.$F$ 
							[.\texttt{num} ]
						]
					]
					[.$\times$ ]
					[.$F$ 
						[.\texttt{num} ]
					]
				]
			]
			[.$+$ ]
			[.$T$ 
				[.$F$ 
					[.\texttt{num} ]
				]
			]
		]
	]
}
\end{center}

We won't verify here that this grammar is in fact equivalent to our original one, though we can intuit that they both generate the same language. Suffice it to say that, with this revised grammar, we're able to guarantee that the addition rule is always applied before the multiplication rule.

\begin{remark}
The reason why we didn't verify that our two context-free grammars are equivalent is because the problem of determining the equivalence of context-free grammars is impossible for a computer to solve in general. We will discuss this in greater depth in Sections~\ref{sec:reductionsTMcomputations} and \ref{sec:undecidableproblemscontextfree}.
\end{remark}

\subsubsection*{Inherent Ambiguity}

Unfortunately, there is no general procedure or algorithm for removing ambiguity from a context-free grammar; indeed, it isn't even possible to remove ambiguity in some cases. Some context-free languages are \emph{inherently ambiguous}, meaning that any grammar generating the language will have some unavoidable ambiguous component to it.

\begin{example}
Let $\Sigma = \{\texttt{a}, \texttt{b}, \texttt{c}\}$, and consider the language
\begin{equation*}
L_{\text{twoequal}} = \{\texttt{a}^{i}\texttt{b}^{j}\texttt{c}^{k} \mid i, j, k \geq 0 \text{ and } i = j \text { or } j = k\}.
\end{equation*}
This language contains all words that have either the same number of \texttt{a}s and \texttt{b}s or the same number of \texttt{b}s and \texttt{c}s.

We can generate this language using the following grammar:
\begin{align*}
S		&\rightarrow S_{1} \mid S_{2} \\
S_{1}	&\rightarrow S_{1}\texttt{c} \mid A \\
A		&\rightarrow \texttt{a}A\texttt{b} \mid \epsilon \\
S_{2}	&\rightarrow \texttt{a}S_{2} \mid B \\
B		&\rightarrow \texttt{b}B\texttt{c} \mid \epsilon
\end{align*}
The rules $S_{1}$ and $A$ generate words of the form $\texttt{a}^{n}\texttt{b}^{n}\texttt{c}^{m}$ and the rules $S_{2}$ and $B$ generate words of the form $\texttt{a}^{m}\texttt{b}^{n}\texttt{c}^{n}$, each where $m, n \geq 0$.

Now, consider words of the form $\texttt{a}^{n}\texttt{b}^{n}\texttt{c}^{n}$, where $n \geq 0$. All words of this form belong to the language $L_{\text{twoequal}}$, but each such word has two distinct derivations in this grammar: it can be generated either by the rules $S_{1}$ and $A$, or by the rules $S_{2}$ and $B$.

While the formal proof that this language is inherently ambiguous is quite long, we can intuitively see that (for instance) any grammar generating this language must have rules similar to $S_{1}$ and $A$ to produce balanced pairs of \texttt{a}s and \texttt{b}s followed by some number of \texttt{c}s. We can make a similar argument for the rules $S_{2}$ and $B$. Thus, any grammar for this language will include some degree of ambiguity.
\end{example}

\subsection{Normal Forms}

Up to now, we've imposed no restrictions on the form of each rule in our context-free grammars. As long as each rule of our context-free grammar looked like $A \rightarrow \alpha$, where $A$ is a nonterminal symbol and $\alpha$ is a combination of terminal and nonterminal symbols, we were happy.

However, computers (and, by extension, the people who program computers) like having structure. For instance, a compiler for a programming language usually incorporates a context-free grammar into its workflow at some point during the compilation of a program, and having a structured grammar makes the compiler's job both easier and faster.

Therefore, at times, we might like to transform a context-free grammar into a more-structured \emph{normal form}; that is, to modify the grammar in such a way that each rule takes a canonical form. There are a number of normal forms to choose from, and each one comes with its own benefits.

\subsubsection{Chomsky Normal Form}

The \emph{Chomsky normal form}, as the name suggests, was first studied by the linguist Noam Chomsky~\citeyearpar{Chomsky1959FormalPropertiesGrammars} as he attempted to develop a model for natural language using grammars. A grammar in Chomsky normal form is one where each rule either has two nonterminal symbols or one terminal symbol on the right-hand side.

\begin{definition}[Chomsky normal form]
A context-free grammar is in Chomsky normal form if every rule in the grammar is of one of the two following forms:
\begin{enumerate}
\item $A \rightarrow BC$ for $A, B, C \in V$ with $B, C \neq S$; or
\item $A \rightarrow a$ for $A \in V$ and $a \in \Sigma$.
\end{enumerate}
Additionally, we may allow the rule $S \rightarrow \epsilon$.
\end{definition}

The main benefit of converting a grammar into Chomsky normal form comes in how we can represent and store derivations of words in memory. Since each rule derives either two nonterminal symbols or one terminal symbol, every parse tree will have a branching factor of either 2 or 1. This fact allows us to use efficient data structures for representing binary trees in memory, as well as to apply efficient algorithms to process parse trees and derivations. Moreover, the number of steps in a derivation using a grammar in Chomsky normal form is easy to bound: if the grammar generates a word $w$, then the derivation of $w$ will contain $|w| - 1$ applications of a rule of the first form and $|w|$ applications of a rule of the second form.

\begin{example}
Let $\Sigma = \{\texttt{a}, \texttt{b}\}$, and consider the following two grammars. Each grammar generates words consisting of one \texttt{b} surrounded on either side by zero or more \texttt{a}s. The grammar on the left is not in Chomsky normal form. The grammar on the right is in Chomsky normal form, and it is equivalent to the grammar on the left.
\begin{equation*}
\begin{aligned}[t]
S	&\rightarrow A\texttt{b}A \\
A	&\rightarrow A\texttt{a} \mid \epsilon
\end{aligned}
\hspace{2cm}
\begin{aligned}[t]
S_{0}	&\rightarrow TA \mid BA \mid AB \mid \texttt{b} \\
T		&\rightarrow AB \\
A		&\rightarrow AC \mid \texttt{a} \\
B		&\rightarrow \texttt{b} \\
C		&\rightarrow \texttt{a}
\end{aligned}
\end{equation*}
\end{example}

Every context-free grammar can be converted into a context-free grammar in Chomsky normal form, and the conversion process consists of five steps:
\begin{colouredbox}
\begin{enumerate}
\item \textbf{START}: Replace the start nonterminal.

Add a new start nonterminal $S_{0}$ together with a new rule $S_{0} \rightarrow S$, where $S$ is the start nonterminal of the original grammar.

This ensures that the new start nonterminal $S_{0}$ will not occur on the right-hand side of any rule.

\item \textbf{TERM}: Remove nonsolitary terminals from the right-hand side of all rules.

For each rule of the form $A \rightarrow \alpha_{1} \dots a \dots \alpha_{n}$, where $A \in V$, $\alpha_{1}, \dots, \alpha_{n} \in V \cup \Sigma$ and $a \in \Sigma$, add a new rule $T_{a} \rightarrow a$ and replace the existing rule with one of the form $A \rightarrow B_{1} \dots T_{a} \dots B_{n}$. If multiple terminals appear on the right-hand side of the rule, replace all terminals simultaneously.

This ensures that the right-hand sides of all rules consist either of a single terminal or some number of nonterminals.

\item \textbf{BIN}: Split up groups of three or more nonterminals on the right-hand side of all rules.

For each rule of the form $A \rightarrow B_{1}B_{2} \dots B_{n}$, where $A, B_{1}, \dots, B_{n} \in V$ and $n \geq 3$, replace the existing rule with the set of rules
\begin{align*}
A		&\rightarrow B_{1}A_{1}, \\
A_{1}	&\rightarrow B_{2}A_{2}, \\
		&\vdots \\
A_{n-2}	&\rightarrow B_{n-1}B_{n}.
\end{align*}

This ensures that every parse tree produced by the resultant grammar will have a bounded branching factor.

\item \textbf{DEL}: Remove epsilon rules.

Remove all rules of the form $A \rightarrow \epsilon$, where $A \neq S_{0}$. For all rules of the form $A \rightarrow BC$ where $A, B, C \in V$ and either $B$ or $C$ is \emph{nullable} (i.e., where there exists a rule $B \rightarrow \epsilon$ or $C \rightarrow \epsilon$), replace the existing rule with one where the nullable nonterminal is removed.

\item \textbf{UNIT}: Remove unit rules.

For each pair of rules of the form $A \rightarrow B$ and $B \rightarrow C$, where $A, B \in V$ and $C \in V^{+}$, replace the existing rule $A \rightarrow B$ with one of the form $A \rightarrow C$, unless this replacement produces a unit rule that has previously been removed.
\end{enumerate}
\end{colouredbox}

The process of converting a context-free grammar to Chomsky normal form can be lengthy and tedious, so the job is often automated by a subroutine within a compiler or a similar program. However, we're here to learn, and learning is best done by doing. Thus, let's work through an example of this conversion process step by step.

\begin{example}
Consider the following grammar not in Chomsky normal form:
\begin{align*}
S	&\rightarrow ASB \\
A	&\rightarrow \texttt{a}AS \mid \texttt{a} \mid \epsilon \\
B	&\rightarrow S\texttt{b}S \mid A \mid \texttt{bb}
\end{align*}
We will convert this grammar to an equivalent grammar in Chomsky normal form.

\begin{enumerate}
\item \textbf{START}: Replace the start nonterminal.

We begin by adding a new start nonterminal and the rule $S_{0} \rightarrow S$, which gives us the following:
\begin{align*}
\highlightmath{S_{0}}	&\rightarrow \highlightmath{S} \\
S				&\rightarrow ASB \\
A				&\rightarrow \texttt{a}AS \mid \texttt{a} \mid \epsilon \\
B				&\rightarrow S\texttt{b}S \mid A \mid \texttt{bb}
\end{align*}

\item \textbf{TERM}: Remove nonsolitary terminals from the right-hand side of all rules.

We have three rules to handle here: $A \rightarrow \texttt{a}AS$, $B \rightarrow S\texttt{b}S$, and $B \rightarrow \texttt{bb}$. Adding the new rules $T_{\texttt{a}} \rightarrow \texttt{a}$ and $T_{\texttt{b}} \rightarrow \texttt{b}$ and making the appropriate substitutions gives us the following:
\begin{align*}
S_{0}					&\rightarrow S \\
S						&\rightarrow ASB \\
\highlightmath{A}			&\rightarrow \highlightmath{T_{\texttt{a}}AS} \mid \texttt{a} \mid \epsilon \\
\highlightmath{B}			&\rightarrow \highlightmath{ST_{\texttt{b}}S} \mid A \mid \highlightmath{T_{\texttt{b}}T_{\texttt{b}}} \\
\highlightmath{T_{\texttt{a}}}	&\rightarrow \highlightmath{\texttt{a}} \\
\highlightmath{T_{\texttt{b}}}	&\rightarrow \highlightmath{\texttt{b}}
\end{align*}

\item \textbf{BIN}: Split up groups of three or more nonterminals on the right-hand side of all rules.

Again, we have three rules to split up: $S \rightarrow ASB$, $A \rightarrow T_{\texttt{a}}AS$, and $B \rightarrow ST_{\texttt{b}}S$. Splitting these rules gives us the following:
\begin{equation*}
\begin{aligned}[t]
S_{0}					&\rightarrow S \\
\highlightmath{S}			&\rightarrow \highlightmath{AS_{1}} \\
\highlightmath{A}			&\rightarrow \highlightmath{T_{\texttt{a}}A_{1}} \mid \texttt{a} \mid \epsilon \\
\highlightmath{B}			&\rightarrow \highlightmath{SB_{1}} \mid A \mid T_{\texttt{b}}T_{\texttt{b}} \\
T_{\texttt{a}}				&\rightarrow \texttt{a} \\
T_{\texttt{b}}				&\rightarrow \texttt{b}
\end{aligned}
\hspace{0.75cm}
\begin{aligned}[t]
\phantom{S_{0}} & \\
\highlightmath{S_{1}}			&\rightarrow \highlightmath{SB} \\
\highlightmath{A_{1}}			&\rightarrow \highlightmath{AS} \\
\highlightmath{B_{1}}			&\rightarrow \highlightmath{T_{\texttt{b}}S}
\end{aligned}
\end{equation*}

\item \textbf{DEL}: Remove epsilon rules.

This grammar has one obvious epsilon rule, which is $A \rightarrow \epsilon$. However, we must also modify the rules that contain the nullable nonterminal $A$: these rules are $S \rightarrow AS_{1}$, $A_{1} \rightarrow AS$, and $B \rightarrow A$. For each of these rules, we add a new rule that is of the same form, but with the nonterminal $A$ removed.
\begin{equation*}
\begin{aligned}[t]
S_{0}					&\rightarrow S \\
\highlightmath{S}			&\rightarrow AS_{1} \mid \highlightmath{S_{1}} \\
A						&\rightarrow T_{\texttt{a}}A_{1} \mid \texttt{a} \mid {\color{\neutralcolour}\cancel{\epsilon}} \phantom{\highlightmath{S_{0}}} \\
\highlightmath{B}			&\rightarrow SB_{1} \mid A \mid \highlightmath{\epsilon} \mid T_{\texttt{b}}T_{\texttt{b}} \\
T_{\texttt{a}}				&\rightarrow \texttt{a} \\
T_{\texttt{b}}				&\rightarrow \texttt{b}
\end{aligned}
\hspace{0.75cm}
\begin{aligned}[t]
\phantom{S_{0}} & \\
S_{1}					&\rightarrow SB \phantom{\highlightmath{S_{0}}} \\
\highlightmath{A_{1}}			&\rightarrow AS \mid \highlightmath{S} \\
B_{1}					&\rightarrow T_{\texttt{b}}S \phantom{\highlightmath{S_{0}}}
\end{aligned}
\end{equation*}

Observe that removing the nullable nonterminal $A$ from the rule $B \rightarrow A$ produced another epsilon rule, $B \rightarrow \epsilon$, so we must remove that rule as well. This also means that $B$ is a nullable nonterminal, and so we must modify rules containing $B$: the only rule affected here is $S_{1} \rightarrow SB$. For this rule, we again add a new rule that is of the same form, but with the nonterminal $B$ removed.
\begin{equation*}
\begin{aligned}[t]
S_{0}					&\rightarrow S \\
S						&\rightarrow AS_{1} \mid S_{1} \phantom{\highlightmath{S_{0}}} \\
A						&\rightarrow T_{\texttt{a}}A_{1} \mid \texttt{a} \\
B						&\rightarrow SB_{1} \mid A \mid {\color{\neutralcolour}\cancel{\epsilon}} \mid T_{\texttt{b}}T_{\texttt{b}} \\
T_{\texttt{a}}				&\rightarrow \texttt{a} \\
T_{\texttt{b}}				&\rightarrow \texttt{b}
\end{aligned}
\hspace{0.75cm}
\begin{aligned}[t]
\phantom{S_{0}} & \\
\highlightmath{S_{1}}			&\rightarrow SB \mid \highlightmath{S} \\
A_{1}					&\rightarrow AS \mid S \\
B_{1}					&\rightarrow T_{\texttt{b}}S
\end{aligned}
\end{equation*}

\item \textbf{UNIT}: Remove unit rules.

Lastly, we handle all of the unit rules in this grammar. We'll begin by removing the unit rule $S_{0} \rightarrow S$ to obtain the following:
\begin{equation*}
\begin{aligned}[t]
\highlightmath{S_{0}}			&\rightarrow \highlightmath{AS_{1}} \mid \highlightmath{S_{1}} \\
S						&\rightarrow AS_{1} \mid S_{1} \\
A						&\rightarrow T_{\texttt{a}}A_{1} \mid \texttt{a} \\
B						&\rightarrow SB_{1} \mid A \mid T_{\texttt{b}}T_{\texttt{b}} \\
T_{\texttt{a}}				&\rightarrow \texttt{a} \\
T_{\texttt{b}}				&\rightarrow \texttt{b}
\end{aligned}
\hspace{0.75cm}
\begin{aligned}[t]
\phantom{S_{0}} & \phantom{\highlightmath{S_{0}}} \\
S_{1}					&\rightarrow SB \mid S \\
A_{1}					&\rightarrow AS \mid S \\
B_{1}					&\rightarrow T_{\texttt{b}}S
\end{aligned}
\end{equation*}

In removing this rule, we added a new unit rule $S_{0} \rightarrow S_{1}$, so let's take care of that rule next. Note that straightforwardly performing the substitution on the right-hand side would again produce the unit rule $S_{0} \rightarrow S$ that we already removed, so we omit that rule and obtain the following:
\begin{equation*}
\begin{aligned}[t]
\highlightmath{S_{0}}			&\rightarrow AS_{1} \mid \highlightmath{SB} \\
S						&\rightarrow AS_{1} \mid S_{1} \\
A						&\rightarrow T_{\texttt{a}}A_{1} \mid \texttt{a} \\
B						&\rightarrow SB_{1} \mid A \mid T_{\texttt{b}}T_{\texttt{b}} \\
T_{\texttt{a}}				&\rightarrow \texttt{a} \\
T_{\texttt{b}}				&\rightarrow \texttt{b}
\end{aligned}
\hspace{0.75cm}
\begin{aligned}[t]
\phantom{S_{0}} & \phantom{\highlightmath{S_{0}}} \\
S_{1}					&\rightarrow SB \mid S \\
A_{1}					&\rightarrow AS \mid S \\
B_{1}					&\rightarrow T_{\texttt{b}}S
\end{aligned}
\end{equation*}

We now remove the unit rule $S \rightarrow S_{1}$. Performing the substitution on the right-hand side would produce the (useless) unit rule $S \rightarrow S$, so we omit that rule and obtain the following:
\begin{equation*}
\begin{aligned}[t]
S_{0}					&\rightarrow AS_{1} \mid SB \\
\highlightmath{S}			&\rightarrow AS_{1} \mid \highlightmath{SB} \\
A						&\rightarrow T_{\texttt{a}}A_{1} \mid \texttt{a} \\
B						&\rightarrow SB_{1} \mid A \mid T_{\texttt{b}}T_{\texttt{b}} \\
T_{\texttt{a}}				&\rightarrow \texttt{a} \\
T_{\texttt{b}}				&\rightarrow \texttt{b}
\end{aligned}
\hspace{0.75cm}
\begin{aligned}[t]
\phantom{S_{0}} & \\
S_{1}					&\rightarrow SB \mid S \phantom{\highlightmath{S_{0}}} \\
A_{1}					&\rightarrow AS \mid S \\
B_{1}					&\rightarrow T_{\texttt{b}}S
\end{aligned}
\end{equation*}

We can now remove the unit rules $S_{1} \rightarrow S$ and $A_{1} \rightarrow S$, which both have the nonterminal $S$ on the right-hand side. This produces the following:
\begin{equation*}
\begin{aligned}[t]
S_{0}					&\rightarrow AS_{1} \mid SB \\
S						&\rightarrow AS_{1} \mid SB \phantom{\highlightmath{S_{0}}} \\
A						&\rightarrow T_{\texttt{a}}A_{1} \mid \texttt{a} \phantom{\highlightmath{S_{0}}} \\
B						&\rightarrow SB_{1} \mid A \mid T_{\texttt{b}}T_{\texttt{b}} \\
T_{\texttt{a}}				&\rightarrow \texttt{a} \\
T_{\texttt{b}}				&\rightarrow \texttt{b}
\end{aligned}
\hspace{0.75cm}
\begin{aligned}[t]
\phantom{S_{0}} & \\
\highlightmath{S_{1}}			&\rightarrow SB \mid \highlightmath{AS_{1}} \\
\highlightmath{A_{1}}			&\rightarrow AS \mid \highlightmath{AS_{1}} \mid \highlightmath{SB} \\
B_{1}					&\rightarrow T_{\texttt{b}}S
\end{aligned}
\end{equation*}

Lastly, we remove the unit rule $B \rightarrow A$, which gives us our Chomsky normal form grammar:
\begin{equation*}
\begin{aligned}[t]
S_{0}					&\rightarrow AS_{1} \mid SB \\
S						&\rightarrow AS_{1} \mid SB\\
A						&\rightarrow T_{\texttt{a}}A_{1} \mid \texttt{a} \\
\highlightmath{B}			&\rightarrow SB_{1} \mid \highlightmath{T_{\texttt{a}}A_{1}} \mid \highlightmath{\texttt{a}} \mid T_{\texttt{b}}T_{\texttt{b}} \\
T_{\texttt{a}}				&\rightarrow \texttt{a} \\
T_{\texttt{b}}				&\rightarrow \texttt{b}
\end{aligned}
\hspace{0.75cm}
\begin{aligned}[t]
\phantom{S_{0}} & \\
S_{1}					&\rightarrow SB \mid AS_{1} \\
A_{1}					&\rightarrow AS \mid AS_{1} \mid SB \\
B_{1}					&\rightarrow T_{\texttt{b}}S \phantom{\highlightmath{S_{0}}}
\end{aligned}
\end{equation*}
\end{enumerate}
\end{example}

\futuresubsubsection{Greibach Normal Form}

\phantom{.} \par

\begin{construction}
Together with the section on Chomsky normal form, I would also like to write a discussion of Greibach normal form~\citeyearpar{Greibach1965NewNormalFormTheorem}, its benefits, and how we can convert a context-free grammar into GNF.
\end{construction}

%\subsubsection{Backus--Naur Form}
\section{Pushdown Automata}\label{sec:pushdownautomata}

\firstwords{When we first introduced} finite automata as a computational model for regular languages, we emphasized the facts that finite automata have no method of storage and no ability to return to a previously read symbol. Naturally, these restrictions limited the kinds of languages the model is able to recognize, and we showed that such restrictions resulted in the model recognizing exactly the class of regular languages.

At the end of the previous lecture, we saw that there exist languages that are not regular, and therefore are not recognized by finite automata. We know now that the next ``step" of our language hierarchy is the class of context-free languages. Thus, a new question arises: what kind of computational model is capable of recognizing context-free languages?

Since every context-free language is generated by a context-free grammar, and since we know that context-free grammars must ``remember" which nonterminal and terminal symbols are being manipulated over the course of a derivation, any model of computation recognizing context-free languages must include a form of memory. What is the best form of memory to use in this situation? If we view a derivation as a parse tree, then the derivation progresses as we go deeper into the parse tree, and we can easily model the depth of a derivation using \emph{stack} memory.

As a brief review, a stack is a data structure with two operations that manipulate data: \emph{push} and \emph{pop}. Pushing a symbol to a stack adds it to the top of the stack, above all other symbols already in the stack. Conversely, popping a symbol from a stack removes it from the top of the stack, leaving all other symbols untouched. (See Figure~\ref{fig:stacks} for an example of pushing and popping.) As a result, a stack provides last-in-first-out, or LIFO, storage---by comparison, a data structure like a queue provides first-in-first-out, or FIFO, storage. We can view the symbol at the top of the stack at any time during a computation, but we cannot view any other symbols in the stack unless we pop the symbol currently at the top of the stack.

\begin{figure}[b]
\centering
\begin{tikzpicture}[stack/.style={rectangle split, rectangle split parts=#1,draw, anchor=center}]
\node[stack=4] (S) {
};
\draw[thick, color=white] (S.north west) -- (S.north east);
\end{tikzpicture}
\hspace{0cm}
\begin{tikzpicture}[stack/.style={rectangle split, rectangle split parts=#1,draw=none, anchor=center}]
\node[stack=3]  {
\nodepart{two}$\xRightarrow{\text{push } \texttt{a}}$
};
\end{tikzpicture}
\hspace{0cm}
\begin{tikzpicture}[stack/.style={rectangle split, rectangle split parts=#1, rectangle split part fill={white,white,white,\fourthcolour}, draw, anchor=center}]
\node[stack=4] (S) {
\nodepart{four}{\color{\maincolour}\texttt{a}}
};
\draw[thick, color=white] (S.north west) -- (S.north east);
\end{tikzpicture}
\hspace{0cm}
\begin{tikzpicture}[stack/.style={rectangle split, rectangle split parts=#1,draw=none, anchor=center}]
\node[stack=3]  {
\nodepart{two}$\xRightarrow{\text{push } \texttt{b}}$
};
\end{tikzpicture}
\hspace{0cm}
\begin{tikzpicture}[stack/.style={rectangle split, rectangle split parts=#1, rectangle split part fill={white,white,\fourthcolour,white}, draw, anchor=center}]
\node[stack=4] (S) {
\nodepart{three}{\color{\maincolour}\texttt{b}}
\nodepart{four}\texttt{a}
};
\draw[thick, color=white] (S.north west) -- (S.north east);
\end{tikzpicture}
\hspace{0cm}
\begin{tikzpicture}[stack/.style={rectangle split, rectangle split parts=#1,draw=none, anchor=center}]
\node[stack=3]  {
\nodepart{two}$\xRightarrow{\text{pop } \texttt{b}}$
};
\end{tikzpicture}
\hspace{0cm}
\begin{tikzpicture}[stack/.style={rectangle split, rectangle split parts=#1, rectangle split part fill={white,white,white,\fourthcolour}, draw, anchor=center}]
\node[stack=4] (S) {
\nodepart{four}{\color{\maincolour}\texttt{a}}
};
\draw[thick, color=white] (S.north west) -- (S.north east);
\end{tikzpicture}
\hspace{0cm}
\begin{tikzpicture}[stack/.style={rectangle split, rectangle split parts=#1,draw=none, anchor=center}]
\node[stack=3]  {
\nodepart{two}$\xRightarrow{\text{push } \texttt{c}}$
};
\end{tikzpicture}
\hspace{0cm}
\begin{tikzpicture}[stack/.style={rectangle split, rectangle split parts=#1, rectangle split part fill={white,white,\fourthcolour,white}, draw, anchor=center}]
\node[stack=4] (S) {
\nodepart{three}{\color{\maincolour}\texttt{c}}
\nodepart{four}\texttt{a}
};
\draw[thick, color=white] (S.north west) -- (S.north east);
\end{tikzpicture}
\hspace{0cm}
\begin{tikzpicture}[stack/.style={rectangle split, rectangle split parts=#1,draw=none, anchor=center}]
\node[stack=3]  {
\nodepart{two}$\xRightarrow{\text{push } \texttt{a}}$
};
\end{tikzpicture}
\hspace{0cm}
\begin{tikzpicture}[stack/.style={rectangle split, rectangle split parts=#1, rectangle split part fill={white,\fourthcolour,white,white}, draw, anchor=center}]
\node[stack=4] (S) {
\nodepart{two}{\color{\maincolour}\texttt{a}}
\nodepart{three}\texttt{c}
\nodepart{four}\texttt{a}
};
\draw[thick, color=white] (S.north west) -- (S.north east);
\end{tikzpicture}
\caption{Pushing symbols to and popping symbols from a stack. At each step, the currently visible symbol at the top of the stack is highlighted}
\label{fig:stacks}
\end{figure}

Since we're dealing with an abstract model of computation and not a real-world computer, we can make the assumption that our stack size is \emph{unbounded}; that is, we can push as many symbols to the stack as we want without worrying about running out of space.

Now that we have our form of storage established, we can define our model of computation. At its core, this model is a finite automaton with a stack added to it. Since the automaton is now able to push symbols to a stack, we give it an appropriate name: a \emph{pushdown automaton}.

\begin{remark}
The name ``pushdown automaton" doesn't specifically come from its ability to push symbols, but rather from an older term for a stack: a \emph{pushdown store}.
\end{remark}

In addition to reading a symbol of its input word on a transition, a pushdown automaton can read from and write to the stack on the same transition. In order to perform this mixture of input and stack actions, we specify two alphabets for a pushdown automaton: the \emph{input alphabet}, which contains symbols used in the input word, and the \emph{stack alphabet}, which contains symbols the pushdown automaton can use in its stack. This allows us to combine actions on the input word and actions on the stack in a single transition, without risking confusion over the meaning of any particular alphabet symbol. The transitions of a pushdown automaton may additionally use $\epsilon$ in place of either the input word action (i.e., when we don't read a symbol of the input word) or the stack action (i.e., when we don't push to/pop from the stack). Just like we denoted a finite automaton's alphabet by $\Sigma$, we will use $\Sigma$ to denote a pushdown automaton's input alphabet. Likewise, we will use $\Gamma$ to denote the stack alphabet.

In order for our model of computation to use two alphabets at once, we must modify its transition function accordingly. Recall that a finite automaton (with epsilon transitions) transitions on a pair $(q, a)$, where $q \in Q$ and $a \in \Sigma \cup \{\epsilon\}$. By comparison, a pushdown automaton transitions on a tuple $(q, a, b)$, where $q \in Q$, $a \in \Sigma \cup \{\epsilon\}$, and $b \in \Gamma \cup \{\epsilon\}$. Thus, a pushdown automaton uses both the current symbol of its input word (or $\epsilon$) as well as the top symbol of its stack (or $\epsilon$) to determine the state to which it will transition. After transitioning, the pushdown automaton will be in a possibly different state, and it will have a possibly different symbol at the top of its stack.

Lastly, a pushdown automaton has no inherent mechanism for detecting whether its stack is empty. To make our lives easier when it comes to keeping track of the stack contents, we can incorporate a special ``bottom of stack" symbol $\bot$ into the transitions of a pushdown automaton in such a way that $\bot$ is both the first symbol pushed to the stack and the last symbol popped from the stack.

\begin{remark}
We don't require this special symbol, since pushdown automata can accept either by being in a final state or by having an empty stack after reaching the end of its input word. As it turns out, these two methods of acceptance are equivalent, and our approach effectively combines the two.
\end{remark}

Having established all of the technical details, we can now formulate the definition of a pushdown automaton. Since a pushdown automaton is essentially just a finite automaton with a stack, we can start by copying the text from Definition~\ref{def:NFA} and adding to it the necessary components for handling the stack: the stack alphabet, a mechanism for popping symbols (if necessary) from the stack, and a mechanism for pushing symbols (if necessary) to the stack.

\begin{definition}[Pushdown automaton]
A pushdown automaton is a tuple $(Q, \Sigma, \Gamma, \delta, q_{0}, F)$, where
\begin{itemize}
\item $Q$ is a finite set of \emph{states};
\item $\Sigma$ is the \emph{input alphabet};
\item $\Gamma$ is the \emph{stack alphabet};
\item $\delta \from Q \times (\Sigma \cup \{\epsilon\}) \times (\Gamma \cup \{\epsilon\}) \to \mathcal{P}\left(Q \times (\Gamma \cup \{\epsilon\})\right)$ is the \emph{transition function};
\item $q_{0} \in Q$ is the \emph{initial} or \emph{start state}; and
\item $F \subseteq Q$ is the set of \emph{final} or \emph{accepting states}.
\end{itemize}
\end{definition}

\begin{figure}[t]
\centering
\begin{tikzpicture}[stack/.style={rectangle split, rectangle split parts=#1, rectangle split part fill={white,\fourthcolour,white,white}, draw, anchor=center}]
\node[draw=none, color=black, align=center, font={\small}] at (1,2.5) {Finite \\ automaton};
\draw[draw=black, fill=\fourthcolour] (0,0) rectangle (2,2);
\draw[draw=\secondcolour, fill=\secondcolour] (0.5,1.5) circle (0.2);
\draw[draw=\secondcolour, fill=\secondcolour] (1.4,1) circle (0.2);
\draw[draw=\secondcolour, fill=\secondcolour] (0.6,0.5) circle (0.2);
\draw[-Latex, thick, draw=\secondcolour] (0.52,1.3) -- (0.575,0.7);
\draw[-Latex, thick, draw=\secondcolour] (0.675,1.4) -- (1.225,1.1);
\draw [-Latex, thick, draw=\secondcolour, looseness=4] (1.275,0.85) .. controls (0.9,0.3) and (1.8,0.3) .. (1.525,0.875);

\node[draw=none, color=black, align=center, font={\small}] at (3.5,2.275) {Input word};
\node[draw=none, color=black, font={\large}] at (3.5,1) {\texttt{0101101110}};
\draw[-Latex, thick, draw=\neutralcolour] (4.55,0.7) -- (2.6,0.7);
\node[trapezium, trapezium left angle=70, trapezium right angle=70, draw=black, fill=\fourthcolour, rotate=90, minimum width=0.75cm, minimum height=0.5cm] (t) at (2.25,1) {};

\node[draw=none, color=black, align=center, font={\small}] at (-1.6,2.275) {Stack};
\draw[draw=black, fill=\fourthcolour] (-0.2,0.8) rectangle (0,1.2);
\draw[-Latex, thick, draw=\secondcolour] (-0.2,1) -- (-0.75,1) -- (-0.75,1.275) -- (-1.375,1.275);

\node[stack=4] (S) at (-1.6,1) {
\nodepart{two}{\color{\maincolour}\texttt{a}}
\nodepart{three}\texttt{b}
\nodepart{four}$\bot$
};
\draw[thick, color=white] (S.north west) -- (S.north east);
\end{tikzpicture}
\caption{An illustration of a pushdown automaton}
\label{fig:pushdownautomaton}
\end{figure}

\noindent
Figure~\ref{fig:pushdownautomaton} illustrates how we might visualize a pushdown automaton. Observe that our definition doesn't mention the ``bottom of stack" symbol $\bot$, since it isn't strictly necessary. However, since we will use it to aid in our understanding, we will assume that $\bot \in \Gamma$ and that $\bot$ is not used in any transition of $\delta$ except for those exiting $q_{0}$ and those entering any final state.

\begin{remark}
Here is another moment for grammatical pedantry. Just like the distinction between the singular ``finite automat\underline{on}" and the plural ``finite automat\underline{a}", it is never correct to write something like ``\underline{a} pushdown automat\underline{a}" in reference to a single instance of such a model.
\end{remark}

\begin{example}
Consider a pushdown automaton $\mathcal{M}$ where $Q = \{q_{0}, q_{1}, q_{2}, q_{3}\}$, $\Sigma = \{\texttt{a}, \texttt{b}\}$, $\Gamma = \{\bot, \texttt{A}\}$, $q_{0} = q_{0}$, $F = \{q_{3}\}$, and $\delta$ is defined as follows:
\begin{center}
\small
\begin{tabular}{c | c c c | c c c | c c c}
$\Sigma$:		& \multicolumn{3}{c}{\texttt{a}}	& \multicolumn{3}{c}{\texttt{b}}	& \multicolumn{3}{c}{$\epsilon$} \\
\hline
$\Gamma$:	& $\bot$ 		& \texttt{A}	& $\epsilon$			& $\bot$ 		& \texttt{A}			& $\epsilon$	& $\bot$ 				& \texttt{A}	& $\epsilon$ \\
\hline
$q_{0}$		& ---			& ---			& ---					& ---			& ---					& ---			& ---					& ---			& $\{(q_{1}, \bot)\}$ \\
$q_{1}$		& ---			& ---			& $\{(q_{1}, \texttt{A})\}$	& ---			& $\{(q_{2}, \epsilon)\}$	& ---			& ---					& ---			& --- \\
$q_{2}$		& ---			& ---			& ---					& ---			& $\{(q_{2}, \epsilon)\}$	& ---			& $\{(q_{3}, \epsilon)\}$	& ---			& --- \\
$q_{3}$		& ---			& ---			& ---					& ---			& ---					& ---			& ---					& ---			& --- 
\end{tabular}
\end{center}
In the transition function table, the top row indicates the input symbol being read and the second-from-top row indicates the symbol to be popped from the stack. Each entry of the table is an ordered pair where the first element is the state being transitioned to and the second element is the symbol being pushed to the stack.

The pushdown automaton $\mathcal{M}$ can be represented visually as follows:
\begin{center}
\begin{tikzpicture}[node distance=2.5cm, >=latex, every state/.style={fill=white}]
\node[state, initial] (q0) {$q_{0}$};
\node[state] (q1) [right of=q0] {$q_{1}$};
\node[state] (q2) [right of=q1] {$q_{2}$};
\node[state, accepting] (q3) [right of=q2] {$q_{3}$};

\path[-latex] (q0) edge [above] node {$\epsilon, \epsilon \mapsto \bot$} (q1);
\path[-latex] (q1) edge [loop above] node {$\texttt{a}, \epsilon \mapsto \texttt{A}$} (q1);
\path[-latex] (q1) edge [above] node {$\texttt{b}, \texttt{A} \mapsto \epsilon$} (q2);
\path[-latex] (q2) edge [loop above] node {$\texttt{b}, \texttt{A} \mapsto \epsilon$} (q2);
\path[-latex] (q2) edge [above] node {$\epsilon, \bot \mapsto \epsilon$} (q3);
\end{tikzpicture}
\end{center}
Notice that each transition has a label of the form $a, B \mapsto C$; this means that, upon reading an input symbol $a$ and popping a symbol $B$ from the stack, the pushdown automaton pushes a symbol $C$ to the stack.

Between states $q_{0}$ and $q_{1}$, the pushdown automaton pushes the symbol $\bot$ to the stack to mark the bottom. In state $q_{1}$, the pushdown automaton reads some number of \texttt{a}s and pushes the same number of \texttt{A}s to the stack. Between states $q_{1}$ and $q_{2}$, as well as in state $q_{2}$, the pushdown automaton reads some number of \texttt{b}s and pops the same number of \texttt{A}s from the stack. Finally, between states $q_{2}$ and $q_{3}$, the pushdown automaton pops $\bot$ from the stack only if there are no more input symbols to read and no more stack symbols to process.

After some observation, we can see that our pushdown automaton accepts all input words of the form $\texttt{a}^{n}\texttt{b}^{n}$ where $n \geq 1$.
\end{example}

You may have noticed in our definition that the transition function maps to the power set of state/stack symbol pairs, which makes the pushdown automaton nondeterministic. This was not done by mistake. Unlike finite automata, where the deterministic and nondeterministic models are equivalent in terms of recognition power, deterministic pushdown automata actually recognize \emph{fewer} languages than nondeterministic pushdown automata. In the interest of full generality, then, we take all of our pushdown automata to be nondeterministic, even if we don't need to use nondeterminism.

\subsection{Computations and Accepting Computations}

Let us now consider precisely what it means for a pushdown automaton to accept an input word. As we had with finite automata, one of the main conditions for acceptance is that there exists some sequence of states through the automaton where it begins reading its input word in an initial state and finishes reading in an accepting state. Since pushdown automata also come with a stack, though, we must account for the contents of the stack over the course of the computation. Specifically, we assume that the stack is empty at the beginning of the computation and, on each transition, the pushdown automaton can modify the top symbol of its stack appropriately.

\begin{definition}[Accepting computation of a pushdown automaton]
Let $\mathcal{M} = (Q, \Sigma, \Gamma, \delta, q_{0}, F)$ be a pushdown automaton, and let $w = w_{0} w_{1} \dots w_{n-1}$ be an input word of length $n$ where $w_{0}, w_{1}, \dots, w_{n - 1} \in \Sigma$. The pushdown automaton $\mathcal{M}$ accepts the input word $w$ if there exists a sequence of states $r_{0}, r_{1}, \dots, r_{n} \in Q$ and a sequence of stack contents $s_{0}, s_{1}, \dots, s_{n} \in \Gamma^{*}$ satisfying the following conditions:
\begin{enumerate}
\item $r_{0} = q_{0}$ and $s_{0} = \epsilon$;
\item $(r_{i+1}, b') \in \delta(r_{i}, w_{i}, b)$ for all $0 \leq i \leq (n-1)$, where $s_{i} = bt$ and $s_{i+1} = b't$ for some $b, b' \in \Gamma \cup \{\epsilon\}$ and $t \in \Gamma^{*}$; and
\item $r_{n} \in F$.
\end{enumerate}
\end{definition}

The second condition is rather notation-heavy, but the underlying idea describes exactly how a pushdown automaton transitions between states: starting in a state $r_{i}$ with a symbol $b$ at the top of the stack, the pushdown automaton reads an input symbol $w_{i}$ and pops the symbol $b$ from the stack. The transition function then sends the pushdown automaton to a state $r_{i+1}$ and pushes the symbol $b'$ to the stack.

Indeed, the second condition corresponds exactly to having the following transition in the pushdown automaton:
\begin{center}
\begin{tikzpicture}[node distance=3cm, >=latex, every state/.style={fill=white}]
\node[draw=none] (r) {$\dots$};
\node[state, minimum size=1cm] (r0) [right=1cm of r] {$r_{i}$};
\node[state, minimum size=1cm] (r1) [right of=r0] {$r_{i+1}$};
\node[draw=none] (rr) [right=1cm of r1] {$\dots$};

\path[-latex] (r) edge [above] node {} (r0);
\path[-latex] (r0) edge [above] node {$w_{i}, b \mapsto b'$} (r1);
\path[-latex] (r1) edge [above] node {} (rr);
\end{tikzpicture}
\end{center}

\subsection{Language of a Pushdown Automaton}

Pushdown automata recognize languages just as finite automata do, and the set of all input words accepted by a pushdown automaton is referred to as the language of that automaton. We denote the class of languages recognized by a pushdown automaton by \PDA.

\begin{example}
Consider $L_{\texttt{()}}$, our language of balanced parentheses from earlier. Suppose $\Sigma = \{\texttt{(}, \texttt{)}\}$ and $\Gamma = \{\bot, \texttt{P}\}$. A pushdown automaton recognizing this language is as follows:
\begin{center}
\begin{tikzpicture}[node distance=2.5cm, >=latex, every state/.style={fill=white}]
\node[state, initial] (q0) {$q_{0}$};
\node[state] (q1) [right of=q0] {$q_{1}$};
\node[state, accepting] (q2) [right of=q1] {$q_{2}$};

\path[-latex] (q0) edge [above] node {$\epsilon, \epsilon \mapsto \bot$} (q1);
\path[-latex] (q1) edge [loop above] node {$\texttt{(}, \epsilon \mapsto \texttt{P}$} (q1);
\path[-latex] (q1) edge [loop below] node {$\texttt{)}, \texttt{P} \mapsto \epsilon$} (q1);
\path[-latex] (q1) edge [above] node {$\epsilon, \bot \mapsto \epsilon$} (q2);
\end{tikzpicture}
\end{center}
As the transitions show, after pushing the symbol $\bot$ to the stack, the pushdown automaton reads left and right parentheses. Every time a left parenthesis \texttt{(} is read, the pushdown automaton pushes a symbol \texttt{P} to the stack. Likewise, every time a right parenthesis \texttt{)} is read, the pushdown automaton pops a symbol \texttt{P} from the stack to account for some left parenthesis being matched.

Note that, if the input word contains more right parentheses than left parentheses, then the pushdown automaton will not be able to pop a symbol \texttt{P} from the stack. Similarly, if the input word contains more left parentheses than right parentheses, then it will not be able to pop the symbol $\bot$ from the stack. In either case, it becomes stuck in state $q_{1}$ and unable to accept the input word.
\end{example}

\begin{example}
Let $\Sigma = \{\texttt{a}, \texttt{b}, \texttt{c}\}$, and consider the language
\begin{equation*}
L_{\text{twoequal}} = \{\texttt{a}^{i}\texttt{b}^{j}\texttt{c}^{k} \mid i, j, k \geq 0 \text{ and } i = j \text{ or } j = k\}.
\end{equation*}
A pushdown automaton recognizing $L_{\text{twoequal}}$ must have two ``branches": one branch to handle the case where $i = j$, and one branch to handle the case where $j = k$. Since we don't know in advance which branch we will need to take, we can use the nondeterminism inherent in the pushdown automaton model.

A pushdown automaton recognizing this language would therefore look like the following, where the upper branch handles the case $i = j$ and the lower branch handles the case $j = k$:
\begin{center}
\begin{tikzpicture}[node distance=2.5cm, >=latex, every state/.style={fill=white}]
\node[state, initial] (q0) {$q_{0}$};
\node[state] (q1) [above right=0.5cm of q0] {$q_{1}$};
\node[state] (q2) [right of=q1] {$q_{2}$};
\node[state, accepting] (q3) [right of=q2] {$q_{3}$};
\node[state] (q4) [below right=0.5cm of q0] {$q_{4}$};
\node[state] (q5) [right of=q4] {$q_{5}$};
\node[state] (q6) [right of=q5] {$q_{6}$};
\node[state, accepting] (q7) [right of=q6] {$q_{7}$};

\path[-latex] (q0) edge [above left] node {$\epsilon, \epsilon \mapsto \bot$} (q1);
\path[-latex] (q0) edge [below left] node {$\epsilon, \epsilon \mapsto \bot$} (q4);

\path[-latex] (q1) edge [loop above] node {$\texttt{a}, \epsilon \mapsto \texttt{A}$} (q1);
\path[-latex] (q1) edge [above] node {$\epsilon, \epsilon \mapsto \epsilon$} (q2);
\path[-latex] (q2) edge [loop above] node {$\texttt{b}, \texttt{A} \mapsto \epsilon$} (q2);
\path[-latex] (q2) edge [above] node {$\epsilon, \bot \mapsto \epsilon$} (q3);
\path[-latex] (q3) edge [loop above] node {$\texttt{c}, \epsilon \mapsto \epsilon$} (q3);

\path[-latex] (q4) edge [loop below] node {$\texttt{a}, \epsilon \mapsto \epsilon$} (q4);
\path[-latex] (q4) edge [above] node {$\epsilon, \epsilon \mapsto \epsilon$} (q5);
\path[-latex] (q5) edge [loop below] node {$\texttt{b}, \epsilon \mapsto \texttt{B}$} (q5);
\path[-latex] (q5) edge [above] node {$\epsilon, \epsilon \mapsto \epsilon$} (q6);
\path[-latex] (q6) edge [loop below] node {$\texttt{c}, \texttt{B} \mapsto \epsilon$} (q6);
\path[-latex] (q6) edge [above] node {$\epsilon, \bot \mapsto \epsilon$} (q7);
\end{tikzpicture}
\end{center}
\end{example}

%\subsection{Constructing Pushdown Automata}

%\begin{construction}
% I would like to write a short section about how one can design a pushdown automaton for a given language/application/etc.
%\end{construction}
\section{Equivalence of Models}\label{sec:equivalenceofmodelscontextfree}

\firstwords{You may recall} from our discussion of regular languages that we proved a number of exciting results: deterministic and nondeterministic finite automata are equivalent in terms of recognition power, regardless of whether epsilon transitions are involved, and each of these models is itself equivalent in recognition power to regular expressions. These results allowed us to establish Kleene's theorem, which characterized the class of regular languages in terms of several different models of computation.

Now that we're focusing on context-free languages, and now that we have two ways of representing context-free languages---namely, context-free grammars and pushdown automata---it would be nice to establish a connection between the two representations. This brings us to yet another exciting result, which will be the focus of this section. Since the overall proof is quite lengthy, we will split the proof of the main result into two parts.

\subsection{$\CFG \Rightarrow \PDA$}

For the first half of our main result, we will show that we can convert any context-free grammar into a pushdown automaton recognizing the language generated by the grammar. Specifically, given a context-free grammar $G$, we will construct a pushdown automaton $\mathcal{M}$ that functions as a \emph{top-down parser} on its input word $w$; that is, beginning with the start nonterminal $S$, $\mathcal{M}$ will repeatedly apply rules from $R$ to check whether $w$ can be generated via a leftmost derivation. If so, then $\mathcal{M}$ will accept $w$.

\begin{remark}
We could alternatively construct $\mathcal{M}$ to act as a \emph{bottom-up parser}, where it applies rules backward starting from the input word $w$ to see if the start nonterminal $S$ can be reached. The outcome is the same, though, so we will not discuss this alternative construction here.
\end{remark}

Note that, for the purposes of this proof, we will ``condense" multiple transitions of our pushdown automaton into one transition; that is, if we have some sequence of transitions
\begin{center}
\begin{tikzpicture}[node distance=2.75cm, >=latex, every state/.style={fill=white}]
\node[state] (qi) {$q_{i}$};
\node[state] (q1) [right of=qi] {};
\node[state] (q2) [right of=q1] {};
\node[state] (qj) [right of=q2] {$q_{j}$};

\path[-latex] (qi) edge [above] node {$x, A \mapsto B$} (q1);
\path[-latex] (q1) edge [above] node {$\epsilon, \epsilon \mapsto C$} (q2);
\path[-latex] (q2) edge [above] node {$\epsilon, \epsilon \mapsto D$} (qj);
\end{tikzpicture}
\end{center}
then we will depict this sequence of transitions as one single transition of the form
\begin{center}
\begin{tikzpicture}[node distance=3.25cm, >=latex, every state/.style={fill=white}]
\node[state] (qi) {$q_{i}$};
\node[state] (qj) [right of=qi] {$q_{j}$};

\path[-latex] (qi) edge [above] node {$x, A \mapsto BCD$} (qj);
\end{tikzpicture}
\end{center}
and we replace the symbol $A$ on the stack with the symbols $BCD$, in that order from bottom to top.

\begin{lemma}\label{lem:CFGtoPDA}
Given a context-free grammar $G$ generating a language $L(G)$, there exists a pushdown automaton $\mathcal{M}$ such that $L(\mathcal{M}) = L(G)$.

\begin{proof}
Suppose we are given a context-free grammar $G = (V, \Sigma_{G}, R, S)$. We construct a pushdown automaton $\mathcal{M} = (Q, \Sigma, \Gamma, \delta, q_{0}, F)$ that recognizes the language generated by $G$ in the following way:
\begin{itemize}
\item The set of states is $Q = \{q_{S}, q_{R}\}$. The first state, $q_{S}$, corresponds to the point during the computation at which the context-free grammar $G$ begins to generate the word. The second state, $q_{R}$, corresponds to the remainder of the computation where $G$ applies rules from its rule set.
\item The input alphabet is $\Sigma = \Sigma_{G}$. If $\mathcal{M}$ accepts its input word, then the word could be generated by $G$, and therefore it must consist of terminal symbols.
\item The stack alphabet is $\Gamma = V \cup \Sigma_{G}$. We will use the stack of $\mathcal{M}$ to keep track of where we are in the leftmost derivation of the word.
\item The initial state is $q_{0} = q_{S}$.
\item The final state is $F = \{q_{R}\}$.
\item The transition function $\delta$ consists of three types of transitions:
	\begin{enumerate}
	\item \textbf{Initial transition}: $\delta(q_{S}, \epsilon, \epsilon) = \{(q_{R}, S)\}$. This transition initializes the stack by pushing to it the start nonterminal $S$, and then moves to the state $q_{R}$ for the remainder of the computation.
	
	\item \textbf{Nonterminal transition}: $\delta(q_{R}, \epsilon, A) = \{(q_{R}, \alpha_{n}\dots\alpha_{2}\alpha_{1})\}$ for each rule of the form $A \rightarrow \alpha_{1}\alpha_{2}\dots\alpha_{n}$, where $A \in V$ and $\alpha_{i} \in V \cup \Sigma_{G}$ for all $i$. Transitions of this form simulate the application of a given rule by popping the left-hand side ($A$) from the stack and pushing the right-hand side ($\alpha_{1}\alpha_{2}\dots\alpha_{n}$) to the stack in its place in reverse order. Pushing the symbols in reverse ensures that the next symbol we need to read ($\alpha_{1}$) is at the top of the stack.
	
	Note that if $n = 0$, then the transition will be of the form $\delta(q_{R}, \epsilon, A) = \{(q_{R}, \epsilon)\}$.
	
	\item \textbf{Terminal transition}: $\delta(q_{R}, c, c) = \{(q_{R}, \epsilon)\}$ for each terminal symbol $c \in \Sigma_{G}$. Transitions of this form compare a terminal symbol on the stack to the current input word symbol. If the two symbols match, then the computation continues.
	\end{enumerate}
\end{itemize}
During the computation, after the initial transition is followed, $\mathcal{M}$ follows either nonterminal transitions or terminal transitions until its stack is empty or it runs out of input word symbols. If a nonterminal symbol $A$ is at the top of the stack, $\mathcal{M}$ nondeterministically chooses one of the rules for $A$ and follows the corresponding transition. If a terminal symbol $c$ is at the top of the stack, $\mathcal{M}$ performs the comparison between input and stack symbol as described earlier.

By this construction, we can see that $\mathcal{M}$ finishes its computation with an empty stack and no input word symbols of $w$ left to read whenever $S \Rightarrow^{*} w$, and so $\mathcal{M}$ accepts the input word $w$ if $w$ can be generated by the context-free grammar $G$. Therefore, $L(\mathcal{M}) = L(G)$ as desired.
\end{proof}
\end{lemma}

Visually, we can think of the pushdown automaton constructed in the proof of Lemma~\ref{lem:CFGtoPDA} in the following way, where the number of each transition corresponds to its type:
\begin{center}
\begin{tikzpicture}[node distance=2.75cm, >=latex, every state/.style={fill=white}]
\node[state, initial] (qs) {$q_{S}$};
\node[state, accepting] (qr) [right of=qs] {$q_{R}$};

\path[-latex] (qs) edge [above] node {1.\ $\epsilon, \epsilon \mapsto S$} (qr);
\path[-latex] (qr) edge [loop right, align=left] node {2.\ $\epsilon, A \mapsto \alpha_{n}\dots\alpha_{2}\alpha_{1}$ \\ 2.\ $\epsilon, A \mapsto \epsilon$ \\ 3.\ $c, c \mapsto \epsilon$} (qr);
\end{tikzpicture}
\end{center}

Note that we don't require the symbol $\bot$ here, since we're only using the stack to keep track of where we are in the grammar's derivation.

\begin{example}
Consider the following context-free grammar $G$, where $V = \{S, A\}$ and $\Sigma_{G} = \{\texttt{0}, \texttt{1}, \texttt{\#}\}$:
\begin{align*}
S	&\rightarrow	\texttt{0}S\texttt{1} \mid A \\
A	&\rightarrow	\texttt{\#}
\end{align*}
This grammar generates words of the form $\texttt{0}^{n}\texttt{\#}\texttt{1}^{n}$, where $n \geq 0$.

We convert the context-free grammar $G$ to a pushdown automaton $\mathcal{M}$. Take $Q = \{q_{S}, q_{R}\}$, $\Sigma = \Sigma_{G}$, $\Gamma = V \cup \Sigma_{G}$, $q_{0} = q_{S}$, and $F = \{q_{R}\}$. Finally, add the following transitions to $\delta$:
\begin{itemize}
\item $\delta(q_{S}, \epsilon, \epsilon) = \{(q_{R}, S)\}$. This initial transition pushes the start nonterminal $S$ to the stack.
\item $\delta(q_{R}, \epsilon, S) = \{(q_{R}, \texttt{1}S\texttt{0}), (q_{R}, A)\}$. These nonterminal transitions account for the $S$ rules.
\item $\delta(q_{R}, \epsilon, A) = \{(q_{R}, \texttt{\#})\}$. This nonterminal transition accounts for the $A$ rule.
\item $\delta(q_{R}, \texttt{0}, \texttt{0}) = \{(q_{R}, \epsilon)\}$, $\delta(q_{R}, \texttt{1}, \texttt{1}) = \{(q_{R}, \epsilon)\}$, and $\delta(q_{R}, \texttt{\#}, \texttt{\#}) = \{(q_{R}, \epsilon)\}$. These terminal transitions match the terminal symbols on the stack to the input word symbols.
\end{itemize}
This pushdown automaton $\mathcal{M}$ looks like the following:
\begin{center}
\begin{tikzpicture}[node distance=2.75cm, baseline=(current bounding box.center), >=latex, every state/.style={fill=white}]
\node[state, initial] (qs) {$q_{S}$};
\node[state, accepting] (qr) [right of=qs] {$q_{R}$};

\path[-latex] (qs) edge [above] node {$\epsilon, \epsilon \mapsto S$} (qr);
\path[-latex] (qr) edge [loop right] node[align=left] {$\epsilon, S \mapsto \texttt{1}S\texttt{0} \quad \texttt{0}, \texttt{0} \mapsto \epsilon$ \\ $\epsilon, S \mapsto A \quad\quad \texttt{1}, \texttt{1} \mapsto \epsilon$ \\ $\epsilon, A \mapsto \texttt{\#} \quad\quad\, \texttt{\#}, \texttt{\#} \mapsto \epsilon$} (qr);
\end{tikzpicture}
\end{center}

As an illustration of the computation of $\mathcal{M}$, let's look at the stack as $\mathcal{M}$ reads an example input word \texttt{00\#11}. We can see that $G$ generates this word by the derivation $S \Rightarrow \texttt{0}S\texttt{1} \Rightarrow \texttt{00}S\texttt{11} \Rightarrow \texttt{00}A\texttt{11} \Rightarrow \texttt{00\#11}$.
\begin{center}
\begin{tikzpicture}[stack/.style={rectangle split, rectangle split parts=#1, rectangle split part fill={white,white,white,white}, draw, anchor=center}]
\node[stack=4] (S) {
};
\draw[thick, color=white] (S.north west) -- (S.north east);
\node[yshift=-1.25cm] {\texttt{00\#11}};
\end{tikzpicture}
\hspace{0.1cm}
\begin{tikzpicture}[stack/.style={rectangle split, rectangle split parts=#1, rectangle split part fill={white,white,white,\fourthcolour}, draw, anchor=center}]
\node[stack=4] (S) {
\nodepart{four}{\color{\maincolour}$S$}
};
\draw[thick, color=white] (S.north west) -- (S.north east);
\node[yshift=-1.25cm] {\texttt{00\#11}};
\end{tikzpicture}
\hspace{0.1cm}
\begin{tikzpicture}[stack/.style={rectangle split, rectangle split parts=#1, rectangle split part fill={white,\fourthcolour,white,white}, draw, anchor=center}]
\node[stack=4] (S) {
\nodepart{four}\texttt{1}
\nodepart{three}$S$
\nodepart{two}{\color{\maincolour}\texttt{0}}
};
\draw[thick, color=white] (S.north west) -- (S.north east);
\node[yshift=-1.25cm] {\texttt{00\#11}};
\end{tikzpicture}
\hspace{0.1cm}
\begin{tikzpicture}[stack/.style={rectangle split, rectangle split parts=#1, rectangle split part fill={white,white,\fourthcolour,white}, draw, anchor=center}]
\node[stack=4] (S) {
\nodepart{four}\texttt{1}
\nodepart{three}{\color{\maincolour}$S$}
};
\draw[thick, color=white] (S.north west) -- (S.north east);
\node[yshift=-1.25cm] {{\color{\neutralcolour}\cancel{\texttt{0}}}\texttt{0\#11}};
\end{tikzpicture}
\hspace{0.1cm}
\begin{tikzpicture}[stack/.style={rectangle split, rectangle split parts=#1, rectangle split part fill={\fourthcolour,white,white,white}, draw, anchor=center}]
\node[stack=4] (S) {
\nodepart{four}\texttt{1}
\nodepart{three}\texttt{1}
\nodepart{two}$S$
\nodepart{one}{\color{\maincolour}\texttt{0}}
};
\draw[thick, color=\fourthcolour] (S.north west) -- (S.north east);
\node[yshift=-1.25cm] {{\color{\neutralcolour}\cancel{\texttt{0}}}\texttt{0\#11}};
\end{tikzpicture}
\hspace{0.1cm}
\begin{tikzpicture}[stack/.style={rectangle split, rectangle split parts=#1, rectangle split part fill={white,\fourthcolour,white,white}, draw, anchor=center}]
\node[stack=4] (S) {
\nodepart{four}\texttt{1}
\nodepart{three}\texttt{1}
\nodepart{two}{\color{\maincolour}$S$}
};
\draw[thick, color=white] (S.north west) -- (S.north east);
\node[yshift=-1.25cm] {{\color{\neutralcolour}\cancel{\texttt{0}}}{\color{\neutralcolour}\cancel{\texttt{0}}}\texttt{\#11}};
\end{tikzpicture}

\medskip

\begin{tikzpicture}[stack/.style={rectangle split, rectangle split parts=#1, rectangle split part fill={white,\fourthcolour,white,white}, draw, anchor=center}]
\node[stack=4] (S) {
\nodepart{four}\texttt{1}
\nodepart{three}\texttt{1}
\nodepart{two}{\color{\maincolour}$A$}
};
\draw[thick, color=white] (S.north west) -- (S.north east);
\node[yshift=-1.25cm] {{\color{\neutralcolour}\cancel{\texttt{0}}}{\color{\neutralcolour}\cancel{\texttt{0}}}\texttt{\#11}};
\end{tikzpicture}
\hspace{0.1cm}
\begin{tikzpicture}[stack/.style={rectangle split, rectangle split parts=#1, rectangle split part fill={white,\fourthcolour,white,white}, draw, anchor=center}]
\node[stack=4] (S) {
\nodepart{four}\texttt{1}
\nodepart{three}\texttt{1}
\nodepart{two}{\color{\maincolour}\texttt{\#}}
};
\draw[thick, color=white] (S.north west) -- (S.north east);
\node[yshift=-1.25cm] {{\color{\neutralcolour}\cancel{\texttt{0}}}{\color{\neutralcolour}\cancel{\texttt{0}}}\texttt{\#11}};
\end{tikzpicture}
\hspace{0.1cm}
\begin{tikzpicture}[stack/.style={rectangle split, rectangle split parts=#1, rectangle split part fill={white,white,\fourthcolour,white}, draw, anchor=center}]
\node[stack=4] (S) {
\nodepart{four}\texttt{1}
\nodepart{three}{\color{\maincolour}\texttt{1}}
};
\draw[thick, color=white] (S.north west) -- (S.north east);
\node[yshift=-1.25cm] {{\color{\neutralcolour}\cancel{\texttt{0}}}{\color{\neutralcolour}\cancel{\texttt{0}}}{\color{\neutralcolour}\cancel{\texttt{\#}}}\texttt{11}};
\end{tikzpicture}
\hspace{0.1cm}
\begin{tikzpicture}[stack/.style={rectangle split, rectangle split parts=#1, rectangle split part fill={white,white,white,\fourthcolour}, draw, anchor=center}]
\node[stack=4] (S) {
\nodepart{four}{\color{\maincolour}\texttt{1}}
};
\draw[thick, color=white] (S.north west) -- (S.north east);
\node[yshift=-1.25cm] {{\color{\neutralcolour}\cancel{\texttt{0}}}{\color{\neutralcolour}\cancel{\texttt{0}}}{\color{\neutralcolour}\cancel{\texttt{\#}}}{\color{\neutralcolour}\cancel{\texttt{1}}}\texttt{1}};
\end{tikzpicture}
\hspace{0.1cm}
\begin{tikzpicture}[stack/.style={rectangle split, rectangle split parts=#1, rectangle split part fill={white,white,white,white}, draw, anchor=center}]
\node[stack=4] (S) {
};
\draw[thick, color=white] (S.north west) -- (S.north east);
\node[yshift=-1.25cm] {{\color{\neutralcolour}\cancel{\texttt{0}}}{\color{\neutralcolour}\cancel{\texttt{0}}}{\color{\neutralcolour}\cancel{\texttt{\#}}}{\color{\neutralcolour}\cancel{\texttt{1}}}{\color{\neutralcolour}\cancel{\texttt{1}}}};
\end{tikzpicture}
\end{center}
\end{example}

\subsection{$\PDA \Rightarrow \CFG$}

Now, we consider the other half of our main result. In order to convert a pushdown automaton to a context-free grammar, we must first ensure the pushdown automaton has certain properties: namely, the pushdown automaton must have a single accepting state, it must empty its stack before accepting, and each transition of the pushdown automaton must either push to or pop from the stack, but not both simultaneously. Let us refer to a pushdown automaton with these properties as a \emph{simplified pushdown automaton}.

Fortunately, it's easy to convert from a pushdown automaton to a simplified pushdown automaton.

\begin{colouredbox}
\begin{itemize}
\item To ensure the pushdown automaton has a single accepting state, we make each original accepting state non-accepting and add epsilon transitions from those states to a new single accepting state.
\begin{center}
\begin{tikzpicture}[node distance=2cm, baseline=(current bounding box.center), >=latex, every state/.style={fill=white}]
\node[state, accepting, inner sep=1pt, minimum size=1.5em] (q1) {};
\node[state, accepting, inner sep=1pt, minimum size=1.5em] [below right=1cm of q1] (q2) {};
\node[state, accepting, inner sep=1pt, minimum size=1.5em] [above right=1cm of q1] (q3) {};
\end{tikzpicture}
\hspace{0.25cm}
$\Longrightarrow$
\hspace{0.5cm}
\begin{tikzpicture}[node distance=2cm, baseline=(current bounding box.center), >=latex, every state/.style={fill=white}]
\node[state, inner sep=1pt, minimum size=1.5em] (q1) {};
\node[state, inner sep=1pt, minimum size=1.5em] [below right=1cm of q1] (q2) {};
\node[state, inner sep=1pt, minimum size=1.5em] [above right=1cm of q1] (q3) {};
\node[state, accepting, inner sep=1pt, minimum size=1.5em] [right=1.66cm of q1] (qa) {};

\path[-latex] (q1) edge [above, pos=0.45] node {$\epsilon, \epsilon \mapsto \epsilon$} (qa);
\path[-latex] (q2) edge [below right] node {$\epsilon, \epsilon \mapsto \epsilon$} (qa);
\path[-latex] (q3) edge [above right] node {$\epsilon, \epsilon \mapsto \epsilon$} (qa);
\end{tikzpicture}
\end{center}

\item To ensure the pushdown automaton empties its stack before accepting, we add a state immediately before the accepting state that removes all symbols from the stack.
\begin{center}
\begin{tikzpicture}[node distance=2cm, baseline=(current bounding box.center), >=latex, every state/.style={fill=white}]
\node[state, accepting, inner sep=1pt, minimum size=1.5em] (q1) {};
\end{tikzpicture}
\hspace{0.5cm}
$\Longrightarrow$
\hspace{-0.25cm}
\begin{tikzpicture}[node distance=2.75cm, baseline={([yshift=-0.6cm]current bounding box.center)}, >=latex, every state/.style={fill=white}]
\node[state, inner sep=1pt, minimum size=1.5em] (q0) {};
\node[state, accepting, inner sep=1pt, minimum size=1.5em] (q1) [right=1.75cm of q0] {};

\path[-latex] (q0) edge [loop above] node[align=center, font={\small}] {$\epsilon, A \mapsto \epsilon$ \\ for each $A \in \Gamma$} (q0);
\path[-latex] (q0) edge [above] node {$\epsilon, \epsilon \mapsto \epsilon$} (q1);
\end{tikzpicture}
\end{center}

\item To ensure that each transition of the pushdown automaton either pushes to or pops from the stack, but not both, we split each transition that both pushes and pops into two separate transitions.
\begin{center}
\begin{tikzpicture}[node distance=2.75cm, baseline=(current bounding box.center), >=latex, every state/.style={fill=white}]
\node[state, inner sep=1pt, minimum size=1.5em] (q1) {};
\node[state, inner sep=1pt, minimum size=1.5em] (q2) [right=1.75cm of q1] {};

\path[-latex] (q1) edge [above] node {$x, A \mapsto B$} (q2);
\end{tikzpicture}
\hspace{0.25cm}
$\Longrightarrow$
\hspace{0.25cm}
\begin{tikzpicture}[node distance=2.75cm, baseline=(current bounding box.center), >=latex, every state/.style={fill=white}]
\node[state, inner sep=1pt, minimum size=1.5em] (q1) {};
\node[state, inner sep=1pt, minimum size=1.5em] (qa) [right=1.75cm of q1] {};
\node[state, inner sep=1pt, minimum size=1.5em] (q2) [right=1.75cm of qa] {};

\path[-latex] (q1) edge [above] node {$x, A \mapsto \epsilon$} (qa);
\path[-latex] (qa) edge [above] node {$\epsilon, \epsilon \mapsto B$} (q2);
\end{tikzpicture}
\end{center}

Additionally, if we have an epsilon transition that neither pushes nor pops, then we replace it with two ``dummy" transitions that push and then immediately pop the same stack symbol.
\begin{center}
\begin{tikzpicture}[node distance=2.75cm, baseline=(current bounding box.center), >=latex, every state/.style={fill=white}]
\node[state, inner sep=1pt, minimum size=1.5em] (q1) {};
\node[state, inner sep=1pt, minimum size=1.5em] (q2) [right=1.75cm of q1] {};

\path[-latex] (q1) edge [above] node {$x, \epsilon \mapsto \epsilon$} (q2);
\end{tikzpicture}
\hspace{0.25cm}
$\Longrightarrow$
\hspace{0.25cm}
\begin{tikzpicture}[node distance=2.75cm, baseline=(current bounding box.center), >=latex, every state/.style={fill=white}]
\node[state, inner sep=1pt, minimum size=1.5em] (q1) {};
\node[state, inner sep=1pt, minimum size=1.5em] (qa) [right=1.75cm of q1] {};
\node[state, inner sep=1pt, minimum size=1.5em] (q2) [right=1.75cm of qa] {};

\path[-latex] (q1) edge [above] node {$x, \epsilon \mapsto A$} (qa);
\path[-latex] (qa) edge [above] node {$\epsilon, A \mapsto \epsilon$} (q2);
\end{tikzpicture}
\end{center}
\end{itemize}
\end{colouredbox}

With a simplified pushdown automaton, we can now perform the conversion to a context-free grammar.

\begin{lemma}\label{lem:PDAtoCFG}
Given a simplified pushdown automaton $\mathcal{M}$ recognizing a language $L(\mathcal{M})$, there exists a context-free grammar $G$ such that $L(G) = L(\mathcal{M})$.

\begin{proof}
Suppose we are given a simplified pushdown automaton $\mathcal{M} = (Q, \Sigma, \Gamma, \delta, q_{0}, q_{\text{accept}})$. We will construct a context-free grammar $G = (V, \Sigma_{G}, R, S)$ that generates the language recognized by $\mathcal{M}$.

For each pair of states $p$ and $q$ in $\mathcal{M}$, our grammar will include a rule $A_{pq}$ that simulates the computation of $\mathcal{M}$ starting in state $p$ with some stack contents and ending in state $q$ with the same stack contents. (Note that the stack may be manipulated during this computation; we just ensure that the contents of the stack are the same at the beginning and the end.)

We construct $G$ in the following way:
\begin{itemize}
\item The set of nonterminal symbols is $V = \{A_{pq} \mid p, q, \in Q\}$.
\item The set of terminal symbols is $\Sigma_{G} = \Sigma$.
\item The start nonterminal is $S = A_{q_{0} q_{\text{accept}}}$ (i.e., the rule corresponding to the computation starting in state $q_{0}$ and ending in state $q_{\text{accept}}$).
\item The set of rules $R$ consists of three types of rules:
	\begin{enumerate}
	\item For each state $q \in Q$, add the rule $A_{qq} \rightarrow \epsilon$ to $R$.
	\begin{center}
	\begin{tikzpicture}[node distance=1.5cm, >=latex, every state/.style={fill=white}]
	\node[state, inner sep=1pt, minimum size=1.5em] (q) {$q$};
	\end{tikzpicture}
	\end{center}
	
	\item For each triplet of states $p, q, r \in Q$, add the rule $A_{pr} \rightarrow A_{pq}A_{qr}$ to $R$.
	\begin{center}
	\begin{tikzpicture}[node distance=1.5cm, >=latex, every state/.style={fill=white}, decoration={%
	snake,
	segment length=2mm,
	amplitude=0.4mm,
	pre length=4pt,
	post length=4pt,
	}]
	\node[state, inner sep=1pt, minimum size=1.5em] (p) {$p$};
	\node[state, inner sep=1pt, minimum size=1.5em] (q) [right of=p] {$q$};
	\node[state, inner sep=1pt, minimum size=1.5em] (r) [right of=q] {$r$};
	
	\path[-latex, draw=black, decorate] (p) -- (q);
	\path[-latex, draw=black, decorate] (q) -- (r);
	\end{tikzpicture}
	\end{center}
	
	\item For each quadruplet of states $p, q, r, s \in Q$, input symbols $\texttt{a}, \texttt{b} \in \Sigma \cup \{\epsilon\}$, and stack symbol $\texttt{T} \in \Gamma$, if $(q, \texttt{T}) \in \delta(p, \texttt{a}, \epsilon)$ and $(s, \epsilon) \in \delta(r, \texttt{b}, \texttt{T})$, then add the rule $A_{ps} \rightarrow \texttt{a}A_{qr}\texttt{b}$ to $R$.
	\begin{center}
	\begin{tikzpicture}[node distance=1.5cm, >=latex, every state/.style={fill=white}, decoration={%
	snake,
	segment length=2mm,
	amplitude=0.4mm,
	pre length=4pt,
	post length=4pt,
	}]
	\node[state, inner sep=1pt, minimum size=1.5em] (p) {$p$};
	\node[state, inner sep=1pt, minimum size=1.5em] (q) [right=2cm of p] {$q$};
	\node[state, inner sep=1pt, minimum size=1.5em] (r) [below of=p] {$r$};
	\node[state, inner sep=1pt, minimum size=1.5em] (s) [right=2cm of r] {$s$};
	
	\path[-latex] (p) edge [above] node {$\texttt{a}, \epsilon \mapsto \texttt{T}$} (q);
	\path[-latex, draw=black, decorate] (q) -- (r);
	\path[-latex] (r) edge [below] node {$\texttt{b}, \texttt{T} \mapsto \epsilon$} (s);
	\end{tikzpicture}
	\end{center}
	\end{enumerate}
\end{itemize}

The first type of rule is a ``dummy" rule that essentially corresponds to staying in the state $q$ and adding nothing to the derivation. The second type of rule breaks down the overall computation into smaller components, taking into account intermediate states. Finally, the third type of rule adds terminal symbols to the derivation depending on the components of the overall computation.

With these rules, we can establish that the rule $A_{q_{0}q_{\text{accept}}}$ generates a word $w$ if and only if, starting in the state $q_{0}$ with an empty stack, the computation of $\mathcal{M}$ on $w$ ends in the state $q_{\text{accept}}$ also with an empty stack. Therefore, $w$ is generated by the context-free grammar $G$ if $\mathcal{M}$ accepts $w$, and $L(G) = L(\mathcal{M})$ as desired.
\end{proof}
\end{lemma}

\begin{example}
Consider the following simplified pushdown automaton $\mathcal{M}$, where $\Sigma = \{\texttt{0}, \texttt{1}\}$ and $\Gamma = \{\texttt{X}, \texttt{Y}\}$:
\begin{center}
\begin{tikzpicture}[node distance=2.75cm, baseline=(current bounding box.center), >=latex, every state/.style={fill=white}]
\node[state, initial] (q0) {$q_{0}$};
\node[state] (q1) [right of=q0] {$q_{1}$};
\node[state] (q2) [below right=1cm of q1] {$q_{2}$};
\node[state] (q3) [below of=q1] {$q_{3}$};
\node[state, accepting] (q4) [left of=q3] {$q_{4}$};

\path[-latex] (q0) edge [above] node {$\epsilon, \epsilon \mapsto \bot$} (q1);
\path[-latex] (q1) edge [loop above] node[align=center] {$\texttt{0}, \epsilon \mapsto \texttt{X}$ \\ $\texttt{1}, \epsilon \mapsto \texttt{Y}$} (q1);
\path[-latex] (q1) edge [above right] node {$\epsilon, \epsilon \mapsto \texttt{X}$} (q2);
\path[-latex] (q2) edge [below right] node {$\epsilon, \texttt{X} \mapsto \epsilon$} (q3);
\path[-latex] (q3) edge [loop below] node[align=center] {$\texttt{0}, \texttt{Y} \mapsto \epsilon$ \\ $\texttt{1}, \texttt{X} \mapsto \epsilon$} (q3);
\path[-latex] (q3) edge [above] node {$\epsilon, \bot \mapsto \epsilon$} (q4);
\end{tikzpicture}
\end{center}
This pushdown automaton recognizes words of the form $w \cdot \overline{w}^{\text{R}}$, where $\overline{w}$ is $w$ with \texttt{0}s and \texttt{1}s swapped.

We convert the pushdown automaton $\mathcal{M}$ to a context-free grammar $G$. Let $V = \{A_{00}, A_{01}, A_{02}, A_{03}, A_{04}, A_{11}, A_{12}, A_{13}, A_{14}, A_{22}, A_{23}, A_{24}, \allowbreak A_{33}, A_{34}, A_{44}\}$ and take $\Sigma_{G} = \Sigma$. We also take $S = A_{04}$, since $q_{0}$ is the initial state and $q_{4}$ is the accepting state of $\mathcal{M}$. Finally, we add the following rules to the rule set $R$:
\begin{itemize}
\item Type 1 rules: 
$A_{00} \rightarrow \epsilon$, 
$A_{11} \rightarrow \epsilon$, 
$A_{22} \rightarrow \epsilon$,  
$A_{33} \rightarrow \epsilon$, and 
$A_{44} \rightarrow \epsilon$.
\item Type 2 rules: 
\begin{align*}
A_{01} &\rightarrow A_{00}A_{01} \mid A_{01}A_{11} \\
A_{02} &\rightarrow A_{00}A_{02} \mid  A_{01}A_{12} \mid A_{02}A_{22} \\
A_{03} &\rightarrow A_{00}A_{03} \mid A_{01}A_{13} \mid A_{02}A_{23} \mid A_{03}A_{33} \\
A_{04} &\rightarrow A_{00}A_{04} \mid A_{01}A_{14} \mid A_{02}A_{24} \mid A_{03}A_{34} \mid A_{04}A_{44} \\
A_{12} &\rightarrow A_{11}A_{12} \mid A_{12}A_{22} \\
A_{13} &\rightarrow A_{11}A_{13} \mid A_{12}A_{23} \mid A_{13}A_{33} \\
A_{14} &\rightarrow A_{11}A_{14} \mid A_{12}A_{24} \mid A_{13}A_{34} \mid A_{14}A_{44} \\
A_{23} &\rightarrow A_{22}A_{23} \mid A_{23}A_{33} \\
A_{24} &\rightarrow A_{22}A_{24} \mid A_{23}A_{34} \mid A_{24}A_{44} \\
A_{34} &\rightarrow A_{33}A_{34} \mid A_{34}A_{44}
\end{align*}
\item Type 3 rules: 
\begin{align*}
A_{13} &\rightarrow \texttt{0}A_{13}\texttt{1} \mid \texttt{1}A_{13}\texttt{0} \mid \epsilon A_{22} \epsilon \text{ (or just } A_{22} \text{)} \\
A_{04} &\rightarrow \epsilon A_{13} \epsilon \text{ (or just } A_{13} \text{)}
\end{align*}
\end{itemize}

As an illustration, let's see how $G$ derives an example input word \texttt{001011}. Beginning from the start nonterminal $A_{q_{0}q_{\text{accept}}} = A_{04}$, the derivation proceeds in the following way:
\begin{align*}
A_{04} &\Rightarrow \highlightmath{A_{13}} \\
&\Rightarrow \highlightmath{\texttt{0}A_{13}\texttt{1}} \\
&\Rightarrow \texttt{0}\highlightmath{\texttt{0}A_{13}\texttt{1}}\texttt{1} \\
&\Rightarrow \texttt{00}\highlightmath{\texttt{1}A_{13}\texttt{0}}\texttt{11} \\
&\Rightarrow \texttt{001}\highlightmath{A_{22}}\texttt{011} \\
&\Rightarrow \texttt{001}\highlightmath{\epsilon}\texttt{011} = \texttt{001011}.
\end{align*}
\end{example}

\subsection{$\CFG = \PDA$}

Since we know by Definition~\ref{def:contextfreelanguage} that a language is context-free if there exists a context-free grammar generating the language, we can combine the previous two lemmas to get the main result of this section.

\begin{theorem}\label{thm:CFGPDAequivalence}
A language $C$ is context-free if it satisfies any of the following equivalent properties:
\begin{enumerate}
\item There exists a context-free grammar $G$ such that $L(G) = C$; or
\item There exists a pushdown automaton $\mathcal{M}$ such that $L(\mathcal{M}) = C$.
\end{enumerate}
\end{theorem}

We can think of this result as the final piece to obtain the context-free analogue of Kleene's theorem for the regular languages. Since context-free grammars generate context-free languages, and since context-free grammars can be converted to pushdown automata and vice versa, both models correspond to the exact same language class. Unfortunately, this result doesn't get a nice name like Kleene's theorem did, but perhaps the lack of a name is justified when you consider the diagram we get isn't as interesting as the one we had for the regular languages:

\begin{center}
\begin{tikzpicture}
\node (cfg) at (0,0) {\CFG};
\node (pda) at (3,0) {\PDA};

\draw[latex-latex] (cfg) -- (pda);
\end{tikzpicture}
\end{center}

\noindent
(Not exactly a \textit{Scutum Fidei} as before, but maybe a \textit{Gladius Fidei}?)

We're not yet finished, though. Thanks to the equivalence between context-free grammars and pushdown automata, we can establish an important result that relates the class of context-free languages to the class of regular languages.

\begin{theorem}\label{thm:regulariscontextfree}
Every regular language is also a context-free language.

\begin{proof}
Every regular language is recognized by some finite automaton. Since a finite automaton is a pushdown automaton that does not use the stack, every regular language is also recognized by some pushdown automaton. Furthermore, by Theorem~\ref{thm:CFGPDAequivalence}, every regular language is generated by some context-free grammar. Therefore, every regular language is context-free.
\end{proof}
\end{theorem}

Of course, we already know that there exist some context-free languages that are not regular, so this inclusion only works in one direction.

\section{Closure Properties}\label{sec:closurepropertiescontextfree}

\begin{construction}
Much like in the chapter on regular languages, here I intend to summarize the closure properties of various operations applied to context-free languages. This section may prove to be more interesting, since certain operations turn out not to be closed for the class of context-free languages.
\end{construction}

\subsubsection*{Intersection}

While it is true that context-free languages are closed under union, it is in fact \emph{not} true that they're also closed under intersection.

\begin{theorem}\label{thm:CFLnonclosureintersection}
The class of context-free languages is not closed under intersection.

\begin{proof}
Consider the languages
\begin{align*}
L_{1}	&= \{\texttt{a}^{n}\texttt{b}^{n}\texttt{c}^{m} \mid m, n \geq 0\} \text{ and} \\
L_{2}	&= \{\texttt{a}^{m}\texttt{b}^{n}\texttt{c}^{n} \mid m, n \geq 0\}.
\end{align*}
Both of these languages are context-free, so if the class of context-free languages were closed under intersection, the language $L_{1} \cap L_{2}$ must also be context-free. However, we can see that
\begin{equation*}
L_{1} \cap L_{2} = \{\texttt{a}^{n}\texttt{b}^{n}\texttt{c}^{n} \mid n \geq 0\}.
\end{equation*}
This language is not context-free, and we can reason informally about this fact as follows: we can use the stack of a pushdown automaton to count $n$ \texttt{a}s and match these symbols to $n$ \texttt{b}s, but after this point we can no longer use the stack to count an equal number of \texttt{c}s.
\end{proof}
\end{theorem}

\begin{remark}
In the following section, we will formally prove that the language $L_{1} \cap L_{2} = \{\texttt{a}^{n}\texttt{b}^{n}\texttt{c}^{n} \mid n \geq 0\}$ is non-context-free.
\end{remark}

\subsubsection*{Complement}

We saw in our study of regular language closure properties that, if closure holds under both union and intersection, then closure must also hold under complement by De Morgan's laws. Since the class of context-free languages is not closed under intersection, it is therefore also not closed under complement.

\begin{theorem}\label{thm:CFLnonclosurecomplement}
The class of context-free languages is not closed under complement.

\begin{proof}
Follows as a consequence of non-closure of context-free languages under intersection.
\end{proof}
\end{theorem}
\section{Proving a Language is Non-Context-Free}\label{sec:noncontextfree}

\firstwords{At the end} of our discussion on regular languages, we saw that there exist certain languages that are nonregular, and we also saw that we can prove a language is nonregular by using the pumping lemma. One of the biggest obstacles we observed that results in a language being nonregular was, broadly speaking, having to count or otherwise keep track of symbols. Fortunately, by augmenting our machine model with a stack and creating a pushdown automaton, we were able to overcome this obstacle. Surely, this means that we can now recognize any language we want, right?

Well, not exactly. While the stack goes a long way in helping us to recognize more than just the class of regular languages, it isn't the magic solution we need in order to recognize \emph{any} language. Consider, for example, the language
\begin{equation*}
L_{\text{a}=\text{b}=\text{c}} = \{\texttt{a}^{n}\texttt{b}^{n}\texttt{c}^{n} \mid n \geq 0\}.
\end{equation*}
We know that a pushdown automaton can accept words of the form $\texttt{a}^{n}\texttt{b}^{n}$ by pushing one symbol to the stack for each \texttt{a} that is read, and then popping one symbol from the stack for each \texttt{b} that is read. When it comes to recognizing words of the form $\texttt{a}^{n}\texttt{b}^{n}\texttt{c}^{n}$, however, we run into a problem: after we read all of the \texttt{b}s in the word, our stack will be empty and we will therefore have forgotten the value of $n$ by the time we have to count the \texttt{c}s! We also can't cheat our way around this problem by, for example, pushing two symbols to the stack for each \texttt{a} we read; if we try that, then reading either \texttt{b} or \texttt{c} would require us to pop the same symbol, and we can draw a conclusion that such an approach would result in the pushdown automaton accidentally accepting words where \texttt{b}s and \texttt{c}s are either out of order or having mismatched counts.

Thus, there do indeed exist languages that are not context-free, and so we require a technique to prove the non-context-freeness of a language. Fortunately, we're mostly familiar with such a technique already: the pumping lemma for regular languages is a special case of the more general pumping lemma for context-free languages.

\subsection{The Pumping Lemma for Context-Free Languages}

You might be wondering at this point what we mean by the pumping lemma for regular languages being a ``special case". Since regular languages are, in a sense, simpler than context-free languages, our formulation of the pumping lemma for regular languages was accordingly simpler: pumping the middle portion of any sufficiently long word in a regular language results in us obtaining another word that also belongs to the language. Pumping this middle portion of the word essentially corresponds to us traversing a loop somewhere in the finite automaton recognizing the language.

With context-free languages, however, we can't just pump one portion of the word. To understand why not, recall that we can represent the derivation of a word in a context-free language using a parse tree. If our word is sufficiently long, then the parse tree will be rather deep. Then, since we have only a finite number of both rules and nonterminal symbols, the pigeonhole principle tells us that there must exist some path from the root of the parse tree to a leaf of the parse tree where some nonterminal symbol $R$ appears more than once along that path.

\begin{figure}
\centering
\begin{tikzpicture}[relative]
\draw (0,0) -- (3,-3) -- (-3,-3) -- cycle;
\draw[fill=\fifthcolour] (0,-1) -- (2,-3) -- (-2,-3) -- cycle;
\draw[fill=\fourthcolour] (0,-2) -- (1,-3) -- (-1,-3) -- cycle;

\draw[fill=white] (0,0) circle [radius=0.3] node (S) {$S$};
\draw[fill=white] (0,-1) circle [radius=0.3] node (R1) {$R$};
\draw[fill=white] (0,-2) circle [radius=0.3] node (R2) {$R$};

\node[anchor=north] at (-2.5, -3.15) {$u$};
\node[anchor=north] at (-1.5, -3.15) {\color{\secondcolour}$v$};
\node[anchor=north] at (0, -3.15) {\color{\maincolour}$x$};
\node[anchor=north] at (1.5, -3.15) {\color{\secondcolour}$y$};
\node[anchor=north] at (2.5, -3.15) {$z$};

\draw[dashed, thick, color=\maincolour] (S.240) to[out=-60, in=100] (R1.40);
\draw[dashed, thick, color=\maincolour] (R1.290) to[out=100, in=-60] (R2.130);
\draw[dashed, thick, color=\maincolour] (R2.290) to[out=20, in=180] (0,-3);

\draw[fill=white] (0,0) circle [radius=0.3] node {$S$};
\draw[fill=\maincolour] (0,-1) circle [radius=0.3] node {\color{white}$R$};
\draw[fill=\maincolour] (0,-2) circle [radius=0.3] node {\color{white}$R$};
\end{tikzpicture}
\caption{A path through a parse tree that visits some nonterminal symbol $R$ more than once}
\label{fig:pumpingparsetree}
\end{figure}

\begin{remark}
``Sufficiently long" in this context is measured in terms of both the maximum number of symbols on the right-hand side of any rule of the context-free grammar and the size of the set of nonterminal symbols.
\end{remark}

Suppose that we decompose our word $w$ not into three parts as we did with the regular languages, but into five parts, denoted $uvxyz$. This decomposition, represented as a parse tree, is shown in Figure~\ref{fig:pumpingparsetree}. Considering the subtree rooted at the first occurrence of $R$, we can regard everything ``outside of" this subtree as the $u$ and $z$ portions of the word, while everything ``inside of" this subtree comprises the $vxy$ portion of the word. Since we know that $R$ reappears at some point within this subtree, we can further consider the subtree rooted at the second occurrence of $R$, where everything ``inside of" this subtree is the middle $x$ portion of the word. Observe that we can repeat the first subtree rooted at $R$ as many times as we want by appending the subtree to some later occurrence of $R$. In doing this, we are effectively pumping the segment of the subtree ``between" both occurrences of $R$ when we perform this repetition; that is, we are pumping the $v$ and $y$ portions of the word together, as depicted in Figure~\ref{fig:pumpingparsetreepumped}.

\begin{figure}
\begin{subfigure}{\textwidth}
\centering
\begin{tikzpicture}[relative]
\draw (0,0) -- (3,-3) -- (-3,-3) -- cycle;
\draw (0,-1) -- (2,-3) -- (-2,-3) -- cycle;
\draw[fill=\fourthcolour] (0,-1) -- (1,-2) -- (-1,-2) -- cycle;

\draw[color=white, thick] (-1.98,-3) -- (1.98,-3);

\draw[fill=white] (0,0) circle [radius=0.3] node (S) {$S$};
\draw[fill=white] (0,-1) circle [radius=0.3] node (R2) {$R$};

\node[anchor=north] at (-2.5, -3.15) {$u$};
\node[anchor=north] at (0, -2.15) {\color{\maincolour}$x$};
\node[anchor=north] at (2.5, -3.15) {$z$};

\draw[dashed, thick, color=\maincolour] (S.240) to[out=-60, in=100] (R1.40);
\draw[dashed, thick, color=\maincolour] (R2.290) to[out=20, in=180] (0,-2);

\draw[fill=white] (0,0) circle [radius=0.3] node {$S$};
\draw[fill=\maincolour] (0,-1) circle [radius=0.3] node {\color{white}$R$};
\end{tikzpicture}
\caption{``Pumping" the parse subtree rooted at $R$ zero times}
\end{subfigure}

\bigskip

\begin{subfigure}{\textwidth}
\centering
\begin{tikzpicture}[relative]
\draw (0,0) -- (3,-3) -- (-3,-3) -- cycle;
\draw[fill=\fifthcolour] (0,-1) -- (2,-3) -- (-2,-3) -- cycle;
\draw[fill=\fourthcolour] (0,-2) -- (1,-3) -- (-1,-3) -- cycle;

\draw[fill=white] (0,0) circle [radius=0.3] node (S) {$S$};
\draw[fill=white] (0,-1) circle [radius=0.3] node (R1) {$R$};
\draw[fill=white] (0,-2) circle [radius=0.3] node (R2) {$R$};

\node[anchor=north] at (-2.5, -3.15) {$u$};
\node[anchor=north] at (-1.5, -3.15) {\color{\secondcolour}$v$};
\node[anchor=north] at (0, -3.15) {\color{\maincolour}$x$};
\node[anchor=north] at (1.5, -3.15) {\color{\secondcolour}$y$};
\node[anchor=north] at (2.5, -3.15) {$z$};

\draw[dashed, thick, color=\maincolour] (S.240) to[out=-60, in=100] (R1.40);
\draw[dashed, thick, color=\maincolour] (R1.290) to[out=100, in=-60] (R2.130);
\draw[dashed, thick, color=\maincolour] (R2.290) to[out=20, in=180] (0,-3);

\draw[fill=white] (0,0) circle [radius=0.3] node {$S$};
\draw[fill=\maincolour] (0,-1) circle [radius=0.3] node {\color{white}$R$};
\draw[fill=\maincolour] (0,-2) circle [radius=0.3] node {\color{white}$R$};
\end{tikzpicture}
\caption{``Pumping" the parse subtree rooted at $R$ one time}
\end{subfigure}

\bigskip

\begin{subfigure}{\textwidth}
\centering
\begin{tikzpicture}[relative]
\draw (0,0) -- (3,-3) -- (-3,-3) -- cycle;
\draw[fill=\fifthcolour] (0,-1) -- (2,-3) -- (-2,-3) -- cycle;
\draw[fill=\fifthcolour] (0,-2) -- (2,-4) -- (-2,-4) -- cycle;
\draw[fill=\fourthcolour] (0,-3) -- (1,-4) -- (-1,-4) -- cycle;

\draw[fill=white] (0,0) circle [radius=0.3] node (S) {$S$};
\draw[fill=white] (0,-1) circle [radius=0.3] node (R1) {$R$};
\draw[fill=white] (0,-2) circle [radius=0.3] node (R2) {$R$};
\draw[fill=white] (0,-3) circle [radius=0.3] node (R3) {$R$};

\node[anchor=north] at (-2.5, -3.15) {$u$};
\node[anchor=north] at (-1.7, -3.15) {\color{\secondcolour}$v$};
\node[anchor=north] at (-1.5, -4.15) {\color{\secondcolour}$v$};
\node[anchor=north] at (0, -4.15) {\color{\maincolour}$x$};
\node[anchor=north] at (1.5, -4.15) {\color{\secondcolour}$y$};
\node[anchor=north] at (1.7, -3.15) {\color{\secondcolour}$y$};
\node[anchor=north] at (2.5, -3.15) {$z$};

\draw[dashed, thick, color=\maincolour] (S.240) to[out=-60, in=100] (R1.40);
\draw[dashed, thick, color=\maincolour] (R1.290) to[out=100, in=-60] (R2.130);
\draw[dashed, thick, color=\maincolour] (R2.290) to[out=100, in=-60] (R3.130);
\draw[dashed, thick, color=\maincolour] (R3.290) to[out=20, in=180] (0,-4);

\draw[fill=white] (0,0) circle [radius=0.3] node {$S$};
\draw[fill=\maincolour] (0,-1) circle [radius=0.3] node {\color{white}$R$};
\draw[fill=\maincolour] (0,-2) circle [radius=0.3] node {\color{white}$R$};
\draw[fill=\maincolour] (0,-3) circle [radius=0.3] node {\color{white}$R$};
\end{tikzpicture}
\caption{``Pumping" the parse subtree rooted at $R$ two times}
\end{subfigure}
\caption{Three examples of the pumping lemma for context-free languages applied to the parse tree depicted in Figure~\ref{fig:pumpingparsetree}}
\label{fig:pumpingparsetreepumped}
\end{figure}

With this idea in mind, the formal statement of the pumping lemma for context-free languages is quite similar to that of the pumping lemma for regular languages, modulo the appropriate changes.

\begin{lemma}[Pumping lemma for context-free languages]\label{lem:pumpingcontextfree}
For all context-free languages $L$, there exists $p \geq 1$ where, for all $w \in L$ with $|w| \geq p$, there exists a decomposition of $w$ into five parts $w = uvxyz$ such that
\begin{enumerate}
\item $|vy| > 0$;
\item $|vxy| \leq p$; and
\item for all $i \geq 0$, $uv^{i}xy^{i}z \in L$.
\end{enumerate}
\end{lemma}

\begin{construction}
I plan on adding a little more motivation for the proof here, before we jump right into the fine details.
\end{construction}

\begin{proofbox}[Lemma~\ref{lem:pumpingcontextfree}]
Let $G = (V, \Sigma, R, S)$ be a context-free grammar such that $L(G) = L$, and let $b \geq 2$ denote the \emph{branching factor} of $G$; that is, the maximum number of terminal and nonterminal symbols occurring on the right-hand side of any rule of $G$. If the height of some parse tree of $G$ is $h$, then the length of any word derived using that parse tree will be at most $b^{h}$.

Let $n = |V|$ denote the number of nonterminal symbols of $G$, and take $p = b^{n + 1}$. By our earlier observation, any word generated by $G$ whose parse tree contains no path with a repeated nonterminal must have length at most $b^{n}$. Moreover, since $b \geq 2$, we must have that $b^{n+1} > b^{n}$.

Let $w$ be any word in $L(G)$ where $|w| \geq p$, and let $T$ be a parse tree for $w$ of minimal size. We know, again by our earlier observation, that $T$ must have a height of at least $n + 1$. Choose some path in $T$ with length at least $n + 1$, and take $R$ to be the deepest nonterminal in the parse tree that occurs more than once in this path.

Decompose the word $w$ into five parts, $uvxyz$, such that the subtree rooted at the upper occurrence of $R$ has height at most $n + 1$ and the parts $u$ and $z$ are outside of the subtree rooted at the upper occurrence of $R$. We must have that $|vy| > 0$, since otherwise there would exist a smaller parse tree for $w$, which contradicts our assumption that $T$ was of minimal size. Furthermore, the yield of this subtree, $vxy$, is a subword with length at most $p = b^{n + 1}$. Finally, the word $uxz$ is in $L$ since we could replace the subtree rooted at the upper occurrence of $R$ with the subtree rooted at the lower occurrence of $R$ that yields only the part $x$, and for $i \geq 1$, all words of the form $uv^{i}xy^{i}z$ are in $L$ since we can place copies of the subtree rooted at the upper occurrence of $R$ at each subsequent occurrence of $R$. Therefore, all three conditions of the pumping lemma are satisfied.
\end{proofbox}

\subsubsection*{Using the Pumping Lemma}

Just like before, we can write a proof that some language is non-context-free by simply following a common set of steps. As a consequence, all non-context-freeness proofs share a similar structure. To see such an example of a proof, let's revisit the language we introduced at the beginning of this section.

\begin{example}
Let $\Sigma = \{\texttt{a}, \texttt{b}, \texttt{c}\}$, and consider the language
\begin{equation*}
L_{\text{a}=\text{b}=\text{c}} = \{\texttt{a}^{n}\texttt{b}^{n}\texttt{c}^{n} \mid n \geq 0\}.
\end{equation*}
We will use the pumping lemma to show that this language is non-context-free.

Assume by way of contradiction that the language is context-free, and let $p$ denote the pumping constant given by the pumping lemma. We choose the word $w = \texttt{a}^{p}\texttt{b}^{p}\texttt{c}^{p}$. Clearly, $w \in L_{\text{a}=\text{b}=\text{c}}$ and $|w| \geq p$. Thus, there exists a decomposition $w = uvxyz$ satisfying the three conditions of the pumping lemma.

Observe that the first condition of the pumping lemma requires that \emph{either} part $v$ or part $y$ is nonempty; potentially both could be nonempty. We consider two cases, depending on the contents of the parts $v$ and $y$ of the word $w$:
\begin{enumerate}
\item Both part $v$ and part $y$ contain some number of a single alphabet symbol; that is, $v$ contains only \texttt{a}s, only \texttt{b}s, or only \texttt{c}s, and likewise for $y$. (Note that $v$ and $y$ do not need to contain the \emph{same} alphabet symbol; for example, $v$ could contain only \texttt{a}s and $y$ could contain only \texttt{b}s.)

In this case, pumping $v$ and $y$ once to obtain the word $uv^{2}xy^{2}z$ results in the word containing unequal numbers of \texttt{a}s, \texttt{b}s, and \texttt{c}s. This violates the third condition of the pumping lemma.

\item Either part $v$ or part $y$ contains some number of multiple alphabet symbols; that is, either $v$ or $y$ contains both \texttt{a}s and \texttt{b}s, or both \texttt{b}s and \texttt{c}s.

In this case, pumping $v$ and $y$ once to obtain the word $uv^{2}xy^{2}z$ results in the word containing symbols out of order. This violates the third condition of the pumping lemma.
\end{enumerate}

In all cases, one of the conditions of the pumping lemma is violated. As a consequence, the language cannot be context-free.
\end{example}

By a similar line of reasoning, we can show that the language
\begin{equation*}
L_{\text{a}=\text{c}/\text{b}=\text{d}} = \{\texttt{a}^{n}\texttt{b}^{m}\texttt{c}^{n}\texttt{d}^{m} \mid m, n \geq 1\}
\end{equation*}
is non-context-free, since there's no way for us to separate the counts of \texttt{a}s/\texttt{c}s and \texttt{b}s/\texttt{d}s using only one stack.

Recall that, before, we used the pumping lemma for regular languages to show that the language $L_{\text{pal}} = \{ww^{\text{R}} \mid w \in \Sigma^{*}\}$ was nonregular. We can easily show that the language of palindromes is context-free. However, suppose we modify the language of palindromes so that the reversed occurrence of the word $w$ is instead just a repetition of $w$. This ``doubled" language has an almost identical structure to the language of palindromes, but it happens to be non-context-free!

\begin{example}
Let $\Sigma = \{\texttt{a}, \texttt{b}\}$, and consider the language
\begin{equation*}
L_{\text{double}} = \{ww \mid w \in \Sigma^{*}\}.
\end{equation*}
We will use the pumping lemma to show that this language is non-context-free.

Assume by way of contradiction that the language is context-free, and let $p$ denote the pumping constant given by the pumping lemma. We choose the word $s = \texttt{a}^{p}\texttt{b}^{p}\texttt{a}^{p}\texttt{b}^{p}$. Clearly, $s \in L_{\text{double}}$ and $|s| \geq p$. Thus, there exists a decomposition $s = uvxyz$ satisfying the three conditions of the pumping lemma.

Observe that the second condition of the pumping lemma requires that $|vxy| \leq p$. Here, we will consider two cases, depending on the contents of the middle portion $vxy$ of the word $s$:
\begin{enumerate}
\item If $vxy$ occurs entirely within the first half of $s$ (that is, within the first occurrence of $\texttt{a}^{p}\texttt{b}^{p}$), then as a consequence of the fact that $|vxy| \leq p$, we must have one of the following subcases: $vxy$ contains all \texttt{a}s, $vxy$ contains all \texttt{b}s, or $vxy$ contains both \texttt{a}s and \texttt{b}s, where all \texttt{a}s occur before \texttt{b}s.

In any of these three subcases, pumping $v$ and $y$ once to obtain the word $uv^{2}xy^{2}z$ results in the first half of the word differing from the second half of the word. We can make an analogous argument if $vxy$ occurs entirely in the second half of $s$. This violates the third condition of the pumping lemma.

\item If $vxy$ straddles both halves of $s$, then $v$ and $y$ must contain different symbols as a consequence of the fact that $|vxy| \leq p$. Pumping $v$ and $y$ down to obtain the word $uv^{0}xy^{0}z = uxz$ results in the first half containing fewer \texttt{b}s than the second half, and the second half containing fewer \texttt{a}s than the first half. This violates the third condition of the pumping lemma.
\end{enumerate}

In all cases, one of the conditions of the pumping lemma is violated. As a consequence, the language cannot be context-free.
\end{example}

\futuresubsection{Ogden's Lemma}

\begin{construction}
Sometimes the pumping lemma fails in that non-context-free languages may satisfy the lemma's conditions (see, e.g., \citet{Wise1976StrongPumpingLemmaCFLs} and \citet{Horvath1978LanguagesSatisfyingBarHillel}). Ogden's lemma~\citeyearpar{Ogden1968ProvingInherentAmbiguity} is a stronger formulation of the pumping lemma and, although it is still imperfect (see, e.g., \citet{Boasson1978LanguagesSatisfyingOgdensLemma}, \citet{BaderMoura1982GeneralizationOgdensLemma}, and \citet{Kracht2004TooManyLanguagesOgden}), it allows us to establish non-context-freeness for more languages as well as to prove inherent ambiguity rather more easily.
\end{construction}
%\advancedsection{Subfamilies of Context-Free Languages}\label{sec:subfamilies}

\begin{construction}
Eventually, I'll get around to writing this section about subfamilies of context-free languages. But not today.
\end{construction}

\subsection{Linear Languages}

\subsection{Unary Context-Free Languages}

\subsection{Deterministic Context-Free Languages}

\subsubsection*{Summary}

At this point, let's now revisit the diagram we introduced in Figure~\ref{fig:chomskyregular} and add to it the class of context-free languages. On the one hand, we know that all of the languages we previously showed to be nonregular belong to our new context-free class, but on the other hand, we also now know that there exist languages that aren't context-free. Therefore, sadly, our diagram is still incomplete, but the end is in sight: we will take care of the final classes in the next chapter.

\begin{figure}[h]
\centering
\begin{tikzpicture}
\draw[draw=\thirdcolour, fill=\thirdcolour, thick, rounded corners, shift={(2,1)}] (-4.25,-2.6) rectangle (4.25,3.4);
\draw[draw=\fourthcolour, fill=\fourthcolour, thick, rounded corners, shift={(2,1)}] (-3,-2) rectangle (3,2);
\draw[draw=\fifthcolour, fill=\fifthcolour, thick, rounded corners, shift={(2,1)}] (-1.75,-1.4) rectangle (1.75,0.6);

\node[color=\maincolour] at (2,1.15) {Finite Languages};
\node[color=\maincolour] at (2,0) {\textit{Acyclic DFAs}};
\node[color=black] at (1,0.6) {$\{\epsilon\}$};
\node[color=black] at (2,0.6) {$\{a\}$};
\node[color=black] at (3,0.6) {$\emptyset$};

\node[color=\maincolour] at (2,2.6) {Regular Languages};
\node[color=\maincolour] at (2,-0.667) {\textit{Finite Automata}};
\node[color=black] at (-0.35,1.25) {$\Sigma^{*}$};
\node[color=black] at (0.25,2) {$\texttt{a} \cup \texttt{ba}$};
\node[color=black] at (2,2) {$\{\texttt{a}, \texttt{b}\}^{*}\texttt{c}$};
\node[color=black] at (3.75,2) {$\texttt{01}^{*} \cup \texttt{1}$};
\node[color=black] at (4.35,1.25) {$\texttt{a}^{n}$};

\node[color=\maincolour] at (2,4) {Context-Free Languages};
\node[color=\maincolour] at (2,-1.3) {\textit{Pushdown Automata}};
\node at (-1.6,1.5) {$w\overline{w}$};
\node at (-1.6,2.5) {$L_{\texttt{()}}$};
\node at (-0.75,3.5) {$L_{a>b}$};
\node at (1.1,3.5) {$\texttt{a}^{n}\texttt{b}^{n}$};
\node at (2.9,3.5) {$\texttt{a}^{i}\texttt{\#}\texttt{b}^{i}$};
\node at (4.75,3.5) {$ww^{\text{R}}$};
\node at (5.65,2.5) {$\texttt{a}^{i}\texttt{b}^{j}\texttt{c}^{k}$};
\node at (5.62,2) {{\footnotesize$(i=j \text{ or}$}};
\node at (5.65,1.65) {{\footnotesize$j=k)$}};

\node[color=\maincolour] at (-0.75,4.85) {$ww$};
\node[color=\maincolour] at (2,4.85) {$\texttt{a}^{n}\texttt{b}^{n}\texttt{c}^{n}$};
\node[color=\maincolour] at (4.75,4.85) {$\texttt{a}^{n}\texttt{b}^{m}\texttt{c}^{n}\texttt{d}^{m}$};
\end{tikzpicture}
\caption{The hierarchy of language classes and models of computation, as we know it currently}
\label{fig:chomskycontextfree}
\end{figure}

\unnumberedsection{Chapter Notes}

\firstwords{A comprehensive early survey} of the theory of context-free languages can be found in the book by \citet{Ginsburg1966MathematicalTheoryCFLs}.

\begin{enumerate}
\item[\ref{sec:contextfreegrammars}.] The study of context-free grammars has its origins in the more general study of \emph{phrase structure grammars}, which are a means of performing \emph{immediate constituent analysis}; that is, dividing up a sentence into constituent parts, much like we did with the parse tree in Figure~\ref{fig:parsetree}. In linguistics, immediate constituent analysis dates back to the work of \citet{Bloomfield1933Language}, and attempts to formalize the notion were made by \citet{Harris1946MorphemeUtterance} and \citet{Wells1947ImmediateConstituents}. One can argue that the point at which linguistics and mathematics merged began with the work of \citet{Chomsky1956ThreeModels}, which was inspired by the earlier \emph{canonical systems} of \citet{Post1943FormalReductionsCombinatorial} and the \emph{string rewriting systems} of \citet{Thue1914Veranderungen}.

The name ``context-free" seems to be due to \citet{Chomsky1959NotePhraseStructure}, although this terminology took some time to catch on: Chomsky previously referred to context-free grammars as ``type 2 grammars" \citeyearpar{Chomsky1959FormalPropertiesGrammars}, while \citet*{BarHillel1961FormalPropertiesPhraseStructureGrammars} called them ``simple phrase structure grammars" and \citet{GinsburgRice1962LanguagesRelatedALGOL} referred to the languages generated by such grammars as ``definable sets".

\citet{Parikh1961LanguageGeneratingDevices} was the first to demonstrate the existence of inherently ambiguous context-free grammars; specifically, he showed that no unambiguous grammar exists for the context-free language
\begin{equation*}
\{\texttt{a}^{n}\texttt{b}^{m}\texttt{a}^{n'}\texttt{b}^{m} \mid m, n, n' \geq 1\} \cup \{\texttt{a}^{n}\texttt{b}^{m}\texttt{a}^{n}\texttt{b}^{m'} \mid m, m', n \geq 1\}.
\end{equation*}

As we noted, Chomsky normal form was introduced by \citet{Chomsky1959FormalPropertiesGrammars}. In fact, Chomsky proved an even stronger result in his paper: every context-free language has a context-free grammar where not only do all rules take the form $A \rightarrow a$ or $A \rightarrow BC$, where $A, B, C \in V$, $a \in \Sigma$, and $B \neq C$, but also if the grammar has rules of the form $A \rightarrow \alpha_{1}B\alpha_{2}$ and $A \rightarrow \gamma_{1}B\gamma_{2}$, then either $\alpha_{1} = \gamma_{1} = \epsilon$ or $\alpha_{2} = \gamma_{2} = \epsilon$.

\citet{LangeLeiss2009EfficientCNF} observed a connection between the order in which steps are applied to convert a grammar to Chomsky normal form and the size of the resultant grammar, where size is measured in terms of the number of symbols needed to write the grammar. If we denote the size of the original grammar by $|G|$, a suboptimal ordering of steps (specifically, where \textbf{DEL} comes before \textbf{BIN}) produces a resultant grammar having a size of $2^{2|G|}$ in the worst case, while the order in which we apply the steps here produces a grammar of size $|G|^{2}$ in the worst case. It turns out we cannot do better than this quadratic factor, which is incurred via the \textbf{UNIT} step.

\item[\ref{sec:pushdownautomata}.] The earliest mention of a machine that uses pushdown storage to perform a computation seems to be in an article by \citet*{Burks1954AnalysisLogicalMachine}, where the authors give an algorithm to check the well-formedness of a parenthesis-free Boolean formula; that is, a formula written in the Polish notation of \citet{Lukasiewicz1929MathematicalLogic}. A more explicit construction of a machine using pushdown storage was given by \citet{NewellShaw1957ProgrammingLogicTheoryMachine}, who described a machine that manages memory usage by keeping addresses of free memory locations on an ``available-space list". The addresses of recently freed memory locations are pushed to the front of this list, and memory needs are handled by popping addresses, again, from the front of the list. (On an unrelated note, this so-called ``Logic Theorist" machine described by Newell, Shaw, and Herbert A.\ Simon was the first description of a computer that makes use of automated reasoning, and is considered by some to be the first-ever implementation of artificial intelligence.)

The term ``pushdown store" itself is due to \citet{Oettinger1961AutomaticSyntacticAnalysis}. The first use of pushdown storage pertaining to the class of context-free languages seems to be due to \citet{Chomsky1962CFGsPushdownStorage}.

We observed that nondeterministic pushdown automata are more powerful than deterministic pushdown automata in that the nondeterministic model recognizes more languages. The study of deterministic context-free languages was initiated by \citet{Ginsburg1966DCFLs}.

\item[\ref{sec:equivalenceofmodelscontextfree}.] The equivalence in recognition power between context-free grammars and pushdown automata was established by \citet{Chomsky1962CFGsPushdownStorage} and \citet{ChomskySchutzenberger1963ContextFree}, and independently by \citet{Evey1963ApplicationsPushdownStore, Evey1963PhDThesis}. However, each of their approaches to showing equivalence is rather more complex than the one presented here.

\item[\ref{sec:closurepropertiescontextfree}.] Closure of the class of context-free languages under union, as well as non-closure under intersection and complement, was established by~\citet{Scheinberg1960BooleanPropertiesCFLs}.

\item[\ref{sec:noncontextfree}.] The pumping lemma for context-free languages is due to~\citet*{BarHillel1961FormalPropertiesPhraseStructureGrammars}.

We know from the previous chapter that the language of binary representations of prime numbers, $L_{\text{primes}2}$, is nonregular. \citet{HartmanisShank1968RecognitionPrimesAutomata} further established that this language cannot be accepted by any pushdown automaton, and therefore it is also non-context-free. The same result, proved via a different approach, was published independently by \citet{Schutzenberger1968RemarkAcceptableSets}.
\end{enumerate}